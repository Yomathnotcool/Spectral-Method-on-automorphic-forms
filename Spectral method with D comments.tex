\documentclass[11pt,reqno]{amsart}
\usepackage{mathrsfs}
\usepackage{bbm}
\usepackage{tikz}
\usepackage{amsfonts}
\usepackage{amssymb,amsmath,amscd,amsthm,amsfonts}
\usepackage{hyperref}
\usepackage{color}
\usepackage{xcolor}
\usepackage[pdflatex%dvipdfm,  %pdflatex,pdftexÕâÀï¾ö¶¨ÔËÐÐÎļþµÄ·½Ê½²»Í¬
            pdfstartview=FitH,
            CJKbookmarks=true,
            bookmarksnumbered=true,
            bookmarksopen=true,
            colorlinks=true, %×¢Ê͵ô´ËÏîÔò½»²æÒýÓÃΪ²ÊÉ«±ß¿ò(½«colorlinksºÍpdfborderͬʱעÊ͵ô)
            pdfborder=001,   %×¢Ê͵ô´ËÏîÔò½»²æÒýÓÃΪ²ÊÉ«±ß¿ò
            linkcolor=green,
            anchorcolor=green,
            citecolor=green
            ]{}
\pagestyle{plain}
\marginparwidth    0pt
\oddsidemargin     0pt
\evensidemargin    0pt
\topmargin         0pt
\textheight        21cm
\textwidth         17cm

\usepackage[all]{xy}


\usepackage{CJK}

\renewcommand{\le}{\left}
\newcommand{\ri}{\right}
\newcommand{\bqa}{\begin{equation}}
\newcommand{\eqa}{\end{equation}}
\newcommand{\bea}{\begin{eqnarray}}
\newcommand{\eea}{\end{eqnarray}}
\newcommand{\bna}{\begin{eqnarray*}}
\newcommand{\ena}{\end{eqnarray*}}
\newcommand{\bma}{\begin{pmatrix}}
\newcommand{\ema}{\end{pmatrix}}
\newcommand{\ep}{\delta}
\newcommand{\ve}{\varepsilon}
\newcommand{\vp}{\varphi}
\newcommand{\pal}{\partial}
\newcommand{\mk}{\mathfrak}
\newcommand{\ml}{\mathcal}
\newcommand{\diff }[1]{{\rm d}#1}
\newcommand{\mmod}[1]{\;{\rm mod}\;#1}
\newcommand{\mpmod}[1]{\;({\rm mod}\;#1)}
\newcommand{\sumstar}[1]{\underset{#1}{\sum\nolimits^*}}

\def\fxx{\backslash}
\def\riaw{\rightarrow}
\def\leaw{\leftarrow}
\def\cl{{\mathcal L}}
\def\ct{{\mathcal T}}
\def\bz{{\mathbb Z}}
\def\br{{\mathbb R}}
\def\bc{{\mathbb C}}
\def\bh{{\mathbb H}}
\def\sl2z{SL(2,\bz)}
\def\psl2z{PSL(2,\bz)}
\def\sltr{SL(2,\br)}
\def\gl2r{GL(2,\br)}
\def\glnr{GL(n,\br)}
\def\rme{{\rm e}}
\def\rmi{{\rm i}}
\def\re{{\mathrm{Re}}}
\def\im{{\mathrm{Im}}}
\def\res{{\rm Res}}
\def\m{\mathfrak}
\def\C{\mathbb{C}}
\def\B{\mathcal{B}}
\def\A{\mathbb{A}}
\def\R{\mathbb{R}}
\def\Q{\mathbb{Q}}
\def\Z{\mathbb{Z}}
\def\g{\mathfrak{g}}
\def\sign{\mathrm {sign}}
\def\vol{\mathrm {Vol}}


\newtheorem{lemma}{Lemma}[section]
\newtheorem{thm}[lemma]{Theorem}
\newtheorem{cor}{Corollary}
\newtheorem{prop}[lemma]{Proposition}
\newtheorem{conj}[lemma]{Conjecture}
\theoremstyle{definition}
\newtheorem{definition}[lemma]{Definition}
\newtheorem{remark}{Remark}
\newtheorem{exam}{Example}
\newtheorem*{ques}{Question}

\renewcommand{\theequation}{\arabic{section}.\arabic{equation}}
\newcommand{\bit}{\begin{itemize}}
\newcommand{\eit}{\end{itemize}}
\newcommand\Prefix[3]{\vphantom{#3}#1#2#3}

\newcommand{\red}[1]{\textcolor{red}{{#1}}}

\title{Spectral Method on automorphic forms}




\author{Deng Zhiyuan}
\date{\today}


\keywords{Petersson trace formula, Kuznetsov trace formula, Maass newforms, Central $L$-values.}






\begin{document}
\maketitle

\section{\textbf{The  M\"obius transform on the Riemann sphere}}
Let $\hat\C=\C\cup\{\infty\}$ be the Riemann sphere.
It can be realized as $S^2$ in $\R^3$, or as $\mathbb P^1(\C)$.

For $g\in SL_2(\C)$, the M\"obious transform  is defined by
\bna
g\mapsto g.z=\frac{az+b}{cz+d},\quad g=\bma a&b\\c&d\ema.
\ena
This is a bilinear transformation. One has
\bna
z=\infty\mapsto\frac{a}{c},\qquad z=-\frac{d}{c}\mapsto\infty.
\ena
\begin{prop}
The M\"obius transforms form a group of conformal maps of the Riemann sphere.
\end{prop}


Consider the fixed points of $g=\bma a&b\\c&d\ema\in SL_2(\C)=\bigg\{\bma a&b\\c&d\ema: a,b,c,d\in \C,\ and\ ad-bc=1\bigg\}$,
\bna
\frac{az+b}{cz+d}=z.
\ena
One has
\bna
cz^2+(d-a)z-b=0.
\ena
The solutions are
\bna
z_1,z_2=\frac{a-d\pm\sqrt{(d+a)^2-4}}{2c}
\ena
and we know that  $a+d=\mathrm{Tr}(g)\neq \pm 2$ then $z_1\neq z_2$.

We assume that $g$ has two fixed different points $z_1\neq z_2$.
Let  $A=\bma 1&-z_1\\ 1&-z_2\ema$ so that $
z\mapsto A.z$
maps
\bna
z_1\mapsto 0,\qquad z_2\mapsto\infty
\ena
which are the south pole and the north pole in $S^2$, respectively.
Then we have
\bna
\xymatrix{
\hat \C\ar[rr]^{g}\ar[d]^{A}&&\hat \C\ar[d]^{A}\\
\hat \C\ar[rr]^{\tilde g=AgA^{-1}}&&\hat \C
}\qquad
\xymatrix{
z_1,z_2\ar[rr]^{g}&&z_1,z_2\ar[d]^{A}\\
0,\infty\ar[u]^{A^{-1}}\ar[rr]^{\tilde g=AgA^{-1}}&&0,\infty
}.
\ena
Here
\bna
\tilde g=AgA^{-1}=\bma\lambda^{1/2}&\\&\lambda^{-1/2}\ema
\ena
with
\bna
\tilde g.0=0,\quad \tilde g.\infty=\infty,\quad\tilde g.z=\lambda z
\ena
where $\lambda^{1/2}$ and $\lambda^{-1/2}$ are eigenvalues of $\tilde g$ (hence of $g$).
\bit
\item
For $g$ with fixed points $z_1$ and $z_2$,
let $\mathcal H_{z_1,z_2}$ be circles  passing through $z_1$ and $z_2$,
called \underline{hyperbolic pencil};
let $\mathcal E_{z_1,z_2}$ be circles orthogonal to those circles in $\mathcal H_{z_1,z_2}$,
called \underline{elliptic pencil}.
\item
Correspondingly, for $\tilde g$, the hyperbolic pencil $\mathcal H_{0,\infty}$
are circles passing through  the north pole and the south pole, i.e.
\underline{longitude lines};
the elliptic pencil $\mathcal E_{0,\infty}$ are just \underline{latitude lines}
\eit

We give the classification of $g=\bma a&b\\c&d\ema$ as follows.
\bit
\item [1.]elliptic.\\
 If $tr(g)=a+d=\lambda^{1/2}+\lambda^{-1/2}$ satisfies
\bna
-2<tr(g)<2.
\ena
Then we know $\lambda$ must be of the form $\lambda=e^{i\theta}$, $0<\theta<2\pi$.
In this case,
\bna
\tilde g:z\mapsto \lambda z=e^{i\theta}z
\ena
Besides the fixed points  at $0$ and $\infty$, $\tilde g$ moves points along the latitude lines,
or equivalently,
$g$ moves points along the \underline{elliptic pencil}.
Such $g$ is called elliptic.
\item[2.]hyperbolic.\\ If $tr(g)=a+d=\lambda^{1/2}+\lambda^{-1/2}$ satisfies
\bna
tr(g)>2,\quad\mbox{or}\quad tr(g)<-2,
\ena
then we know $\lambda$ must be real and $\lambda=r\neq 1$.
In this case,
\bna
\tilde g:z\mapsto \lambda z=rz
\ena
which moves points along the longitude lines;
or equivalently, $g$ moves points along the \underline{hyperbolic pencil}.
such $g$ is called \underline{hyperbolic}.
\item[3.]loxodromic.\\ If $tr(g)=a+d=\lambda^{1/2}+\lambda^{-1/2}$ is not real, then $\lambda$ is of the form
\bna
\lambda=re^{i\theta},\quad r\neq 1,0<\theta<2\pi.
\ena
In this case, $\tilde g$  and hence $g$ is called loxodromic.
\item[4.]parabolic.\\ If $tr(g)=a+d=\lambda^{1/2}+\lambda^{-1/2}$ and $g\neq id$,
 we know that $g$ has only one fixed point,
and  we can choose $A$ so that
\bna
\tilde g =AgA^{-1}=\bma 1&1\\&1\ema.
\ena
in this case, $\tilde g$ has the only fixed point $\infty$.
Such $g$ is called \underline{parabolic}.
\eit
\begin{remark}We consider the finite order move, i.e. those $g$ with $g^n=id$ for some $n\in\mathbb N$.
Note that $id=g^n= (AgA^{-1})^n=\tilde g^n$.
By the classification above,
\bna
\tilde g^n.z=\lambda^n.z=z,\quad\forall z\in\hat \C
\ena
if and only if $\lambda=e^{\alpha 2\pi i}$ for some $\alpha\in \Q$.
Thus the finite orders are those elliptic ones with the rotation angle being rational multiple  of $2\pi i$.
\end{remark}

\section{\textbf{The hyperbolic geometry - The Poincare Upper half plane with rectangular coordinate}}
One of the realization of the hyperbolic plane is the Poincare upper half plane with
the action of $SL_2(\R)$.
Let $$\mathfrak h=\{z=x+iy,\quad x,y\in \R, y>0\}$$ be the Poincare Upper half plane.
For $g=\bma a&b\\c&d\ema\in SL_2(\R)$ and $z\in \mathfrak h$,
we have the map
\bna
z\mapsto g.z=\frac{az+b}{cz+d}.
\ena
which makes $\mk h$ to be a $SL_2(\R)$-space. the action of $SL_2(\R)$ on $\mk h$
 is transitive and the stablizer at $z=i$ is
\bna
\{g\in SL_2(\R),\quad g.i=i\}=SO(2)
\ena
so that one has $G$-space isomorphism
\bna
SL_2(\R)/SO(2)\rightarrow \mathfrak h,\quad gSO(2)\mapsto g.i.
\ena

By Iwasawa decomposition,  each $g\in SL_2(\R)$ can be uniquely expressed as
\bna
g=\bma 1&x\\&1\ema\bma y^{1/2}\\&y^{-1/2}\ema\kappa_\theta
\ena
with $\kappa_\theta=\bma\cos\theta&\sin\theta\\-\sin\theta&\cos\theta\ema\in SO(2)$.
Thus we can take the representative elements in $g SO(2)$ as
\bna
\bma 1&x\\&1\ema\bma y^{1/2}\\&y^{-1/2}\ema
\ena
so that one has the identification of $SL_2(\R)$-spaces,
\bna
SL_2(\R)/SO(2)\rightarrow\mk h,\quad
\bma 1&x\\&1\ema\bma y^{1/2}\\&y^{-1/2}\ema\mapsto \bma 1&x\\&1\ema\bma y^{1/2}\\&y^{-1/2}\ema.i=x+iy=z
\ena
\begin{remark}
Note that for $g\in SL_2(\R)$, $g.z=(-g).z$.
Thus it is natural to consider the action of $PSL_2(\R)=SL_2(\R)/\{\pm 1\}$.
If one consider $PSL_2(\R)$, the isotropic subgroup
at $i$ is $SO(2)/\{\pm id\}$.
\end{remark}
\subsection{Hyperbolic arc density}
We recall the definition of the arc density and geodesic lines.
Let $\Omega\subset \widehat C$ be a region with arc density $\rho(z)$.
for $\gamma(z)$ a curve  in $\Omega$, the length of $\gamma$ is defined as
\bna
d(\gamma)=\int_{\gamma}\rho(z)|dz|
\ena
Here
\bna
z=x+iy,\quad dz=(1,i)\bma dx\\ dy\ema=dx+idy,\quad
|dz|=\sqrt{dx^2+dy^2}.
\ena
For $p,q\in\Omega$, the distance  is defined to be
\bna
d_\Omega(p,q)=\inf_{\gamma\atop{\gamma(0)=p,\gamma(1)=q}}d(\gamma).
\ena
\medskip

The hyperbolic line density $\rho(z)|dz|$ should be invariant under
conformal homeomorphisms, i.e.  for any $g=\bma a&b\\c&d\ema\in SL_2(\R)$, one should has
\bea
\rho(g.z)|d(g.z)|=\rho(z)|dz|\label{arc-relation}
\eea
Here
\bna
d(g.z)=\frac{d}{dz}\left(\frac{az+b}{cz+d}\right)=\frac{1}{(cz+d)^2}dz
\ena
and thus $\rho(z)$ satisfies
\bea
\frac{\rho(g.z)}{|cz+d|^2}=\rho(z).\label{temp-1}
\eea

Since $SL_2(\R)$ acts transitively on $\mk h$,
the value of $\rho(z)$ at $z\in \mk h$ is determined by
\eqref{arc-relation}.
So we assume that $\rho(i)=1$.
For
\bna
z=x+iy=\bma y^{1/2}&y^{-1/2}x\\&y^{-1/2}\ema.i,
\ena
by \eqref{temp-1},
\bna
\rho(z)=\rho\left(\bma y^{1/2}&y^{-1/2}x\\&y^{-1/2}\ema.i\right)
=\frac{\rho(i)}{|y|}=\frac{1}{|y|}.
\ena
Thus we conclude that the hyperbolic line density is just
\bna
ds=\rho(z)|dz|=\frac{|dz|}{|y|}=\frac{\sqrt{dx^2+dy^2}}{|y|}
\ena
and thus
\bea
ds^2=\frac{dx^2+dy^2}{y^2}.\label{hyperbolic-line-density}
\eea
\subsection{The geodesics and distance function}
Let $z_1=is_1$ and $z_2=is_2$ be two different points on $y$-axis.
Let $\gamma$ be the line on $y$-axis with start point $z_1$ and end points $z_2$,
i.e.
\bna
\gamma=\gamma(t)=tz_2+ (1-t)z_1,\qquad 0\leq t\leq 1.
\ena
We have
\bna
d(\gamma)&=&\int_{\gamma}\rho(z)|dz|=\int_{0}^1\frac{1}{|\im(tz_2+ (1-t)z_1)|}
|z_2-z_1||dt|\\
&=&\int_0^1\frac{|s_2-s_1|}{|t(s_2-s_1)+s_1|}dt=\left|\log (t(s_2-s_1)-+s_1)\right|
_{0}^1\\
&=&|\log\frac{s_2}{s_1}|.
\ena

Next, we prove that
\bna
d(is_1,is_2)=\left|\log\frac{s_2}{s_1}\right|,
\ena
or equivalently, $y$-axis is a geodesic.

In fact, assume that
\bna
\gamma(t)=x(t)+iy(t),\qquad  x(0)=x(1)=0,y(0)=s_1,y(1)=s_2
\ena
is any differential line with start point $is_1$ and end point $is_2$.
One has
\bna
d(\gamma)&=&\int_0^1\frac{\sqrt{x'(t)^2+y'(t)^2}}{|y(t)|}dt
\geq \int_0^1\left|\frac{y'(t)}{y(t)}\right|dt\\
&\geq& \left|\int_0^1\frac{y'(t)}{y(t)}dt\right|\\
&=&\left|\log\frac{s_2}{s_1}\right|.
\ena

Next, since $g\in SL_2(\R)$ acts on $\mk h$ as conformal and isometry translations,
it maps geodesics into geodesics. So we have the following proposition
\begin{prop}
The geodesics on $\mk h$ are those lines and semi-circles orthogonal to $x$-axis.
For any points $z,w\in \mk h$,
\bna
d(z,w)=\log\frac{|z-\overline w|+|z-w|}{|z-\overline w|-|z-w|}.
\ena
\end{prop}
\begin{proof}
We have obtained the distance for points on $y$-axis.
To establish the distance function on general  points,
we need only to find some $g\in SL_2(\R)$ so that
\bna
z=g.i,\qquad w=g.(is)
\ena
so that $d(z,w)=d(i,is)=|\log s|$.

Assume that $z=x+iy$ and $w=u+iv$. The idea is the following.
On taking $g_z=\bma y^{1/2}& y^{-1/2}x\\&y^{-1/2}\ema$ and $g_w=
\bma v^{1/2}& v^{-1/2}u\\&v^{-1/2}\ema$, one has
\bna
z=g_z.i,\quad w=g_w.i=g_z.(g_z^{-1}g_w.i).
\ena
We can use $KAK$ decomposition to write
\bna
g_z^{-1}g_w= \kappa_\varphi.\bma s^{\frac{1}{2}}&\\&s^{-\frac{1}{2}}\ema \kappa_\theta
\ena
so that
\bna
z&=&g_z.i\\
&=&g_z\kappa_\varphi.i,\\
w&=&g_z\kappa_\varphi.\left(\bma s^{\frac{1}{2}}&\\&s^{-\frac{1}{2}}\ema \kappa_\theta.i\right)\\
&=&g_z\kappa_\varphi.(is)
\ena
and thus
\bna
d(z,w)=d(i,is)=|\log s|.
\ena
\end{proof}
\begin{remark}We define
\bea
u(z,w):=\frac{\cosh(d(z,w))-1}{2}=\frac{|z-w|^2}{4\im z\im w}\label{u-function-conformal-invariance}
\eea
which is also a conformal invariance. It is used in the invariant automorphic kernel.
\end{remark}
\subsection{The invariant measure and the Laplacian operator}
The invariant measure and the Laplacian-Beltrami operator are obtained
via group theory,
\bna
 d\mu(z):=\frac{dxdy}{y^2},
 \qquad \Delta:=-y^2
 \left(\frac{\partial ^2}{\partial x^2}+\frac{\partial ^2}{\partial y^2}\right).
\ena

Consider $L^2(\mk h)$ with the inner product
\bna
\langle f_1,f_2\rangle:=\int_{\mathfrak h}f_1(z)\overline{f_2(z)}d\mu(z).
\ena
We are interested in the spectrum of $L^2(\mathfrak h)$.
The Laplacian-Beltrami operator is important for the decomposition of $L^2(\mk h)$.
In fact, $C_c^\infty(\mk h)$
is a dense subspace in $L^2(\mk h)$ with respect to Fr\'echet topology.
Roughly speaking, elements in $L^2(\mk h)$ can be expressed as `limit' of
 a sequence of
functions in $C_c^\infty(\mk h)$.
The action of $\Delta$ on such sequence is also a sequence.
So we can extend $\Delta$ to be an operator on $L^2(\mk h)$.

\begin{prop}$\Delta$ is defined on $C_c^\infty(\mk h)$,
which is a dense subspace of $L^2(\mk h)$.
It is extended to a positive definite, unbounded, self-adjoint operators on $L^2(\mk h)$
and satisfies
\bna
\langle \Delta f,f\rangle\geq\frac{1}{4}\langle f,f\rangle.
\ena
\end{prop}
\begin{proof}
Set $\Delta^e=\frac{\partial^2}{\partial x^2}+\frac{\partial^2}{\partial y^2}$.
Let $d$ be the exterior derivative, which takes $1$-forms  to $2$-forms.
We have
\bna
dh&=&\frac{\partial h}{\partial x} dx +\frac{\partial h}{\partial y}dy,\\
d\left(h_1dx+h_2dy\right)&=&d(h_1)\wedge dx+ d(h_2)\wedge dy\\
&=&\frac{\partial h_1}{\partial y}dy\wedge dx +\frac{\partial h_2}{\partial x}
dx\wedge dy\\
&=&\left(
\frac{\partial h_2}{\partial x}
-\frac{\partial h_1}{\partial y}
\right) dx\wedge dy.
\ena
The Green's identity assert that
\bna
\int_{\Omega}
\left(\frac{\partial h_2}{\partial x}
-\frac{\partial h_1}{\partial y}
\right) dx\wedge dy=
\int_{\Omega} d\left(h_1dx+h_2dy\right)
=\int_{\partial\Omega}h_1dx+h_2dy
\ena

Let $f$ and $g$ be two smooth functions defined in n.b.d. of a bounded region $\Omega$,
whose boundary is a smooth curve (or union of smooth curve) $\partial \Omega$.
Note that
\bna
(\overline g\Delta^ef )dx\wedge dy&=&\left(\overline g\frac{\partial ^2f}{\partial x^2}
+\overline g\frac{\partial ^2f}{\partial y^2}\right)dx\wedge dy\\
&=&\overline g d\left(\frac{\partial f}{\partial x}\right)\wedge dy-\overline g
\left(\frac{\partial f}{\partial y}\right)\wedge dx\\
&=&d\left(\overline g
\frac{\partial f}{\partial x}\right)\wedge dy-
d\left(\overline g\frac{\partial f}{\partial y}\right)\wedge dx
-
\left(\frac{\partial f}{\partial x}d(\overline g) \wedge dy
-
\frac{\partial f}{\partial y} d(\overline g)\wedge dx\right).
\ena
Thus
\bna
-\langle \Delta f,g\rangle&=&\int_{\mk h}\overline{g} \Delta^e fdx\wedge dy
\\
&=&
\int_{\Omega} d\left(\overline g
\frac{\partial f}{\partial x}\right)\wedge dy-
d\left(\overline g\frac{\partial f}{\partial y}\right)\wedge dx
-
\left(\int_{\Omega} \frac{\partial f}{\partial x}\frac{\partial \overline g}{\partial x}dx\wedge dy
+
\frac{\partial f}{\partial y}\frac{\partial \overline g}{\partial y}dx\wedge dy\right)\\
&=&
\int_{\partial\Omega} \left(\overline g
\frac{\partial f}{\partial x}\right)dy-
\left(\overline g\frac{\partial f}{\partial y}\right)dx
-
\left(\int_{\Omega} \frac{\partial f}{\partial x}\frac{\partial \overline g}{\partial x}dx\wedge dy
+
\frac{\partial f}{\partial y}\frac{\partial \overline g}{\partial y}dx\wedge dy\right)
\ena
Note that $\partial \Omega$ is the boundary enclosed the support of $f$ and $g$ and thus the integrand vanishes, which gives
\bna
-\langle \Delta f,g\rangle=-\int_{\Omega}
\left(\frac{\partial f}{\partial x}\frac{\partial \overline g}{\partial x}+
\frac{\partial f}{\partial y}\frac{\partial \overline{g}}{\partial y}\right)dx\wedge dy
=-\int_{\Omega}\nabla f\cdot \nabla\overline g dx\wedge dy.
\ena
Thus one should has $\langle\Delta f,g\rangle=\langle f,\Delta g\rangle$,
i.e. $\Delta$ is self-adjoint,  and is positive definite.

Moreover, for $f\in C_c^\infty(\mk h)$,
\bna
\langle \Delta f,f\rangle&=&
\int_{-\infty}^\infty \int_0^\infty
\nabla f\cdot  \overline{\nabla f}dydx\\
&=&
\int_{-\infty}^\infty \int_0^\infty
\left(\left|\frac{\partial f}{\partial x}\right|^2+\left|\frac{\partial f}{\partial y}\right|^2\right)dydx\\
&\geq &
\int_{-\infty}^\infty \int_0^\infty
\left|\frac{\partial f}{\partial y}\right|^2dydx.
\ena
Viewing $f$ as functions in $y$, $f(y)=u(y)+iv(y)$ and
\bna
|f|^2=u^2+v^2,\quad
\frac{\partial f}{\partial y}=
u'+iv',\quad
\left|\frac{\partial f}{\partial y}\right|^2=u'^2+v'^2
\ena
and one needs to show that
\bna
\int_0^\infty (u'^2+v'^2 )dy\geq \frac{1}{4}\int_0^\infty \frac{u^2+v^2}{y^2}dy.
\ena
Note that
\bna
\int_0^\infty \frac{u^2}{y^2}dy=
-\int_0^\infty u^2d\frac{1}{y}
=2\int_0^\infty\frac{u}{y}u'dy\leq 2\left(\int_0^\infty \frac{u^2}{y^2}dy\right)^{1/2}
\left(\int_0^\infty u'^2 dy\right)^{1/2}
\ena
which gives that
\bna
\int_0^\infty\frac{u^2}{y^2}dy\leq 4\int_0^\infty u'^2dy.
\ena
This finishes the proof.
\end{proof}

\subsection{Remark on spectral decomposition}
We have showed that $\Delta$ is a `good' enough  operators on $L^2(\mk h)$.
We hope to build `Fourier analysis' on $L^2(\mk h)$ as eigenfunctions
of $\Delta$.
We recall the spectral decomposition in the simplest cases as follows.

Consider $L^2(\R)$. Eigenfunctions of $\frac{d^2}{dx^2}$
are of the form
\bna
e^{2\pi i y},\quad y\in \R
\ena
and each $\phi\in L^2(\R)$ has spectral decomposition
\bna
\phi(x)=\int_{\R} \widehat\phi(r) e^{2\pi irx}dr
\ena
where
\bna
\widehat\phi(r)=\langle\phi, e^{2\pi ir*}\rangle=\int_{\R}\phi(y) e^{-2\pi iry}dy
\ena
is the Fourier transform of $\phi$.
Here  $e^{2\pi iry}$ are eigenfunctions of $\frac{d^2}{dx^2}$ which is not in $L^2(\R)$.

Consider $L^2(\Z\backslash\R)$, the space of functions on $\R$ with period in $\Z$.
It admits discrete spectrum
\bna
e^{2\pi i mx},\quad m\in \Z.
\ena
which are also eigenfunctions of $\Delta$.
They can be also obtained via the following theory.
Let
\bna
\hat{(\R/\Z)}=\{\psi: \R/\Z\rightarrow S^1,\quad \psi(a+b)=\psi(a)\psi(b)\}
\ena
be the group of complex continuous characters of $\R/\Z$
(all continuous group homomorphisms from $\R$ to $S^1$ with period in $\Z$).
Then $\psi$ is of the form $\psi(x)=e(mx)$ for some $m\in \Z$ and one has
\bna
\hat{(\R/\Z)}\simeq \Z,\qquad  e(m*)\mapsto m.
\ena
Then we can build Fourier analysis as that $\phi\in L^2(\Z\backslash \R)$,
\bna
\phi(x)=\sum_{m\atop{\mbox{\tiny spectral parameter}}}
\frac{\langle\phi,e(m*)\rangle_{\R/\Z}}
{\langle e(m*),e(m*)\rangle_{\R/\Z}}
 e(m x)
\ena
where
\bna
\langle\phi,e(m*)\rangle_{\R/\Z}=\int_{\R/\Z}\phi(y) \overline{e(my)}dy
\ena
is the $m$-th Fourier coefficient of $\phi$.
We refer to GTM186 - Fourier analysis on number field (Ramakrishnan-Valenza)
for much information.



\subsection{Eigenfunctions of $\Delta$}
To build the spectral decomposition of $L^{2}(\mk h)$,
we need to find eigenfunctions of $\Delta$ firstly.

Assume that $f$ is an eigenfunction of $\Delta$ with $\Delta f=\lambda f$.
We can assume that $f(z)=v(x)w(y)$ by separating parameters.
One has
\bna
-y^2\left(v''(x)w(y)+v(x)w''(y)\right)=\lambda v(x)w(y)
\ena
dividing $v(x)w(y)$ on both sides, one has
\bna
\frac{w''(y)}{w(y)}+\frac{\lambda}{y^2}=k= -\frac{v''(x)}{v(x)}
\ena
where $k$ is the separable parameter.

Consider the  differential equation
\bna
k= -\frac{v''(x)}{v(x)},\quad \mbox{or equivalently,}\quad
v''+kv=0.
\ena
It is independent of the eigenvalue $\lambda$.
The characteristic function of the differential equation is
\bna
r^2+k=0,\quad r=\pm \sqrt{-k}
\ena
and thus we have solutions
\bna
e^{\sqrt{-k}x},\quad e^{-\sqrt{-k}x}.
\ena
\underline{The growth condition implies that $k\geq 0$}.
Let $k=4\pi ^2 a^2$. The solutions are
\bna
e^{2\pi ai }\quad\mbox{and}\quad e^{-2\pi a i}.
\ena

Consider another differential equation,
\bna
\frac{w''(y)}{w(y)}+\frac{\lambda}{y^2}=4\pi^2 a^2.
\ena
Assume that $w=y^{1/2}u(y)$ one has
\bna
\frac{dw}{dy}=\frac{u(y)}{2\sqrt{y}}+y^{1/2}u'(y),
\quad\frac{d^2w}{dy^2}=y^{1/2}u''(y)+ \frac{1}{\sqrt{y}}u'-\frac{1}{4y^{3/2}}u(y)
\ena
and thus the differential equation becomes
\bea
y^2u''+yu'+\left((\lambda-\frac{1}{4})-4\pi a^2 y^2\right)u=0.
\label{Bessel-differential-equation}
\eea



\begin{lemma}
Let $\lambda=s(1-s)=\frac{1}{4}+t^2$ with $s=\frac{1}{2}+it$.
The solutions of \eqref{Bessel-differential-equation} are
as follows.
\bit
\item
If $a=0$,
\bna
\frac{y^s+y^{1-s}}{2},\qquad\frac{y^s-y^{1-s}}{2s-1}
\ena
\item If $a\neq 0$,
 solutions of \eqref{Bessel-differential-equation} are
\bna
K_{s-\frac{1}{2}}(2\pi |a|y),\quad I_{s-\frac{1}{2}}(2\pi|a|y).
\ena
with asymptotic formulas as $|z|\rightarrow\infty$
\bna
K_{s-\frac{1}{2}}(z)\sim
\sqrt{\frac{\pi}{2z}} e^{-z},\quad I_{s-\frac{1}{2}}(z)\sim\sqrt{\frac{1}{2\pi z}}e^{z}
\ena
\eit

\end{lemma}
\begin{remark}
For the case $a=0$ and $s\neq 1/2$, $\{y^s, y^{1-s}\}$
are linear dependent solutions in this case;
For $a=0$ and $s=1/2$,
\bna
\{y^{1/2},\quad y^{1/2}\log y\}
\ena
are the independent solutions.
\end{remark}
\begin{proof}
Let $u(y)=F(2\pi |a|y)$ and $z=2\pi|a|y$.
One has
\bna
u'=2\pi|a| F'(2\pi |a|y),\quad
u''=4\pi^2 a^2=F''(2\pi |a|y),
\ena
and thus the differential equation \eqref{Bessel-differential-equation} is of the form
\bna
(2\pi |a|y)^2F''(2\pi|a|y)+ (2\pi|a|y)F'(2\pi|a|y)
+\left(\lambda-\frac{1}{4}-(2\pi|a|y)^2\right)F(2\pi|a|y)=0.
\ena
Note that  $z=2\pi|a|y$. It is
\bna
z^2F''(z)+zF'(z)+\left(\lambda-\frac{1}{4}-z^2\right)F(z)=0
\ena
or equivalently,
\bea
F''(z)+\frac{1}{z}F'(z)-\left(1+\frac{(it)^2}{z^2}\right)F(z)=0
\label{Bessel-real}
\eea
which is a Bessel equation.
The linear independent solutions \eqref{Bessel-real} are $K_{it}(z)$ and $I_{it}(z)$
with the growth condition as above.
We refer to appendix \ref{appendix-Bessel-equ} for detail.
\end{proof}



\begin{prop}
Eigenfunctions of $\Delta$ with eigenvalues
$\lambda=s(1-s)=\frac{1}{4}+t^2$  satisfying moderate growth conditions
are
\bna
y^{1/2+it}=y^{s}\quad\mbox{,} \quad y^{1/2-it}=y^{1-s};
\ena
and
\bna
\sqrt{y}K_{s-\frac{1}{2}}(2\pi|a|y)e(ax),\qquad 0\neq a\in \R.
\ena
\end{prop}

\subsection{Relation with the spherical function
in the Whittaker model of the principle series}
We have proved that most of the eigenfunctions are of the form (in the case $a\neq0$)
\bna
W_s(az)=
2\sqrt{|a|y}K_{s-\frac{1}{2}}(2\pi |a|y)e(ax).
\ena
Here
\bea
W_s(z):= 2\sqrt{y}K_{s-\frac{1}{2}}(2\pi y)e(x)\label{Whittkaer}
\eea
is called the Whittaker function.


The expression \eqref{Whittkaer} is good for explicit calculation and estimation.
However, we need another way to construct the sphercial Whittaker function
associated to the spectral parameter $s$, which is simple and good for generalization
(Not good for explicit calculation and estimation).


Let $s\in \C$ be a spectral parameter.  We define
\bna
I_s(z)=(\im z)^s
\ena
which is eigenfunction of $\Delta$ with eigenvalue $s(1-s)$.
We let
\bna
\psi:N(\R)\rightarrow\C,\quad \bma 1&u\\&1\ema\mapsto e^{2\pi i u}
\ena
be a fixed non-trivial additive character on the unipotent group.
The sphercial Whittaker function associated to $\psi$ is defined by
\bea
\tilde W_s(z)
=\int_{-\infty}^\infty \I_s\left(\bma&-1\\1\ema\bma 1&u\\&1\ema.z\right)\psi(-u)du
\label{spherical-Whittaker}
\eea
for $\im z>0$, and is generalized to the lower half plane via (see (1.27) in Iwaniec's book)
\bna
\tilde W_s(\overline z)=\tilde W_s(z).
\ena

This definition of $\tilde W_s(z)$ is easy to be generalized to the case $SL_n$
(See formula (5.5.1) in Goldfeld's book, {\it automorphic forms and $L$-functions for the group $GL_n(\R)$}.)
However, for the explicit calculation and estimation, one needs much precisely
information on the behaviour of $W_s(z)$ in terms of the so called
generalized Bessel functions.
These have been worked out recently for the case $GL_3$ (V. Blomer, Buttcane)
and for the case $GL_2(\C)$ (Qi Zhi.)
\begin{prop}
For $\tilde W_s(z)$, and $a\neq 0$, we have
\bna
\Delta \tilde W_s(z)&=&s(1-s)\tilde W_s(z),\\
\tilde W_{s}(az)&=&e( ax)W_s(i|a|y)
=
\frac{\pi^{s}}{\Gamma(s)} 2\sqrt{|a|y}K_{s-\frac{1}{2}}(2\pi |a|y)e(ax).
\ena
\end{prop}
\begin{remark}
Note that the $K$-Bessel function satisfies $K_s(y)=K_{-s}(y)$
and $\overline{K_s(y)}=K_{\overline{s}}(y)$ by the integral representation in \eqref{integral-transform-K-Bessel-in-appendix}.
For a general definition of the spherical Whittaker function in
the Whittaker model of the unramified principle series of $PGL_2(\R)$, see appendix \ref{appendix-spherical-whittaker-model}.
\end{remark}
\begin{proof}
The first and the second result follows from the fact
\bna
z=x+iy=\bma 1&x\\&1\ema\bma y^{1/2}\\&y^{-1/2}\ema.i
\ena
and that $\Delta$ commutes with the action of $g\in SL_2(\R)$,
\bna
\Delta \left(f(g.z)\right)=\left(\Delta f\right)(g.z).
\ena

Note that
\bna
&I_s(z)=(\im z)^s, \quad I_s(g.z)=\left(\frac{y}{|cz+d|^2}\right)^s,&\\
&I_s\left(\bma &-1 \\ 1\ema\bma 1&u\\&1\ema.z\right)=\left(\frac{y}{(x+u)^2+y^2}\right)^s&
\ena
Thus
\bea
\tilde W_s(z)=
\int_{-\infty}^\infty \left(\frac{y}{(x+u)^2+y^2}\right)^se(-u)du
=e(x)\int_{-\infty}^\infty \left(\frac{y}{u^2+y^2}\right)^se(-u)du\label{tilde-whittaker}.
\eea
By the above and the definition on the lower half plane, obviously one has
\bna
\tilde W_s(az)=e(a x)W_s(i|a|y).
\ena
Moreover, by the definition,
\bna
\overline{\tilde W_s(x+iy)}
=e(-x)\int_{-\infty}^{\infty}\left(\frac{y}{u^2+y^2}\right)^{\overline s} e(u)du
=\tilde W_{\overline s}(-x+iy).
\ena
The final step follows from the
following proposition immediately.
\end{proof}
\begin{lemma}\label{lemma-integral-Whitaker-Bessel}Let $y\in(0,+\infty)$.
For $\re(s)>0$ we have
\bna
\int_{-\infty}^\infty \left(\frac{y}{u^2+y^2}\right)^se(-au)du
=\frac{\pi^s}{\Gamma(s)}\left\{
\begin{aligned}
&\pi^{-s+\frac{1}{2}}\Gamma(s-\frac{1}{2})y^{1-s},\quad &&a=0\\
&2|a|^{s-\frac{1}{2}}\sqrt{|a|}K_{s-\frac{1}{2}}(2\pi|a|y),\quad &&a\neq 0
\end{aligned}
\right.
\ena

\end{lemma}
\begin{proof}
Recall that
\bna
\Gamma(s)=\int_0^\infty e^{-t}t^s\frac{dt}{t}
\ena
i.e. $\Gamma(s)$ is the Mellin transform of $e^{-t}$ with transform kernel $t^s$.
The integral involves $\left(\frac{y}{u^2+y^2}\right)^s$ which can be combined with $t^s$
to get new kernel and then change variable.

By multiplying  $\Gamma(s)$ and exchanging the integration,
\bna
&&\Gamma(s)
\int_{-\infty}^\infty \left(\frac{y}{u^2+y^2}\right)^se(-au)du
=\int_{0}^\infty e^{-t}t^{s}\frac{dt}{t}\int_{-\infty}^\infty \left(\frac{y}{u^2+y^2}\right)^se(-au)du\\
&=&\int_{-\infty}^\infty  e(-au)
\left\{\int_{0}^\infty e^{-t}\left(\frac{ty}{u^2+y^2}\right)^s\frac{dt}{t}\right\}du
=\int_{-\infty}^\infty  e(-au)
\left\{\int_{0}^\infty e^{-t\frac{u^2+y^2}{y}}t^s\frac{dt}{t}\right\}du\\
&=&
\int_{0}^\infty e^{-ty}t^s
\left\{\int_{-\infty}^\infty e^{-\frac{t}{y}u^2-2\pi i au}du\right\}\frac{dt}{t}
\ena
Now we can calculate the inner integral and apply the fact that Bessel function is expressed
as Mellin transform, namely
\bna
\int_0^\infty e^{-\frac{y}{2}(t+\frac{1}{t})}t^s\frac{dt}{t}=2 K_s(y).
\ena

\end{proof}



\subsection{Spectral decomposition of $L^2(\mk h)$}
We have obtained  eigenfunctions of $\Delta$. Now we give the
spectral decomposition of $L^2(\mk h)$ as follows.
\begin{thm}
Denote by
\bna
e_{a,s}(z)=
\left\{
\begin{aligned}
&y^s,\qquad &&a=0\\
&\sqrt{y}K_{s-\frac{1}{2}}(2\pi |a|y) e(ax),\quad &&a\neq 0
\end{aligned}
\right.
\ena
For  $f\in C_c^\infty(\mk h)$, denote by
\bna
\widehat f(a,s)=\int_{\mk h}f(z)\overline{e_{a,s}(z)}\frac{dxdy}{y^2}.
\ena
One has
\bna
f(z)=\int_{a\in \R}\int_{\re(s)=\frac{1}{2}}\widehat f(a,s) e_{a,s}(z)\frac{t\sinh(\pi t)}{\pi^2}dtda.
\ena
\end{thm}
\begin{remark}
Note that $W_s(az)=2\sqrt{\pi|a|} e_{s,a}(z)$. The above formula can be expressed as
\bna
f(z)&=&
\int_{a\in \R}\int_{\re(s)=\frac{1}{2}}
\langle f,e_{a,s}\rangle e_{a,s}(z) \frac{t\sinh(\pi t)}{\pi^2}dtda\\
&=&\frac{1}{2\pi i}
\int_{a\in \R}\int_{\re(s)=\frac{1}{2}}
\langle f,W_{s}(a*)\rangle W_s(az) \frac{t\sinh(\pi t)}{2\pi^2|a|}dads
\ena
which is the same as the formula in Iwaniec's book.

\end{remark}

\begin{proof}
To prove the identity, it is sufficient to prove it for $f(z)=h(x)g(y)$
with special values $x=0$ and $y=1$. Note
\bna
\hat f(a,\frac{1}{2}+it)
&=&\int_{-\infty}^\infty h(u)e(-au)du
\left(
\int_0^\infty g(v)\sqrt{v}K_{it}(2\pi |a|v)\frac{dv}{v^2}
\right).
\ena
Here note that
$\overline{K_{s-\frac{1}{2}}}(2\pi |a|y)= K_{\overline s-\frac{1}{2}}(2\pi |a|y)$
and $K_{-s}(y)=K_{s}(y)$, $s=\frac{1}{2}+it$.
%Replacing $v$ by $2\pi |a|v$, one has
%\bna
%\hat f(a,\frac{1}{2}+it)
%&=&\int_{-\infty}^\infty h(u)e(-au)
%\left(
%\int_0^\infty \frac{g(\frac{v}{2\pi|a|})}{\sqrt{\frac{v}{2\pi|a|}}}K_{it}(v)\frac{dv}{v}
%\right)du.
%\ena

We want to show that
\bna
h(x)g(y)&=&
\int_{a\in\R}\int_{-\infty}^\infty
\left(\int_{-\infty}^\infty h(u)e(-au)du
\left(
\int_0^\infty g(v)\sqrt{v}K_{it}(2\pi |a|v)\frac{dv}{v^2}
\right)\right)\\
&&\quad\sqrt{y}K_{it}(2\pi |a|y)e(ax)
 \frac{t\sinh(\pi t)}{\pi^2}dtda\\
 &=&
\int_{a\in\R}
\int_{-\infty}^\infty h(u)e(-au)du
\left(\int_{-\infty}^\infty
\int_0^\infty g(v)\sqrt{v}K_{it}(2\pi |a|v)\frac{dv}{v^2}
\sqrt{y}K_{it}(2\pi |a|y) \frac{t\sinh(\pi t)}{\pi^2}dt
\right)\\
&&\quad e(ax)da.
 \ena
 It is sufficient to consider
\bna
I(a,y)&=&
\int_{-\infty}^\infty
\int_0^\infty g(v)\sqrt{v}K_{it}(2\pi |a|v)\frac{dv}{v^2}
\sqrt{y}K_{it}(2\pi |a|y) \frac{t\sinh(\pi t)}{\pi^2}dt\\
&=&\int_0^\infty
\left(\int_0^\infty g(v)\frac{K_{it}(2\pi|a|v)}{\sqrt{v}}
\frac{dv}{v}\right)
\sqrt{y}K_{it}(2\pi |a|y) \frac{2t\sinh(\pi t)}{\pi^2}dt\\
&=&
2\pi|a|y\int_0^\infty
\left(\int_0^\infty
\frac{g(\frac{v}{2\pi|a|})}
{v}\frac{K_{it}(v)}{\sqrt{v}}
dv\right)
\frac{K_{it}(2\pi |a|y)}{\sqrt{2\pi|a|y}} \frac{2t\sinh(\pi t)}{\pi^2}dt.
\ena
Applying Kontorovitch-Lebedev transform,
\bna
I(a,y)=2\pi |a|y \times \left.\frac{g\left(\frac{v}{2\pi |a|}\right)}{v}\right|_{v=2\pi|a|y}=g(y)
\ena
We finish the proof.
\end{proof}
\subsection{Kontorovitch-Lebedev trasform}
\begin{prop}[Kontorovitch-Lebedev]
Let $h(y)$ with $y>0$ be a function, one has
\bna
g(y)&=&\int_0^\infty
\left(\int_{0}^\infty g(v)\frac{K_{it}(v)}{\sqrt{v}}dv\right)
\frac{K_{it}(y)}{\sqrt{y}} \frac{2t\sinh(\pi t)}{\pi^2}dt\\
f(t)&=&\frac{2t\sinh(\pi t)}{\pi^2}
\int_0^\infty\frac{K_{it}(y)}{y}
\left\{
\int_0^\infty f(u)K_{iu}(y)du
\right\}dy.
\ena

\end{prop}

\begin{proof}
Recall the integral representation of $K$-Bessel function, for $s=it$,
by viewing $y$ as a parameter,
\bna
K_{it}(y)&=&
\frac{1}{2}\int_0^\infty e^{-\frac{y}{2}(t_0+\frac{1}{t_0})}t_0^{it}\frac{dt_0}{t_0},\quad t_0=e^x\\
&=&
\frac{1}{2}\int_{-\infty}^\infty e^{-y\cosh x}e^{2\pi i\frac{x}{2\pi}t} dx
=
\pi\int_{-\infty}^\infty e^{-y\cosh (2\pi x)}e^{2\pi i xt} dx
\ena
and we know that $t\mapsto K_{it}(y)$ is the Fourier inverse transform  of
\bea
h_y(x):=\pi e^{-y\cosh 2\pi x}\label{Fourier-tem}
\eea


Recall the multiplication formula $$\int f\hat g=\int \hat f g.$$
For $f$ defined on $\R_+$, we extend it to $\R$ via $f(-t)=f(t)$, then
\bna
A&=&\frac{2}{\pi^2}t\sinh(\pi t)\int_{y=0}^\infty K_{it}(y)\frac{1}{y}
\left(\int_{u=0}^\infty f(u) K_{iu}(y)du\right)dy\\
&=&\frac{1}{\pi^2}t\sinh(\pi t)\int_{y=0}^\infty K_{it}(y)\frac{1}{y}
\left(\int_{u=-\infty}^\infty f(u) K_{iu}(y)du\right)dy,\quad\\
&=&\frac{1}{\pi^2}t\sinh(\pi t)\int_{y=0}^\infty K_{it}(y)\frac{1}{y}
\left(\int_{u=-\infty}^\infty \widehat f(u) \pi e^{-y\cosh 2\pi u}du\right)dy\\
&=&\frac{1}{2\pi^2}t\sinh(\pi t)\int_{y=0}^\infty K_{it}(y)\frac{1}{y}
\left(\int_{u=-\infty}^\infty \widehat f(\frac{u}{2\pi}) e^{-y\cosh  u}du\right)dy\\
&=&\frac{1}{2\pi^2}t\sinh(\pi t)
\int_{u=-\infty}^\infty \widehat f(\frac{u}{2\pi})
\left(\int_{y=0}^\infty K_{it}(y)\frac{1}{y}e^{-y\cosh  u} dy \right)du.
\ena
Next we applying the formula
\bea
\int_{y=0}^\infty e^{-y\cosh u}K_{it}(y)\frac{1}{y}dy=\pi\frac{\cos(tu)}{t\sinh(\pi t)}
\label{some-formula},
\eea
whose proof is in Page 177 in {\it Harmonic analysis on symmetric space},
one has
\bna
A=\int_{u=-\infty}^\infty\widehat f(\frac{u}{2\pi})\frac{\cos(tu)}{2\pi}du
=f(t)
\ena
since $f$ is even. We finish the proof.
\end{proof}
\begin{remark}

By \eqref{Fourier-tem},
\bna
&&\pi e^{-y\cosh 2\pi x}=\int_{-\infty}^\infty K_{it}(y)e^{-2\pi i t x}dt\\
&\Leftrightarrow&\pi e^{-y\cosh x}=\int_{-\infty}^\infty K_{it}(y)e^{-i t x}dt
=\pi e^{-y\cosh (-x)}=\int_{-\infty}^\infty K_{it}(y)e^{i t x}dt\\
&\Leftrightarrow&
\pi e^{-y\cosh x}=\int_{-\infty}^\infty K_{it}(y)\frac{ e^{-itx}+ e^{itx}}{2}dt\\
&\Leftrightarrow&\frac{\pi}{2}\exp(-y\cosh x)=\int_0^\infty K_{it}(y)\cos(tx)dt,\quad \re(y)>0.
\ena
Via the integral representation, we can also prove that
\bna
\int_0^\infty y^{r-1}K_{s}(y)dy=2^{r-2}\Gamma\left(\frac{r+s}{2}\right)\Gamma\left(\frac{r-s}{2}\right).
\ena

\end{remark}
\begin{proof}
We give another proof as follows. It is sufficient to prove that the kernel function
\bna
W_R(x,y)=\frac{1}{\pi^2}\int_{-R}^Rt \sinh(\pi t)
\frac{K_{it}(x)K_{it}(y)}{\sqrt{xy}}dt
\ena
approaches $\delta(x-y)$, as $R\rightarrow\infty$.
Since the problem is invariant under $SL_2(\R)$, it is sufficient to consider the problem as $x,y\sim 0$.

Note
\bna
K_{it}(y)\sim 2^{it-1}\Gamma(it)y^{-it}+2^{-it-1}\Gamma(-it)y^{it},\quad y\rightarrow 0^+
\ena
which follows from the relation with $I$-Bessel and the power series for $I$-Bessel.
Moreover,
\bna
\Gamma(it)\Gamma(-it)=\pi (t\sinh(\pi t))^{-1}
\ena
implies that
\bna
W_R(x,y)\sim \frac{1}{2\pi}\int_{-R}^Ry^{-\frac{1}{2}-it}x^{-\frac{1}{2}+it}dt,
\quad x,y\rightarrow 0^+.
\ena
Thus spectral measure is chosen to cancel the Gamma-factors.
\end{proof}
\subsection{Fourier expansion of functions
in $L^2(\Gamma_\infty\backslash \mk h)$}
Let
\bna
\Gamma_\infty=\left\{\bma 1&n\\&1\ema, n\in \Z\right\}
\ena
be a discrete subgroup. We know that elements in $\Gamma_\infty$
are parabolic, and
\bna
\bma 1&n\\&1\ema.z=z+n
\ena
has only one fixed point $z=\infty$.

Let $L^2(\Gamma_\infty\backslash \mk h)$
be the functions  $f:\mk h\rightarrow \C$
with
\bna
&&f\left(\bma 1&n\\&1\ema.z\right)=f(z+n)=f(z),\quad\forall n\in \Z\\
&&\int_{\Gamma_\infty\backslash \mk h}f(z)\overline{f(z)}\frac{dxdy}{y^2}<\infty.
\ena
Here a fundamental mesh is
\bna
\Gamma_\infty\backslash\mk h=\{z=x+iy, 0\leq x<1, y>0\}.
\ena

\begin{prop}
Let $f(z)$ be eigenfunctions of $\Delta$ with eigenvalue
$\lambda=s(1-s)$ which satisfies
\bit
\item $f(z+m)=f(z)$, $\forall m\in \Z$
\item $f(z)$ is of moderate growth, i.e.
\bna
f(z)=o\left(e^{2\pi y}\right),\quad y\rightarrow\infty
\ena
\eit
Then $f(z)$ has expansion
\bna
f(z)=a_{f,0}(y)+\sum_{n\neq 0}a_{f}(n) W_s(nz)
\ena
where $a_{f,0}(y)$ is a linear combination of
$y^s$ and $y^{1-s}$ if $\lambda\neq \frac{1}{4}$,
and $y^{1/2}$ and $y^{1/2}\log y$ if $\lambda=\frac{1}{4}$; and $a_{f}(n)$
are some coefficients (depending on $f$, called the Fourier coefficients of $f$).
\end{prop}

\section{The Hyperbolic Geometry - the Geodesic Polar coordinates}
Recall that we have defined the distance function. Consider the radius with center $i$, i.e.
\bna
d(z,i)=r
\ena
It has hyperbolic area $4\pi(\sinh(r/2))^2$ and circumference $2\pi\sinh r$.
On the other hand, its Euclidean center is $i\cosh r$ and radius $\sinh r$.

\subsection{Cartan decomposition and polar coordinates}
Recall Cartan decomposition,
$G=KAK$. For $g\in PSL_2(\R)$,  $g=\kappa(\varphi)a(e^r)\kappa(\theta)$ with
\bna
&\kappa(\varpi)=\bma\cos\varphi&\sin\varphi\\ -\sin\varphi&\cos\varphi\ema,
\quad 0\leq\varphi<\pi&,\\
&a(e^{-r})=\bma e^{-\frac{r}{2}}\\&e^{r/2}\ema,\quad r\geq 0&\\
&\kappa(\theta)\in SO(2)&.
\ena
It gives the geodesic polar coordinates
\bna
z=x+iy=\kappa(\varphi) a(e^{-r}).i=\kappa(\varphi) e^{-r}.i
\ena
Here $a(e^{-r})$ selects the point on y-axis  with distance $r$ with $i$ on the geodesic,
and $\kappa(\varphi)$ gives rotation of angle $2\varphi$.
With respect to the geodesic polar coordinates,
\bna
&ds^2=dr^2+(\sinh r)^2du^2,\quad d\mu(z)=\sinh r dr d\varphi &\\
&\Delta=-\frac{1}{\sinh r}\frac{\partial}{\partial r}\left(\sinh r\frac{\partial}{\partial r}\right)
-\frac{1}{\sinh^2 r}\frac{\partial^2}{\partial \varphi^2}
\ena
\subsection{Spherical functions and spectral decomposition}

We still want to obtain eigenfunctions of $\Delta$ with eigenvalue $\lambda=s(1-s)$. By separable parameters, eigenfunctions should be of the form
\bna
f(\kappa_\theta z)=\chi(\kappa_\theta)f(z),\quad\forall\kappa_\theta\in K
\ena
where
\bna
\chi:\kappa_\theta\mapsto e^{2i m\theta},\quad 0\leq\theta<\pi.
\ena

By simlar argument, we start from the function
\bna
I_s(z)&=(\im z)^s=(\im \kappa_\varphi e^{-r}.i)^d
\ena
and thus to form such $f(z)$ as
\bna
f(z)&=&\frac{1}{\pi}\int_{0}^\pi \im (\kappa(-\theta)\kappa(\varphi) e^{-r}.i)^s \chi(\kappa_\theta)d\theta\\
&=&\frac{1}{\pi}\int_0^\pi(\cosh r+\sinh r\cos2\theta)^{-s}e^{2im(\theta+\varphi)}d\theta\\
&=&\frac{\Gamma(1-s)}{\Gamma(1-s+m)}P_{-s}^m(\cosh r)e^{2im\varphi}
\ena
where $P_{-s}^m$ is the Legendre function defined by
\bna
P_\nu(z)&=&F(-\nu,\nu+1,1,\frac{1-z}{2})\\
P_\nu^m(z)&=&(z-1)^{m/2}\frac{d^m}{dz^m}P_\nu(z)\\
&=&\frac{\Gamma(\nu+m+1)}{\pi\Gamma(\nu+1)}\int_0^\pi (z+\sqrt{z^2-1}\cos\alpha)^\nu\cos(m\alpha)d\alpha\\
&=&\frac{\Gamma(\nu+m+1)}{2\pi\Gamma(\nu+1)}\int_0^{2\pi} (z+\sqrt{z^2-1}\cos\alpha)^\nu e^{im\alpha}d\alpha,\quad \re(z)>0
\ena

\begin{prop}
The spherical function is defined by
\bna
U_s^m(z):=P_{-s}^m(\cosh r)e^{2im\varphi},
\ena
Then for any $f\in C_c^\infty(\mk h)$, we have
\bna
\widehat f(m,s):=\langle f, U_s^m\rangle=\int_{\mk h}f(z)U_s^m(z)d\mu(z)
\ena
and
\bna
f(z)=\sum_{m\in \Z}\frac{(-1)^m}{2\pi i}\int_{\re(s)=1/2}
\widehat f(m,s) U_s^{m}(z) t\tanh (\pi t)ds.
\ena
\end{prop}
\begin{remark}
Spherical functions of order $m=0$ depends only on the hyperbolic distance.
\end{remark}
\begin{proof}
It follows from Fourier expansion of periodic function with the following inversion formula
\bna
g(u)=\int_0^\infty P_{-1/2+it}(u)
\left(\int_{1}^\infty P_{-1/2+it}(v)g(v)dv\right) t\tanh(\pi t)dt
\ena
with $P_s(u):=P_s^0(u)$.


It is sufficient to show the kerne
\bna
V_R(x,y)=\int_{0}^Rt\tanh(\pi t)P_{-\frac{1}{2}+it}(x)P_{-\frac{1}{2}+it}(y)dt
\ena
approaches $\delta(x-y)$ as $R\rightarrow\infty$.

Note
\bna
P_{-1/2+it}(x)\sim\frac{\Gamma(it)}{\sqrt{\pi}\Gamma(\frac{1}{2}+it)}(2x)^{-\frac{1}{2}+it}
+
\frac{\Gamma(-it)}{\sqrt{\pi}\Gamma(\frac{1}{2}-it)}(2x)^{-\frac{1}{2}+it},\quad x\rightarrow\infty
\ena
for fixed real $t$,
 and
 \bna
 \frac{\Gamma(it)\Gamma(-it)}{\pi\Gamma(\frac{1}{2}+it)
 \Gamma(\frac{1}{2}-it)
 }=\frac{1}{\pi t \tanh(\pi t)}r,
\ena
thus
\bna
V_R(x,y)\sim\frac{1}{\pi}\int_0^R x^{-\frac{1}{2}+it}y^{-\frac{1}{2}-it}dt,\quad x,y\sim \infty
\ena
on right side of which  is a Dirac delta family by Mellin inversion formula.
\end{proof}

\section{Helgason trasform on $\mk h$}
Set $B=K/M$ with $M=\{\-I,I\}$, where $B$ is called the `boundary' of $\mk h$.


Let $f\in C_c^\infty(\mk h)$.
For $s\in\C, k\in SO(2)$, we define
\bna
\mathcal H f(s,k)=\int_{\mk h} f(z)\overline{\im (kz)^s} \frac{dxdy}{y^2}.
\ena
\begin{prop}
One has
\bna
f(z)=\frac{1}{4\pi}\int_{t\in\ R} \frac{1}{2\pi}\int_{\theta=0}^{2\pi}
\mathcal Hf(\frac{1}{2}+it,k_\theta) \im(k_\theta z)^{\frac{1}{2}+it}
t\tanh(\pi t) d\theta dt,
\ena
where $k_\theta\in SO(2)$.

The map $f\mapsto\mathcal Hf$ takes $C_c^\infty(\mk h)$
one-to-one, onto the space of $C^\infty$ functions  $G(s,k)$
on $\C\times SO(2)$ which are holomorphic in $s$.
It extends to an isometry mapping $L^2(\mk h,\frac{dxdy}{y^2})$
onto $L^2(\R\times K,\frac{1}{8\pi^2}t\tanh(\pi t)dtd\theta)$
where $K=SO(2)$ is identified with $(0,2\pi)$
\end{prop}
\begin{prop}
The Helgason transform of $K$-invariant functions is a composition of Harish-Chandra
and Mellin transforms.
For $f_0\in C_c^\infty(GL_2(\R)^+,Z_\infty K_\infty)$,
the action $\pi_{\epsilon_\pi,it_{\pi}}(f_0)$ on  $\phi_0$ is a scalar given by
\bna
\mathcal S(f_0)(it_\pi):=\int_{0}^\infty
\left[y^{-1/2}
\int_{-\infty}^\infty f_0\left(\bma 1&x\\&1\ema\bma y^{1/2}&\\&y^{-1/2}\ema\right)
dx\right] y^{it_{\pi}}\frac{dy}{y},
\ena
where $\mathcal S(f_0)$ is called the spherical transform of $f_0$. Moreover,
the spherical transform $\mathcal S$ defines a map
\bna
\mathcal S: C_c^\infty(GL_2(\R)^+,Z_\infty K_\infty)\rightarrow PW^\infty(\C)^{\mathrm{even}},\quad f_0\mapsto \mathcal S(f_0)
\ena
which is an isomorphism to the Paley-Wiener space of even functions.
\end{prop}

\section{Automorphic forms for $SL_2(\Z)$}
\begin{prop}
$SL_2(\Z)$ is a disconnected subgroup of $SL_2(\R)$. It has two generators, namely
\bna
T=\bma 1&1\\&1\ema,\quad S=\bma &-1\\ 1\ema
\ena
\end{prop}
\begin{proof}
Note that $T^m=\bma 1&m\\&1\ema$ and $S^2=\bma -1\\&-1\ema$.
Thus if $\bma a&b\\0&d\ema\in SL_2(\Z)$, then $a=d\in\{\pm 1\}$ and $b\in \Z$
can be expressed by product of $T$ and $S$. This shows that
\bna
\Gamma_\infty=\left\{\bma\pm 1&m\\&\pm 1\ema,m\in\Z\right\}
=\{\gamma\in SL_2(\Z),\gamma.\infty=\infty\}
\ena
can be expressed by $T$ and $S^2$.


Assume $\gamma=\bma a&b\\c&d\ema\in SL_2(\Z)$ with $c\neq 0$.
Note that $\det\gamma=ad-bc=1$ which implies that $(a,c)=1$, otherwise
\bna
ax+cy=1
\ena
has no integral solution $(x,y)\in \Z^2$.
Multiplying $\bma 1&m\\&1\ema$ on left one has
\bna
\bma 1&m\\&1\ema\bma a&b\\c&d\ema=\bma a+mc&b+md\\c&d\ema=\bma a_1&b_1\\c&d\ema
\ena
by choosing suitable $m$ we can assume that $0\leq a_1=a+mc<|c|$.
Next, we multiplying $S$ to obtain
\bna
S\bma 1&m\\&1\ema\bma a&b\\c&d\ema=\bma &-1\\1\ema\bma a_1&b_1\\c&d\ema
=\bma -c&-d\\ a_1&b_1\ema .
\ena
Note that $0\leq a_1<|c|.$
repeat the above steps we will finally obtain a matrix with
\bna
\bma a_n&b_n\\ c_n&d_n\ema,\quad c_n=0
\ena
which is an element in $\Gamma_\infty$.
\end{proof}



\subsection{Fundamental domain}
we have
\bna
\Gamma\backslash \mk h=\{z=x+iy,\quad -\frac{1}{2}\leq x\leq\frac{1}{2},  \sqrt{1-x^2}<y<\infty\}
\ena
This is non-compact and of finite volume, and has only one cusp $\infty$.

\subsection{Fourier expansion of $L^2(\Gamma\backslash\mk h)$}
We have shown that the Fourier expansion of functions $L^2(\Gamma_\infty\backslash \mk h)$
should be
\bna
f(z)=a_{f}(0,y)+\sum_{n\neq 0}a_f(n) \sqrt{y}K_{s-\frac{1}{2}}(2\pi|n|y)e(nx)
\ena
where
\bna
a_f(0,y)=
\left\{
\begin{aligned}
a_{f,1}(0)y^{s}+a_{f,2}(0)y^{1-s},\quad s=\neq 1/2\\
a_{f,1}(0)y^{1/2}+ a_{f,2}(0)y^{1/2}\log y,\quad s=1/2
\end{aligned}
\right.
\ena
\subsection{Cusp forms}
Note that  we add an compact region to the fundamental domain to consider
\bna
\int_{\Gamma\backslash \mk h}|f(z)|^2dz&\leq& \int_{-1/2}^{1/2}\int_{\frac{\sqrt{3}}{2}}^{\infty}|f(z)|^2dz\\
&=&\int_{\sqrt{3}/2}^\infty |a_f(0,y)|^2\frac{dy}{y^2}
+\sum_{n\neq 0}|a_f(n)|^2\int_{\sqrt{3}/{2}}^\infty |y|\left|K_{s-1/2}(2\pi |n|y)\right|^{2}\frac{dy}{y^2}
\ena
The asymptotic formula
\bna
K_{s-1/2}(z)\sim
\sqrt{\frac{\pi}{2z}} e^{-z}
\ena
implies the sum over $n\neq 0$ is absolutely convergent.


For the constant term,
\bna
\int_{\sqrt{3}/2}^{\infty} |y^{2\re(s)}|\frac{dy}{y^2}
=\int_{\sqrt{3}/2}^{\infty} |y^{2(\re(s)-1)}|dy
=\infty
\ena
unless $\re(s)<\frac{1}{2}$.
\begin{prop}
A function $f$ which admits no constant term, namely
\bna
\int_{0}^1f(z)dx\neq 0
\ena
for all $y$, is called cusp form.
\end{prop}
Via the Fourier expansion, and the property of K-Bessel function,
if $f$ is a cusp form, then it vanishes at $y\rightarrow\infty$.

If we realized $z=x+iy$ as matrix $\bma 1&x\\ &1\ema\bma y^{1/2}\\&y^{-1/2}\ema\kappa_\theta$
and lifting maass forms as functions on
\bna
f:SL_2(\R)\rightarrow\C
\ena
The cuspidal condition is equivalently to
\bna
\int_{\Z\backslash\R}f\left(\bma 1&x\\&1\ema g\right)dx=0
\ena
i.e. $f$ admits no-`trivial` property at $N(\Z\backslash \R)$.
\section{$L$-function and functional equation of even and odd maass cusp forms}
\subsection{Even and odd Maass cusp forms}
We define
\bna
\iota:\mk h\rightarrow \mk  h,\quad z=x+iy\mapsto -\overline{z}=-x+iy
\ena
Extend it to be an operator
\bna
f(z)\mapsto f(\iota z)
\ena
One has $\iota^2=id$. Thus if $f$ is eigen function of $\iota$, then the eigenvalues
should be $\pm 1$.
\bit
\item We call $f$ is even Maass cusp form, if $\iota f=f$.
In this case, by the Fourier expansion of $f(z)$, we have
\bna
a_f(-n)=a_f(n),\quad n\in \Z
\ena
\item We call $f$ is odd Maass cusp form, if $\iota f=-f$.
In this case $a_f(-n)=-a_f(n)$.
\eit
\begin{remark}
The operator $\iota$ commutes with $\Delta$ and Hecke operators defined later.
\end{remark}
\subsection{L-function asscoaited to even Maass cusp $f$}
Let $f$ be an even Maass cusp form with spectral parameter $\frac{1}{2}+it$.
We consider
\bna
I(s,f):=\int_0^\infty f(iy)|y|^{s-\frac{1}{2}}\frac{dy}{y}.
 \ena

 Note that
\bna
f(iy)=f(S.iy)=f\left(-\frac{1}{iy}\right)=f(i\frac{1}{y}).
\ena
Thus
\bna
I(s,f):&=&\int_{1}^\infty f(iy)|y|^{s-\frac{1}{2}}\frac{dy}{y}
+\int_{0}^1f(i\frac{1}{y})|y|^{s-\frac{1}{2}}\frac{dy}{y}\\
&=&\int_{1}^\infty f(iy)|y|^{s-\frac{1}{2}}\frac{dy}{y}
+\int_{1}^\infty f(iy)|y|^{1-s-\frac{1}{2}}\frac{dy}{y}
\ena
and one know that $I(s,f)$ is an entire function for all
$s\in\C$ and satisfies the functional equation
\bna
I(s,f)=I(1-s,f).
\ena

On the other hand,  for $\re(s)$ large,
by the Fourier expansion and $f$ is even Maass cusp form.
\bna
f(iy)=\sum_{n\neq 0}a_f(n)\sqrt{y}K_{it}(2\pi |n|y)
=2\sum_{n\geq  1}a_f(n)\sqrt{y}K_{it}(2\pi ny),
\ena
we have
\bna
I(s,f)&=&2\sum_{n\geq 1}a_f(n)
\int_0^\infty K_{it}(2\pi ny)  y^{s}\frac{dy}{y}\\
&=&2\sum_{n\geq 1}a_f(n)\frac{1}{(2\pi n)^{s}}
\int_0^\infty K_{it}(y)  y^{s}\frac{dy}{y}\\
&=&2(2\pi)^{-s}\sum_{n\geq 1}\frac{a_f(n)}{n^s}\int_0^\infty K_{it}(y)y^s\frac{dy}{y}
\ena
\begin{lemma}
One has
\bna
\int_0^\infty K_{\nu}(y)y^s\frac{dy}{y}=2^{s-2}\Gamma\left(\frac{s+\nu}{2}\right)\Gamma\left(\frac{s-\nu}{2}\right)
\ena
which is absolutely convergent if $\re(s)>\re(\nu)$.
\end{lemma}
\begin{proof}
Recall that
\bna
K_{\nu}(y)=\frac{1}{2}\int_0^\infty e^{-y(t+\frac{1}{t})/2}t^s\frac{dt}{t}
\ena
for all values of $\nu$. Thus
\bna
L.H.S.&=&\frac{1}{2}\int_0^\infty\int_0^\infty e^{-\frac{yt}{2}-\frac{y}{2t}}t^{\nu} y^s\frac{dy}{y}\frac{dt}{t}.
\ena
We hope to separable parameters and thus take $u=\frac{ty}{2}$ $v=\frac{y}{2t}$ so that
\bna
\frac{du}{u}\wedge \frac{dv}{v}=2\frac{dt}{t}\wedge \frac{dy}{y}
\ena
and thus
\bna
L.H.S.=2^{s-2}\int_0^\infty\int_0^\infty e^{-u-v}u^{(s+\nu)/2} v^{(s-\nu)/2}\frac{du}{u}\frac{dv}{v}
=R.H.S.
\ena
\end{proof}
\begin{thm}
Let $f$ be an even maass cusp form  for $SL_2(\Z)$
with eigenvalue $\lambda=\frac{1}{4}+t^2$ (spectral parameter $\frac{1}{2}+it$).
Then we have
\bna
f(z)=\sum_{n\neq 0}a_f(n)\sqrt{y}K_{it}(2\pi|n|y)e(nx)
\ena
with $a_f(-n)=a_f(n)$.
The $L$-function
\bna
L(s,f):=\sum_{n\geq 1}\frac{a_f(n)}{n^s}
\ena
is defined for $\re(s)$ large and has analytic continuation to all $s\in\C$.
Denote by
\bna
\Lambda(s,f)=\pi^{-s}\Gamma(\frac{s+it}{2})\Gamma(\frac{s-it}{2})L(s,f)
\ena
one has the functional equation
\bna
\Lambda(s,f)=\Lambda(1-s,f).
\ena
\end{thm}
\subsection{$L$-function associated to odd maass cusp form}

Assume that $f$ is odd maass cusp form.
It has Fourier expansion with $a_f(-n)=-a_f(n)$ and we can define
\bna
L(s,f)=\sum_{n\geq 1}\frac{a_f(n)}{n^s}
\ena
for $\re(s)$ large.
Since $f$ is odd, the integral $I(s,f)$ vanishes.

To establish the functional equation, on taking
\bna
g(z):=\frac{1}{4\pi i}\frac{\partial f}{
\partial x}(z),
\ena
one needs to consider
\bna
I(s,g):=\int_{0}^\infty g(iy)y^{s+\frac{1}{2}}\frac{dy}{y}
\ena
and  establishes
\bna
\Lambda(s,f)&=&\pi^{-s}\Gamma\left(\frac{s+it-1}{2}\right)
\Gamma\left(\frac{s-it-1}{2}\right)L(s,f)\\
&=&(-1)\Lambda(1-s,f).
\ena
For more information, we refer to page 107 in Bump's book.

\section{The theory of Hecke operators
-Does the $L$-function admits Euler product}
Let $\Gamma=SL_2(\Z)$. For any subgroups $G\subset \Gamma$, we have
\bna
f(\gamma.z)=f(z),\quad \gamma\in G\subset\Gamma.
\ena
To obtain much more information, we consider
the action of a  much bigger discontinuous subgroup.

Consider
\bna
M_2(\Z)=\left\{\bma a&b\\c&d\ema,a,b,c,d\in \Z\right\}
\ena
It is the biggest in some sense.
We decompose $M_2(\Z)$ as
\bna
M_2(\Z)=\sum_{n}G_n
\ena
where
\bna
G_n=\left\{g=\bma a&b\\c&d\ema, \det g =ad-bc=n\right\}.
\ena
One has
\bna
\Gamma G_n=G_n\Gamma
\ena
\begin{remark}
We are interested in those $G_n$ with $n\geq 1$.
\end{remark}
\subsection{Slash operator}
For $g\in GL_2(\R)$  and $f\in L^2(\Gamma\backslash \mk h)$,
we define the operator
\bna
f\mapsto f|_{g},\quad f|_{g}(z)=f(g.z)
\ena
Note that $f$ is automorphic for $\Gamma$ then
\bna
f|_{\gamma}(z)=f(\gamma z)=f(z).
\ena

\begin{remark}
The slash operator is defined by left translation, which commutes with the
action of $\Delta$ naturally.
Thus $f$ is automorphic for $\Gamma$ if and only if
\bna
f|_{\gamma}=f,\quad \forall \gamma\in\Gamma.
\ena
\end{remark}
\subsection{The right cosets $\Gamma\backslash G_n$}
Start from elements in $g\in G_n$, we hope to construct new operators $T_n$
which map automorphic forms to automorphic forms.

Note that for $\Gamma g$,
\bna
f|_{\Gamma g}(z)=f(\Gamma g.z)=f(g.z)=f|_{g}(z)
\ena
So we need only consider the right cosets $\Gamma\backslash G$.


\begin{lemma}For $G_n$, we have the right coset decomposition
\bna
G_n=\bigcup_{g\in \Delta_n}\Gamma g,
\ena
where $\Delta_n$ is the set of representative elements given by
\bna
\Delta_n=\{\bma a&b\\&d\ema,\quad ad=n,0\leq b<d\}
\ena
\end{lemma}
\begin{proof}
For any $\rho=\bma a&c\\*&*\ema\in G_n$, and
 $\gamma=\bma *&*\\ \tau&\delta\ema\in\Gamma$, i.e. $(\tau,\delta)=1$
\bna
\gamma\rho=\bma a&*\\  \tau a+\delta c &*\\\ema
\ena
Note that  $ax+ cy=0$ always have solutions $(x_0,y_0)\in \Z^2-\{0,0\}$.
By dividing the greatest common divisor
$(x_0,y_0)$, we can assume that they are coprime.
Thus on taking $\tau=x_0$ and $\delta=y_0$, we have $(\tau,\delta)=1$.
So there exists such $\gamma=\bma *&*\\ \tau&\delta\ema$
so that
\bna
\gamma\rho=\bma *&*\\ 0&*\ema
\ena


So we consider the representative elements in $\{\bma a&*\\&d \ema,\quad ad=n\}$
By multiplying $\pm I\in\Gamma$, we can assume that $a>0$ and $d>0$.

By multiplying $\bma 1&m\\&1\ema$, with $m\in \Z$
\bna
\bma 1&m\\&1\ema \bma a&b\\&d\ema=\bma a&b+md\\&d\ema
\ena
and we can assume that $0\leq b_1=b_md\leq d-1$. This shows the final result.

\end{proof}

\begin{lemma}
There exists an $1-1$ correspondence between
\bna
\Delta_n\times\Gamma=\Gamma\times\Delta_n
\ena
i.e. for any $\rho,\gamma\in\Gamma$,
there exists unique $\rho'$ and $\gamma'$ so that
\bna
\rho.\gamma=\gamma'.\rho
\ena

\end{lemma}
\begin{remark}
Although elements  $\rho\in \Delta_n$ is not in the normalizer of $\Gamma$,
\bna
g.\Gamma=\Gamma.g
\ena
But all the set should be. This suggests us to define
\bna
T_n f(z):=\sum_{g\in\Delta_n} f|_{g}(z)
\ena
Then for any $\gamma\in \Gamma$,
\bna
(T_nf)(\gamma.z)&=&\sum_{g\in\Delta_n} f_g(\gamma z)
=\sum_{g\in\Delta_n} f(g\gamma z)=\sum_{g\in\Delta_n} f(\gamma'g' z)\\
&=&\sum_{g'\in\Delta_n} f(g' z)=(T_nf)(z).
\ena
So $T_n$ maps automorphic forms to be automorphic forms.


\end{remark}
\subsection{Definition of the Hecke operators}
\begin{prop}
For $G_n=\{g\in M_2(\Z),\det g=n\}$, we have
\bna
\Gamma G_n=G_n\Gamma.
\ena
A set of representative elements of the right cosests $\Gamma\backslash G_n$ is
\bna
\Delta_n=\left\{\bma a&b\\&d\ema,\quad ad=n, 0\leq b<d\right\}
\ena
We define
\bna
(T_nf)(z)=\sum_{g\in \Delta_n} f|_{g}(z)
\ena
\bit
\item $T_n$ commutes to each other for all $n\geq 1$, and $T_n$ commutes with $\Delta$
and $\iota$.
\item $T_n$  maps automorphic forms to be automorphic forms.
\eit

\end{prop}
\begin{remark}
Thus we can assume an automorphic cuspidal forms are eigenfunction of $\Delta$
with eigenvalues $\frac{1}{4}+t^2$, even or odd, and is eigenfunctions for all Hecke operators.
\end{remark}
\subsection{The action of Hecke operators}.
Let
\bna
f(z)=\sum_{n\neq 0} a_f(n)\sqrt{y}K_{it}(2\pi|n|y) e(nx)
\ena
be a Maass cusp form with spectral parameter $\frac{1}{2}+it$.
Assume that $f(z)$ is an eigenfunction of $T_m$ with eigenvalue $\lambda_f(m)$,
\bna
T_m f(z)=\lambda_f(m)f(z)=\sum_{n\neq 0} a_f(n)\lambda_f(m)\sqrt{y}K_{it}(2\pi|n|y) e(nx).
\ena
On the other hand,
\bna
T_mf(z)=\sum_{ad=m}\sum_{b\mod d} f|_{\bma a&b\\0&d\ema}(z)
\ena
Note that
\bna
f|_{\bma a&b\\0&d\ema}(z)
=f\left(\frac{az+b}{d}\right)=
\sum_{n\neq 0} a_f(n)\lambda_f(m)\sqrt{\frac{a}{d}y}K_{it}(2\pi|n|\frac{a}{d}y) e(n\frac{ax+b}{d}).
\ena
Thus $T_mf(z)$ has Fourier expansion
\bna
T_m f(z)&=&
\sum_{ad=m}\sum_{b\bmod d}
\sum_{n\neq 0} a_f(n)\sqrt{\frac{a}{d}y}K_{it}(2\pi|n|\frac{a}{d}y) e(n\frac{ax+b}{d})\\
&=&
\sum_{ad=m}
\sum_{n\neq 0} a_f(n)\sqrt{\frac{a}{d}y}K_{it}(2\pi|n|\frac{a}{d}y)
e(n\frac{ax}{d})
\sum_{b\bmod d}e\left(n\frac{b}{d}\right)
\ena
Note that $\sum_{b\bmod d}e\left(n\frac{b}{d}\right)=d\delta_{d\mid n}$.
One has
\bna
T_m f(z)
&=&
\sum_{ad=m}
\sum_{d\mid n} a_f(n)\sqrt{\frac{a}{d}y}K_{it}(2\pi|n|\frac{a}{d}y)
e(n\frac{ax}{d}) d\\
&=&\sqrt{m}
\sum_{ad=m}
\sum_{d\mid n} a_f(n)\sqrt{y}K_{it}(2\pi|n|\frac{a}{d}y)
e(n\frac{ax}{d}) \\
&=&\sqrt{m}
\sum_{ad=m}
\sum_{\ell } a_f(d\ell)\sqrt{y}K_{it}(2\pi|d\ell |\frac{a}{d}y)
e(d\ell \frac{ax}{d}) \\
&=&\sqrt{m}
\sum_{a\mid m}
\sum_{\ell } a_f(\frac{m}{a}\ell)\sqrt{y}K_{it}(2\pi|\ell |ay)
e(\ell a x) \\
\ena
Let $n=\ell a$, then $n\geq 1$ and $a$ satisfies the condition $a\mid n$ and $a\mid m$.
Thus
\bna
T_mf(z)=\sqrt{m}\sum_{n} \sum_{a\mid (m,n)}a_f\left(\frac{mn}{a}\right)
\sqrt{y} K_{it}(2\pi |n| y)e(nx)
\ena
\begin{prop}Define
\bna
T_mf(z)=\frac{1}{\sqrt{m}} \sum_{ad=m}\sum_{b\bmod d} f|_{\bma a&b\\&d\ema}(z)
\ena
Assume $f(z)$ has Fourier expansion
\bna
f(z)=\sum_{n}a_f(n)\sqrt{y} K_{it}(2\pi |n|y) e(nx)
\ena
and $f$ is eigenfunction of $T_m$ with eigenvalue $T_mf(z)=\lambda_f(m) f(z)$.
Then
\bea
\lambda_f(m)a_f(n)=\sum_{d\mid (m,n)} a_f\left(\frac{mn}{d^2}\right)\label{tem1}
\eea
\end{prop}
\begin{remark} Assume that $f$ is eigenfunction of  all Hecke operators.
One has
\bea
\lambda_f(n)a_f(\pm 1)=\sum_{d\mid (1,n)}a_f\left(\frac{\pm 1n}{d}\right)=a_f(\pm n)\label{tem2}
\eea
So we can write $a_f(n)=a_f(1)\lambda_f(n)$ and thus
 the Fourier expansion of $f(z)$ is
\bna
f(z)=\sum_{n\neq 0}a_f(\mathrm{sign}(n)) \lambda_f(n)
\sqrt{y}K_{it}(2\pi|n|y)e(nx).
\ena
 \end{remark}
 \subsection{Properties of Hecke eigenvalues}
Assume that $f$ is eigenfunction of all Hecke operators $T_m$
with eigenvalues $\lambda_f(m)$.
 By \eqref{tem1} and \eqref{tem2}, we have the following Hecke relation.
\bea
\lambda_f(m_1)\lambda_f(m_2)=\sum_{d\mid (m_1,m_2)}\lambda\left(\frac{m_1m_2}{d^2}\right)
\eea
Thus $m\mapsto \lambda_f(m)$ is a multiplicative function with $\lambda_f(1)=1$.

By the Hecke relation, we have the recurrent formula
\bna
\lambda_f(p^n)\lambda_f(p)=\lambda_f(p^{n+1})+\lambda_f(p^{n-1})
\ena
and thus for $\lambda_f(1)=1$ and $\lambda_f(p)$,
\bna
\lambda_f(p^2)&=&-\lambda_f(1)+\lambda_f(p)^2=-1+\lambda_f(p)^2,\\
\lambda_f(p^3)&=&-\lambda_f(p)+\lambda_f(p)\lambda_f(p^2)
=-2\lambda_f(p)+\lambda_f(p)^3\\
\lambda_f(p^4)&=&-\lambda_f(p^2)+\lambda_f(p)\lambda_f(p^3)
=-\lambda_f(1)+\lambda_f(p)^2+\lambda_f(p)\left(-2\lambda_f(p)+\lambda_f(p)^3\right)\\
&=&-1-\lambda_f(p)^2+\lambda_f(p)^4\\
\lambda_f(p^5)&=&-\lambda_f(p^3)+\lambda_f(p)\lambda_f(p^4)
=-(-2\lambda_f(p)+\lambda_f(p)^3)+\lambda_f(p)(-1-\lambda_f(p)^2+\lambda_f(p)^4)\\
&=&-\lambda_f(p)
\ena

\section{Eisenstein series}
We recall the definition of the Maass forms as follows.
For
\bna
f:\mk h\rightarrow\C
\ena
\bit
\item [1.] $f$ is eigenfunction of $\Delta$ with eigenvalue $\lambda=s(1-s)=\frac{1}{4}+t^2$, $s=\frac{1}{2}+it$.
\item [2.] $f(\gamma.z)=f(z)$ for $\gamma\in SL_2(\Z)$.
\item [3.] $f\in L^2(\Gamma\backslash \mk h)$
\item [3'.] $f$ is of moderate growth, i.e. $f(x+iy)=o(e^{2\pi y})$ for some $N$.
\eit
We call $f$ is cusp form, if
\bna
\int_{0}^1f(x+iy)dx=0
\ena
for all but finite number of $y$, and we know that
\bna
&&\mbox{$f$ cusp form}\Leftrightarrow \mbox{$f$ vanishes at the cusp}\\
&\Leftrightarrow&\mbox{As functions on $SL_2(\R)=N(\R)A(\R)SO(2)$, $f$ vanishes
on $N(\Z\backslash \R)$}.
\ena
\medskip

We will introduce the Eisenstein series which is related to the spectrum of
\bna
L^2(\Gamma\backslash \mk h)- L^2_{\mathrm{cusp}}(\Gamma\backslash \mk h)
\ena


\subsection{Definition}
We start from the function
\bna
I_s(z):=(\im z)^s
\ena
which is eigenfunction of $\Delta$ with eigenvalue $s(1-s)$ and is $\Gamma_\infty$-invariant.
To construct function which is invariant under $\Gamma_\infty$, we define
\bna
E(z,s):=\sum_{\delta\in\Gamma_\infty\backslash \Gamma}I_s(\delta.z).
\ena

Formally, $E(z,s)$ is an automorphic forms in $z$ with eigenvalue $s(1-s)$ of $\Delta$.
However, the sum over $\Gamma_\infty\backslash \Gamma$ is an infinite sum,
and $E(z,s)$ may be divergent.
\begin{lemma}\label{lemma-right-cosets}
A set of the representative elements for $\Gamma_\infty\backslash\Gamma$
is
\bna
\bma 1&0\\0&1\ema, \quad \bma *&*\\ c&d\ema, c>0, d\in\Z, (c,d)=1.
\ena
\end{lemma}
\begin{proof}
Note that for given $\bma a&b\\c&d\ema$,
\bna
I_s\left(\bma a&b\\c&d\ema.z\right)=\frac{y^s}{|cz+d|^{2s}}
\ena
so we need only to determine $\bma *&*\\ c&d\ema$ in the representative element
for $\Gamma_\infty\backslash \Gamma$.

Let $\bma 1&m\\&1\ema\in\Gamma_\infty$ and $\bma a&b\\c&d\ema\in\Gamma$. We have
\bna
\bma 1&m\\&1\ema\bma a&b\\c&d\ema
=\bma a-cm&b-dm\\ c&d\ema=\bma a^*&b^*\\c&d\ema
\ena
For fixed $(c,d)=1$, as $m\in\Z$ varies, $(a^*,b^*)$ varies over solutions of
\bna
xd-yc=1.
\ena
So the representative elements in this equivalence class is uniquely
 characterized by $(c,d)$.

By multiplying $\pm I$, we can assume either $c>0$, or $c=0$ and $d=1$.
Thus a representative elements for the  coset $\Gamma_\infty\backslash \Gamma$ are
\bna
\bma *&*\\0&1\ema +
\bigcup_{c> 0}\bigcup_{d\in\Z\atop{(d,c)=1}}\bma *&*\\ c&d\ema
\ena
\end{proof}
By the above lemma, we have formually
\bna
E(z,s)&=&\left(\sum_{c=0}\sum_{d=1}+\sum_{c> 0}\sum_{d\in\Z\atop (d,c)=1}\right)
\frac{y^s}{|cz+d|^{2s}}\\
&=&\frac{1}{2}\frac{1}{2}\sum_{(c,d)\in \Z^2-\{(0,0)\}\atop (c,d)=1}\frac{y^s}{|cz+d|^{2s}}.
\ena
Multiplying $\zeta(2s)$, we can get rid of the coprime condition to obtain
\bna
\zeta(2s)E(z,s)
&=&\sum_{m\geq 1}\frac{1}{m^{2s}}
\left(\sum_{c=0}\sum_{d=1}+\sum_{c> 0}\sum_{d\in\Z\atop (d,c)=1}\right)
\frac{y^s}{|cz+d|^{2s}}\\
&=&
\left(\sum_{c=0}\sum_{d\geq 1}+
\sum_{c\geq 1}\sum_{d\in\Z}\right)\frac{y^s}{|cz+d|^{2s}}\\
&=&\frac{1}{2}\sum_{(c,d)\neq (0,0)}\frac{y^s}{|cz+d|^{2s}}
\ena
Thus
for any fixed $z\in \mk h$,
the infinite sum over $c$ and $d$ are absolutely convergent if $\re(s)>1$.
\begin{prop}Let $s\in\C$ be a spectral parameter.
 The Eisenstein series $E(z,s)$ is defined by
 \bna
 E(z,s)=\sum_{\delta\in\Gamma_\infty\backslash\Gamma}I_s(\delta.z)
 =\frac{1}{2}\sum_{(c,d)\in \Z^2-\{(0,0)\}\atop(c,d)=1}\frac{y^s}{|cz+d|^{2s}}
 \ena
 for $\re(s)>1$;
it is an automorphic form for $SL_2(\Z)$
in the sense
\bna
E(\gamma.z,s)=E(z,s),\quad\gamma\in SL_2(\Z)
\ena
and is an eigenfunction of $\Delta$
with eigenvalue
\bna
\lambda=s(1-s).
\ena
\end{prop}
\subsection{Fourier expansion of Eisenstein series}
For $E(z,s)$ defined as above, we consider the Fourier expansion of $E(z,s)$,
\bna
E(z,s)=\sum_n
a(y,n;s) e(nx)
\ena
where
\bna
a(y,n;s):&=&\int_{0}^1 E(x+iy,s)e(-nx)dx\\
&=&\int_{x\in \Z\backslash \R}\sum_{\delta \in \Gamma_\infty\backslash \Gamma}
I_s\left(\delta.z\right)e(-nx)dx
\ena
We need the following proposition which will give the relation
between the right coset $\Gamma_\infty\backslash\Gamma$ and the double cosets
$\Gamma_\infty\backslash\Gamma/\Gamma_\infty$.
\begin{lemma}\label{lemma-doublecoset}
The double coset of $\Gamma_\infty\backslash\Gamma/\Gamma_\infty$ is
\bna
\bigcup_{c\geq 0}\bigcup_{d\bmod c\atop{(d,c)=1}}\bma *&*\\c&d\ema
=\bma 1&0\\0&1\ema\bigcup \bigcup_{c>0}\bigcup_{d\bmod c\atop(d,c)=1}\bma *&*\\c&d\ema
\ena
and thus the representative elements in the right coset $\Gamma_\infty\backslash\Gamma$
can be expressed as
\bea
\bma 1\\&1\ema + \bigcup_{c>0}\bigcup_{d\in\Z\atop(c,d)=1}\bigcup_{m\in\Z}\bma *&*\\ c&d\ema
\bma 1&m\\&1\ema\label{express-rightcosets-via-double-cosets}
\eea
\end{lemma}
\begin{remark}
Formula \eqref{express-rightcosets-via-double-cosets} can be obtained directly from
Lemma \ref{lemma-right-cosets}.
\end{remark}
\begin{proof}
For $\bma 1&m\\&1\ema$ and $\bma 1&n \\&1\ema$ in $\Gamma_\infty$ (multiplying $\pm I$
if necessary), and $\bma a&b\\c&d\ema\in\Gamma$,
\bna
\bma 1&m\\&1\ema\bma a&b\\c&d\ema \bma 1&n\\&1\ema
=\bma a+mc&b+md\\c&d\ema\bma 1&n\\&1\ema=\bma a+mc& b+md+n(a+mc)\\ c& d+nc\ema
\ena
Multiplying $\pm I$ if necessary, we can assume that $c\geq 0$ and $d\geq 0$.

Note that $c\geq 0$, and $d$ is restricted in the reduced class modulo $c$.
elements in the first row are determined by the same reason in the right coset $\Gamma_\infty\backslash\Gamma$.
\end{proof}
By the above lemma, for $\re(s)>1$,
\bna
a(y,n;s)&=&\int_0^1 I_s(\bma1\\&1\ema.z)e(-nx)dx \\
&&+\sum_{c\geq1}\sum_{d\bmod c\atop(d,c)=1}
\sum_{m\in\Z}\int_0^1 I_s\left(\bma *&*\\ c&d\ema\bma 1&m\\&1\ema.z\right)e(-nx)dn\\
&=&y^s\int_0^1 e(-nz)dx+\sum_{c\geq 1}\sum_{d \bmod c\atop (d,c)=1}\sum_{m\in\Z}
\int_{0}^1I_s\left(\bma *&*\\ c&d\ema. (x+m+iy)\right) e(-n(x+m))dx\\
&=&y^s\delta_{n,0} + \sum_{c\geq 1}\sum_{d\bmod c\atop(c,d)=1}
\int_{-\infty}^\infty\frac{y^s}{\left((cx+d)^2+c^2y^2\right)^s}e(-nx)dx\\
&=&y^s\delta_{n,0} + \sum_{c\geq 1}\frac{1}{c^{2s}}\sum_{d\bmod c\atop((c,d)=1)}
y^s
\int_{-\infty}^\infty\frac{1}{\left((x+\frac{d}{c})^2+y^2\right)^s}e(-nx)dx\\
&=&y^s\delta_{n,0} + \sum_{c\geq 1}\frac{1}{c^{2s}}\sum_{d\bmod c\atop((c,d)=1)}e\left(\frac{d}{c}n\right)
y^s
\int_{-\infty}^\infty\frac{1}{\left(x^2+y^2\right)^s}e(-nx)dx
\ena
\begin{lemma}[Lemma \ref{lemma-integral-Whitaker-Bessel}]\label{lemma-integral-in-bump}
We have
\bna
\pi^{-s}\Gamma(s)y^s\int_{-\infty}^\infty\frac{1}{(x^2+y^2)^s}e(-nx)dx
=\left\{
\begin{aligned}
&\pi^{-s+\frac{1}{2}}\Gamma(s-\frac{1}{2})y^{1-s},\quad && n=0\\
&2|n|^{s-\frac{1}{2}}\sqrt{y}K_{s-\frac{1}{2}}(2\pi|n|y),\quad &&n\neq 0.
\end{aligned}
\right.
\ena
\end{lemma}
\begin{proof}
Note that
\bna
\Gamma(s)=\int_0^\infty e^{-t}t^{s}\frac{dt}{t}.
\ena
Thus
\bna
&&\pi^{-s}y^s\int_{0}^\infty e^{-t}t^s\frac{dt}{t}
\int_{-\infty}^\infty\frac{1}{(x^2+y^2)^s}e(-nx)dx\\
&=&\int_{-\infty}^\infty\left(\int_0^\infty
\left(\frac{y}{\pi(x^2+y^2)}t\right)^{s} e^{-t}\frac{dt}{t}
\right)dx
=\int_{-\infty}^\infty\left(\int_0^\infty
t^{s} e^{-t\frac{\pi(x^2+y^2)}{y}}\frac{dt}{t}
\right)dx\\
&=&\int_0^\infty
t^{s} e^{-t\pi y}
\left(\int_{-\infty}^\infty
 e^{-\frac{\pi t}{y}x^2} e^{-2\pi i n x}dx\right)
 \frac{dt}{t}
\ena
Since
\bna
\int_{-\infty}^\infty
 e^{-\frac{\pi t}{y}x^2} e^{-2\pi i n x}dx=\left\{
 \begin{aligned}
 &\sqrt{\frac{y}{t}},\quad &n=0\\
 &\sqrt{\frac{y}{t}}e^{-\frac{\pi y n^2}{t}},\quad &n\neq 0
 \end{aligned}
 \right.
\ena
The result follows immediately by the expression of $\Gamma$-function and $K$-Bessel
function.
\end{proof}
\subsubsection{The constant term}
By the above lemma, the constant term of the normalized  Eisenstein series
$E^*(z,s)$ is
\bna
a^*_0(y,s):=\pi^{-s}\Gamma(s)\zeta(2s)a(y,0;s)
&=&\pi^{-s}\Gamma(s)\zeta(2s) y^s +\zeta(2s)
\sum_{c\geq 1}\frac{\varphi(c)}{c^{2s}}
\pi^{-s+\frac{1}{2}}\Gamma(s-1/2)y^{1-s}\\
&=&\pi^{-s}\Gamma(s)\zeta(2s) y^s +\zeta(2s)
\sum_{c\geq 1}\frac{\sum_{d\mid c}\mu(d)\frac{c}{d}}{c^{2s}}
\pi^{-s+\frac{1}{2}}\Gamma(s-1/2)y^{1-s}
\ena
Note that
\bna
\sum_{c\geq 1}\frac{\sum_{d\mid c}\mu(d)\frac{c}{d}}{c^{2s}}
=\sum_{c\geq 1}\frac{\mu(d)}{c^{2s}}\sum_{c\geq 1}\frac{c}{c^{2s}},
\ena
thus
\bna
a^*_0(y,s)&=&\pi^{-s}\Gamma(s)\zeta(2s)y^s+\pi^{-s+\frac{1}{2}}\Gamma(s-\frac{1}{2})
\zeta(2s-1)y^{1-s}
\ena
and
\bna
\zeta(2s-1)=\frac{\pi^{-\frac{1-(2s-1)}{2}}\Gamma\left(\frac{1-(2s-1)}{2}\right)\zeta(1-(2s-1))}
{\pi^{-\frac{2s-1}{2}}\Gamma(\frac{2s-1}{2})}
\ena
Thus finally, the constant of the Eisenstein series is
\bna
a^*_0(y,s)=\pi^{-s}\Gamma(s)\zeta(2s)y^s+\pi^{-(1-s)}\Gamma(1-s)\zeta(2-2s)y^{1-s}.
\ena
Here $s=\frac{1}{2}$ is a possible pole of $a^*_0(y,s)$, with
\bna
\res_{s=\frac{1}{2}}a^*_0(y,s)=\pi^{-\frac{1}{2}}\Gamma(1/2) y^{1/2}\frac{1}{2}
+\pi^{-\frac{1}{2}}\Gamma(1/2) y^{1/2}\frac{-1}{2}=0,
\ena
i.e. $s=\frac{1}{2}$ is not a pole; $s=1$ is a simple pole with residue
\bna
\res_{s=1}a^*_0(y,s)=\frac{1}{2},\quad \res_{s=0}a^*_0(y,s)=-\frac{1}{2}
\ena
\begin{prop}The constant term of $E^*(z,s)$ is
\bna
a^*_0(y,s)=\pi^{-s}\Gamma(s)\zeta(2s)y^s+\pi^{-(1-s)}\Gamma(1-s)\zeta(2-2s)y^{1-s},
\ena
it has meromorphic continuation for $s\in\C$ with simple poles at $s=1$ and $s=0$
with the residue
\bna
\res_{s=1}a^*_0(y,s)=\frac{1}{2},\quad \res_{s=0}a^*_0(y,s)=-\frac{1}{2}.
\ena
Moreover,
\bna
a^*_0(y,s)=a^*_0(y,1-s).
\ena
\end{prop}


\subsubsection{The non-constant term}
Next, we consider the non-constant term. For $n\neq 0$,
\bna
a^*_n(y,s):&=&\pi^{-s}\Gamma(s)\zeta(2s)\int_0^1 E(z,s)e(-nx)dx\\
&=&\zeta(2s)\sum_{c\geq 1}\frac{1}{c^{2s}}\sum_{d\bmod c\atop(d,c)=1}
e\left(\frac{d}{c}n\right)2|n|^{s-\frac{1}{2}}\sqrt{y}K_{s-\frac{1}{2}}(2\pi|n|y).
\ena
Note that
\bna
\sum_{d\bmod c\atop(c,d)=1}e\left(\frac{dn}{c}\right)=
S(0,n;c)=\sum_{\delta\mid (c,n)}\mu(\frac{c}{\delta})\delta,
\ena
is the Ramanujan sum, and
\bna
\sum_{c\geq 1}\frac{1}{c^{2s}}S(0,n;c)
=\sum_{\delta\mid n}\delta \sum_{c\geq 1}\mu(c)\frac{1}{(c\delta)^{2s}}
=\zeta(2s)^{-1}\sigma_{1-2s}(n).
\ena
Thus
\bna
a^*_n(y;s)=
2|n|^{s-\frac{1}{2}}\sigma_{1-2s}(n)\sqrt{y}K_{s-\frac{1}{2}}(2\pi |n|y).
\ena

One has
\bna
K_{s-\frac{1}{2}}=K_{\frac{1}{2}-s}=K_{1-s-\frac{1}{2}}
\ena
and
\bna
|n|^{s-\frac{1}{2}}\sigma_{1-2s}(|n|)&=&\sum_{d\mid |n|}
\left(\frac{|n|}{d^2}\right)^{s-\frac{1}{2}}=\sum_{d_1d_2=|n|}d_1^{s-\frac{1}{2}}d_2^{\frac{1}{2}-s}\\
&=&
|n|^{1-s-\frac{1}{2}}\sigma_{1-2(1-s)}(|n|).
\ena
This gives the following result.
\begin{prop}
The non-constant term
\bna
a^*_n(y;s)=
2|n|^{s-\frac{1}{2}}\sigma_{1-2s}(n)\sqrt{y}K_{s-\frac{1}{2}}(2\pi |n|y)
\ena
has analytic continuation for $s\in\C$ and satisfies
\bna
a^*_n(y;s)=a^*_n(y;1-s).
\ena
\end{prop}
\subsubsection{Conclusion}
\begin{thm}For $E^*(z,s)=\pi^{-s}\Gamma(s)\zeta(2s)E(z,s)$, we have
\bna
E^*(z,s)&=&\pi^{-s}\Gamma(s)\zeta(2s)y^{s}
+\pi^{-(1-s)}\Gamma(1-s)\zeta(2-2s)y^{1-s}\\
&&
+2\sum_{n\neq 0}|n|^{s-\frac{1}{2}}\sigma_{1-2s}(|n|)\sqrt{y} K_{s-\frac{1}{2}}(2\pi |n|y) e(nx).
\ena
$E^*(z,s)$ is defined for $\re(s)>1$ and has meromorphic continuation
to $s\in\C$ and satisfies the functional equation
\bna
E^*(z,s)=E^*(z,1-s).
\ena
Moreover, $s=1$ and $s=0$ are two simple pole of $E^*(z,s)$,
and the resiue at $s=1$ is the constant function (in $z$)
\bna
\res_{s=1}E^*(z,s)=\frac{1}{2},
\ena
and
\bna
E^*(x+iy,s)=O\left( y^{\max{\re(s),1-\re(s)}}\right),\quad y\rightarrow\infty.
\ena
\end{thm}

\subsection{Orthogonal relation with cusp forms}
For $E(z,s)$, we know that
\bna
\overline{E(z,s)}=\sum_{\delta\in \Gamma_\infty\backslash\Gamma}I_{\overline s}(\delta.z)
=E(z,\overline s).
\ena
For $f\in L^2_{cusp}$,
\bna
\langle f,E(,s)\rangle&=&\int_{\Gamma\backslash \mk h} f(z)\overline{E(z,s)}d\mu(z)
\overset{\mbox{unfold}}{=}\int_{\Gamma\backslash \mk h} \sum_{\delta\in\Gamma_\infty\backslash\Gamma}f(\delta.z)
I_{s}(\delta.z)d\mu(\delta.z)\\
&=&\int_{\Gamma_\infty\backslash \mk h} f(z)
I_{s}(z)d\mu(z)
=\int_0^\infty y^s\left(\int_0^1 f(x+iy)dx\right)\frac{dy}{y^2}\\
&=&0.
\ena

\begin{remark}
We refer to section \ref{Sec-Eisenstein} for an overview of Eisenstein series in
representation language.
\end{remark}

\subsection{Inner product with Eisenstein series}
For any $h\in L^2(\Gamma\backslash\mk h)$ with  the Fourier expansion
\bna
h(z)=a_{h,0}(y)+\sum_{n\neq 0} a_{n,h}(y)e(nx),
\ena
and thus
\bna
\langle h, E^*(\,\,,\overline{s})\rangle&=&
\int_{\Gamma\backslash \mk h}h(z)E(z,s)d\mu(z)
\overset{unfold}{=}\int_{\Gamma_\infty\backslash \mk h}h(z) I_s(z)d\mu(z)\\
&=&\int_0^\infty\left(\int_0^1 h(x+iy)dx\right)  y^s\frac{dy}{y^2}\\
&=&\int_0^\infty a_{h,0}(y)  y^{s-1}\frac{dy}{y}
\ena
Thus the inner product of $h\in L^2(\Gamma\backslash\mk h)$ with Eisenstein series
is just the Mellin transform of the constant term $a_{h,0}(y)$ of $h(z)$.
In the next section, we use this fact to derive the analytic property of the Rankin-Selberg
$L$-function.

\subsection{Application of Eisenstein series -Rankin-Selberg integrals}
We start from two cusp forms, $f$ and $g$ with
\bna
f(z)&=&\sum_{n\neq 0}a_f(n)\sqrt{y}K_{s-\frac{1}{2}}(2\pi|n|y)e(nx),\\
g(z)&=&\sum_{n\neq 0}a_g(n)\sqrt{y}K_{s-\frac{1}{2}}(2\pi|n|y)e(nx),
\ena
Note that $f(z)E(z,s)$ and $g(z)$ are both vanishes at the cusp $i\infty$ and thus
\bna
I(s,f,g)=\langle fE(*,s), g\rangle =\int_{\Gamma\backslash \mk h}f(z)\overline{g(z)}E(z,s)d\mu(z)
\ena
are well defined.

For $\re(s)>1$, by unfolding Eisenstein series we have
\bna
I(s,f,g)&=&\int_{\Gamma\backslash\mk h}\sum_{\delta\in\Gamma_\infty\backslash\Gamma}
f(\delta.z)\overline{g(\delta.z)}I_s(\delta.z)d\mu(\delta.z)\\
&=&\int_{0}^1\int_{0}^\infty f(z)\overline{g(z)}y^sdx\frac{dy}{y^2}.
\ena
Applying the Fourier expansion of $f(z)$ and $g(z)$,
\bna
I(s,f,g)&=&\sum_{m\neq  0}\sum_{n\neq 0}a_g(m)\overline{a_f(n)}
\left\{\int_{0}^\infty
K_{it_f}(2\pi|m|y)K_{it_g}(2\pi|n|y)
y^{s+1}\frac{dy}{y^2}
\right\}\int_0^1 e((m-n)x)dx\\
&=&\left(a_f(1)\overline{a_g(1)}
+a_f(-1)\overline{a_g(-1)}\right)
\sum_{m> 0}\lambda_f(m)\overline{\lambda_g(m)}
\int_{0}^\infty
K_{it_f}(2\pi|m|y)K_{it_g}(2\pi|n|y)
y^{s}\frac{dy}{y}.
\ena
\begin{lemma}
We have
\bna
\int_{0}^\infty K_{\mu}(y)K_{\nu}(y)y^s\frac{dy}{y}
=2^{s-3}\frac{\Gamma\left(\frac{s-\mu-\nu}{2}\right)
\Gamma\left(\frac{s-\mu+\nu}{2}\right)
\Gamma\left(\frac{s+\mu-\nu}{2}\right)
\Gamma\left(\frac{s+\mu+\nu}{2}\right)
}{\Gamma(s)}
\ena
\end{lemma}
\begin{proof}
Recall \eqref{integral-transform-K-Bessel-in-appendix},
\bna
K_s(y)=\frac{1}{2}\int_0^\infty e^{-\frac{y}{2}(t+\frac{1}{t})}t^s\frac{dt}{t}.
\ena
Thus
\bna
I=2^{-2}\int_0^\infty\int_0^\infty\int_0^\infty
e^{-\frac{y}{2}(t_1+\frac{1}{t_1})}
e^{-\frac{y}{2}(t_2+\frac{1}{t_2})}t_1^{\mu}t_2^{\nu}\frac{dt_1}{t_1}
\frac{dt_2}{t_2} y^s\frac{dy}{y}.
\ena
Changing variable in a suitable situation, we will obtain the result.
\end{proof}
Therefore,
\bna
I(s,f,g)&=&
\left(a_f(1)\overline{a_g(1)}
+a_f(-1)\overline{a_g(-1)}\right)
\sum_{m> 0}\lambda_f(m)\overline{\lambda_g(m)}
(2\pi m)^{-s}
\int_{0}^\infty
K_{it_f}(y)K_{it_g}(y)
y^{s}\frac{dy}{y}\\
&=&
\left(a_f(1)\overline{a_g(1)}
+a_f(-1)\overline{a_g(-1)}\right)(2\pi)^{-s}
\sum_{m> 0}\frac{\lambda_f(m)\overline{\lambda_g(m)}}{m^s}
2^{s-3}\prod_{\epsilon_f,\epsilon_g\in\{\pm 1\}}
\frac{\Gamma\left(\frac{s+\epsilon_f it_f+\epsilon_git_g}{2}\right)}{\Gamma(s)}
\ena
and
\bna
I^*(s,f,g)&=&\pi^{-s}\Gamma(s)\zeta(2s)I(s,f,g)\\
&=&\frac{\left(a_f(1)\overline{a_g(1)}
+a_f(-1)\overline{a_g(-1)}\right)}{8}
\pi^{-2s}\prod_{\epsilon_f,\epsilon_g\in\{\pm 1\}}
\Gamma\left(\frac{s+\epsilon_fit_f+\epsilon_git_g}{2}\right)
L(s,f\otimes g)
\ena
where
\bna
L(s,f\otimes g)=\zeta(2s)\sum_{n\geq 1}\frac{\lambda_f(n)\overline{\lambda_g(n)}}{n^s}
\ena
is called the Rankin-Selberg convolution $L$-functions. We also denote by
\bna
\Lambda(s,f\otimes g)=\pi^{-2s}\prod_{\pm,\pm}\Gamma\left(\frac{s\pm it_f\pm it_g}{2}\right)
L(s,f\otimes g)
\ena
as the complete $L$-function and thus
\bna
I(s,f,g)=\frac{a_f(1)\overline{a_g(1)}}{4}\Lambda(s,f\otimes g).
\ena
for $f$ and $g$ both even or odd.


On the other hand, Recall that
\bna
E^*(z,s)=E^*(z,1-s),
\ena
and $s=1$ is a simple pole of $E^*(z,s)$ with residue $\frac{1}{2}$,
 we have
\bna
I^*(s,f,g)=I^*(1-s,f,g),
\ena
which gives the functional equation
\bna
\Lambda(s,f\otimes g)=\Lambda(1-s,f\otimes g),
\ena
 and
\bna
\res_{s=1}I^*(z,f,g)&=&\frac{a_f(1)\overline{a_g(1)}}{4}\res_{s=1}\Lambda(s,f\otimes g)\\
&=&\frac{1}{2}\langle f,g\rangle.
\ena
\begin{lemma}For $f$ and $g$ be two normalized Maass cusp form,
\bna
\langle f,g\rangle =\left\{
\begin{aligned}
&\|f\|^2,\quad &f=g\\
&0,\quad &f\neq g
\end{aligned}
\right.
\ena
\end{lemma}
\begin{proof}Note that we choose $f$ and $g$ be orthogonal basis of $L^2_{cusp}$,
it is obviously.
\end{proof}

\begin{prop}
Let $f,g\in\mathcal B_{cusp}$ be two even or odd maass cusp forms.
 We have
\bna
I^*(s,f,g):&=&\int_{\Gamma\backslash\mk h}f(z)\overline{g(z)}E(z,s)d\mu(z)
=\frac{a_f(1)\overline{a_g(1)}}{4}\Lambda(s,f\otimes g),
\ena
where
\bna
\Lambda(s,f\otimes g)=\pi^{-2s}\prod_{\pm}\prod_{\pm}\Gamma\left(\frac{s\pm it_f\pm it_g}{2}\right)
L(s,f\otimes g)
\ena
with
\bna
L(s,f\otimes g)=\zeta(2s)\sum_{n\geq 1}\frac{\lambda_f(n)\overline{\lambda_g(n)}}{n^s}.
\ena
Then $\Lambda (s,f\otimes g)$ has analytic continuation for $s\in\C$ except for a possible simple pole at $s=1$
and $s=0$ if $f=g$, in which case
\bna
\res_{s=1}\Lambda(s,f\otimes f)=\frac{2\langle f,f\rangle }{|a_f(1)|^2}
=\left\{
\begin{aligned}
&\frac{2}{|a_f(1)|^2},\quad &&\mbox{if we normaliz $f$ to be orthornormal basis, i.e. $\langle f,f\rangle=1$}\\
&2\langle f,f\rangle
\quad &&\mbox{if we normaliz $f$ to be $a_f(1)=1$.}
\end{aligned}
\right.
\ena

\end{prop}
\begin{remark}[Real coefficients]
Note that $\lambda_f(n)$ are eigen values of the Hecke operators, and the Hecke operators
 are self-dual and thus $\lambda_f(n)$ are real!
\end{remark}

\subsection{Euler products of Rankin-Selberg $L$-functions}
Note that
\bna
L(s,f)&=&\sum_{n\geq 1}\frac{\lambda_f(n)}{n^s}=\prod_{p}(1-\alpha_1(p)p^{-s})^{-1}
(1-\alpha_2(p)p^{-s})^{-1}\\
L(s,g)&=&\sum_{n\geq 1}\frac{\lambda_g(n)}{n^s}=\prod_{p}(1-\beta_1(p)p^{-s})^{-1}
(1-\beta_2(p)p^{-s})^{-1}
\ena
\begin{lemma}[Lemma 1.6.1 in Bump]
If
\bna
\sum_{r=0}^\infty A(r)x^r=(1-\alpha_1x)^{-1}(1-\alpha_2x)^{-1}\\
\sum_{r=0}^\infty B(r)x^r=(1-\beta_1x)^{-1}(1-\beta_2x)^{-1}
\ena
then
\bna
\sum_{r=0}^\infty A(r)B(r)x^r=\left(1-\alpha_1\alpha_2\beta_1\beta_2 x^2\right)
\prod_{i,j=1}^2
(1-\alpha_i\beta_j x)^{-1}.
\ena
\end{lemma}

By the above lemma, we have
\bna
L(s,f\otimes g)=\prod_{p}\prod_{i=1}^2\prod_{j=1}^2
\left(1-\alpha_i(p)\beta_j(p)p^{-s}\right)^{-1}
\ena
Specially, if $f=g$,
\bna
L(s,f\otimes f)
&=&\prod_p\prod_{i,j=1}^2\left(1-\alpha_i(p)\overline\alpha_j(p)p^{-s}\right)\\
&=&\zeta(s)\prod_{p}(1-\alpha_1^2(p)p^{-s})^{-1}(1-p^{-s})^{-1}(1-\alpha_2^2p^{-s})^{-1}\\
&=&\zeta(s)L(s,sym^2f).
\ena
and
\bna
\res_{s=1}L(s,f\otimes f)=L(1,sym^2f).
\ena
The symmetric square $L$-function is another story.




\section{Poincare series 1 - General definition}
The Eisenstein series is constructed as follows.
We start from the function
\bna
\tilde h_0(z):=I_s(z)
\ena
which is an eigenfunction of $\Delta$, invariant under $\Gamma_\infty$,
 and then construct
 \bna
 E(z,s):=\sum_{\gamma\in\Gamma_\infty\backslash\Gamma}f_0(\gamma.z)
 \ena
The most important property is that the inner product of Eisenstein series
 and automorphic forms involves the constant term in the Fourier expansion
 of $f$,
 \bna
 \langle f,E(z,\overline s)
 \rangle
 =\int_{0}^1\int_{0}^\infty f(z) h_0(z)\frac{dxdy}{y^2}
 =\int_0^\infty\left( \int_0^1f(x+iy)e_n(x)dx\right)y^s\frac{dy}{y^2}
 \ena

\subsection{Poincare series - definition}
Following the idea above, we can construct a lot of automorphic forms as follows.
\bit
\item We consider $\Gamma_\infty$-functions. Set
\bna
\tilde h(z):=h(\im z)e(m\re z)
\ena
with $m\in \Z$ and $f\in C_c^\infty((0,\infty))$. It is naturally a function which
is invariant under $\Gamma_\infty$.
As $m\in\Z$ and $h\in C_c^\infty((0,\infty))$ varies, it varies over allmost all
these functions.
\item We construct
\bna
\sum_{\gamma\in\Gamma_\infty\backslash \Gamma} h(\im \delta. z)e(m\re \delta z)
\ena
\item Note that $e(mz)=e(mx)e^{-2\pi y}$.
Instead of the above defintion, Kuznetsov us
\bna
P_m(z,h):=\sum_{\gamma\in\Gamma_\infty\backslash \Gamma} h(\im \delta. z)e(m z)
\ena
which is called the Poincare series. It is well-defined by the following lemma.
\eit

\begin{lemma}Let $T>0$. Let $z$ be in the fundamental domain.
The number
\bna
\{\gamma\in\Gamma_\infty\backslash\Gamma,\quad \im\delta.z>T\}
\ena
is finite.
\end{lemma}
\begin{proof}
Recall that
\bna
\Gamma_\infty\backslash\Gamma=
\bma 1\\&1\ema\bigcup \{\bma *&*\\ c&d\ema\bma 1&m\\&1\ema,\quad c>0,b\bmod c,(b,c)=1,m\in\Z\}
\ena
Note that for fixed $z=x_0+iy_0\in\mathcal F$ and  $\delta=\bma &\\ c&d\ema\bma 1&m\\&1\ema$ be a representative element in the coset $\Gamma_\infty\backslash\Gamma$,
 \bna
 \im \delta. z=\frac{y_0}{|c(x_0+m+iy_0)+d|^2}>T
 \leftrightarrow (cx_0+m+d)^2+ c^2y_0^2<\frac{y_0}{T}
 \ena
 Obviously there are only finite number choice of such pair $c>0$,
  and hence $d\bmod c$ with $(d,c)=1$, and hence $m$.
\bna
\ena
\end{proof}

\subsection{Poncare series - Inner product with automorphic forms}
Now,
\bea
\nonumber\langle f,P_m(z,h)\rangle
&=&\int_{\Gamma_\infty\backslash\Gamma}f(z)\overline{P_m(z,h)}\frac{dxdy}{y^2}\\
\nonumber&=&\int_0^\infty \overline{h(y)}e^{-2\pi my} \left(\int_0^\infty f(x+iy)e(-mx)dx\right)\frac{dy}{y^2}\\
&=&\int_0^\infty \overline{h(y)} e^{-2m\pi y} a_f(m,y)\frac{dy}{y^2}
\label{inner-product-with-Poincare-series}
\eea
where $a_f(m,y)$ is the $m$-th Fourier coefficients of $f$.

Note that for $f\in L^2(\Gamma\backslash\mk h)$, $f(z)=\sum_{n} a_f(n,y)e(nx)$,
we know $f\equiv 0$ iff $a_f(n,y)=0$ for all $n$.
The inner product of $f$ with $P_m(z,h)$ implies that
\bna
\left\{ P_{m}(z,h),\quad m\in \Z, h\in C_c^\infty(\mk h)\right\}
\ena
spans $L^2(\Gamma\backslash \mk h)$.
Especially, for the case $m=0$, i.e.
\bna
P(z,h):=P_0(z,h)=\sum_{\delta\in\Gamma_\infty\backslash\Gamma}h(\im(\delta.z)),
\ena
called Pseudo-Eisenstein series, which are orthogonal to $L^2_{cusp}$ and spans $L^2-L^2_{cusp}$.
\begin{prop}
Poincare series $P_m(z,h)$ span $L^2$, and Pseudo-Eisenstein series $P(z,h)$
span $L^2-L^2_{cusp}$.
\end{prop}
\begin{remark}
To study $L^2_{cusp}$, we need to express $P(z,h)$ in terms of sum (integral)
over eigenfunctions of $\Delta$.
\end{remark}

\subsection{Poincare series - Fourier expansion}
We consider the $n$-th Fourier coefficients of $P_m(z,h)$, namely
\bna
a_{m,h}(n)=\int_0^1 P_m(z,h)e(-nx)dx
\ena
By double coset decomposition in Lemma \ref{lemma-doublecoset}, we have
\bna
a_{m,h}(n)&=&\int_0^1 h(\im (z))e(m z)e(-nx)dx\\
&&+\int_0^1\sum_{c\geq 1}\sideset{}{^*}\sum_{d\bmod c}\sum_{\ell\in\Z}
h\left(\im\left(\bma a&b\\c&d\ema\bma1&\ell\\&1\ema.z\right)\right)
e\left(m\bma a&b\\c&d\ema\bma1&\ell\\&1\ema.z\right)
e(-nx)dx\\
&=&
h(y)e^{-2\pi my}\delta_{m,n}+
\sum_{c\geq 1}\sideset{}{^*}\sum_{d\bmod c}
\int_{-\infty}^\infty
h\left(\frac{y}{(cx+d)^2+c^2y^2}\right)
e\left(m\frac{ax+b+iay}{cx+d+icy}\right)
e(-nx)dx\\
&\overset{x+\frac{d}{c}\mapsto x}=&
h(y)e^{-2\pi my}\delta_{m,n}+
\sum_{c\geq 1}\sideset{}{^*}\sum_{d\bmod c}
\int_{-\infty}^\infty
h\left(\frac{y}{c^2x^2+c^2y^2}\right)
e\left(m\frac{a(x-\frac{d}{c})+b+iay}{c(x-\frac{d}{c})+d+icy}\right)
e(-n(x-\frac{d}{c}))dx\\
&=&
h(y)e^{-2\pi my}\delta_{m,n}+
\sum_{c\geq 1}\sum_{ad\equiv 1\bmod c}e\left(\frac{am+dn}{c}\right)
\int_{-\infty}^\infty
h\left(\frac{y}{c^2x^2+c^2y^2}\right)
e\left(-\frac{m}{c^2x+ic^2y}-nx\right)dx
\ena

\subsection{A Remark on Mellin transform}
We use the following notation,
\bna
H(s):=\int_0^\infty h(t)t^{\color{red}-s}\frac{dt}{t},
\quad h(y)=\frac{1}{2\pi i}H(s)y^{\color{red}s}ds
\ena
which coincides with the notation in Arthur's notes, and is different with the original Mellin transfrom.
\section{Pseudo Eisenstein series and the spectrum decomposition}
We are interested in $L^2-L^2_{cusp}$.
Note that for $m=0$,
\bna
P(z,h)=P_0(z,h)=\sum_{\delta\in\Gamma_\infty\backslash\Gamma}h(\delta.z)
\ena
which  is orthogonal to $L^2_{cusp}$, called \underline{Psudo-Eisenstein series}.
\subsection{Relation with Eisenstein series}
Note that $h\in C_c^\infty((0,\infty))$. We need the spectral decomposition of $L^2(0,\infty)$, which is related to Mellin transform, namely
\bna
H(s):=\langle f,*^{s}\rangle= \int_0^\infty f(y)y^{-s}\frac{dy}{y},
\quad
h(y)=\frac{1}{2\pi i}\int_{\re(s)=\sigma}H(s) y^{s}ds
\ena
Applying this one has
\bna
P(z,h)&=&\sum_{\delta\in\Gamma_\infty\backslash\Gamma}h(\delta.z)=\sum_{\delta\in\Gamma_\infty\backslash\Gamma}
\frac{1}{2\pi i}\int_{\re(s)=\sigma}H(s) I_{s}(\delta.z)ds\\
&=&\frac{1}{2\pi i}\int_{\re(s)=\sigma} H(s)E(z,s)ds
\ena


\subsection{Constant term}
Psudo Eisenstein series is orthogonal to cusp forms.
By the double coset decomposition, its constant term is
\bna
\int_0^1P(z,h)dx&=&h(y) + \sum_{c\geq 1}\sideset{}{^*}\sum_{d\bmod c}\int_{-\infty}^\infty h\left(\frac{y}{(cx+d)^2+c^2y^2}\right)dx\\
&=&h(y) + \sum_{c\geq 1}\varphi(c)\int_{-\infty}^\infty h\left(\frac{y}{c^2(x^2+y^2)}\right)dx\\
\ena
For $h\in C_c^\infty(0,\infty)$,  we decompose $h(y)$
as integral (sum) with the power function $y^s$ in $y$.
Recall that for $h(y)$, we have
\bna
H(s)=\int_0^\infty h(t)t^{-s}\frac{dt}{t},\quad h(y)=\frac{1}{2\pi i}
\int_{(\sigma)}H(s)y^{s}ds.
\ena
Thus
\bna
\int_0^1P(z,h)dx=h(y) + \int_{(\sigma)}
\frac{1}{2\pi i}H(s)
\sum_{c\geq 1}\varphi(c)
\frac{y^s}{c^{2s}}
\left\{\int_{-\infty}^\infty \left(\frac{1}{x^2+y^2}\right)^sdx\right\}
\ena

By lemma \ref{lemma-integral-in-bump}, we have
\bna
\int_{-\infty}^\infty \frac{1}{(x^2+y^2)^s}dx=\pi^{1/2}\frac{\Gamma(s-\frac{1}{2})}{\Gamma(s)}y^{1-2s}.
\ena
and hus
\bna
&&\sum_{c\geq 1}\varphi(c)\frac{1}{c^{2s}}y^s\int_{-\infty}^\infty \frac{1}{(x^2+y^2)^s}dx
=y^{1-s}\pi^{1/2}\frac{\Gamma(s-\frac{1}{2})}{\Gamma(s)}\sum_{c\geq 1}\frac{\varphi(c)}{c^{2s}}\\
&=&y^{1-s}\frac{\pi^{-(s-\frac{1}{2})}\Gamma(s-\frac{1}{2})\zeta(2s-1)}{\pi^{-s}\Gamma(s)\zeta(2s)}
=y^{1-s}\frac{\xi(2s-1)}{\xi(2s)}
\ena
This gives that
\bea
\int_0^1P(z,h)dx=h(y) +
\frac{1}{2\pi i}\int_{\sigma} H(s)y^{1-s}\frac{\xi(2s-1)}{\xi(2s)}ds.\label{constant-term-P-Eisenstein-series}
\eea
\subsection{Inner products of Pseudo-Eisenstein series}
Now, we consider
\bna
\langle P(,h_1),P(,h_2)\rangle&=&\int_0^\infty \overline {h_2(y)}
\left(h_1(y) + \sum_{c\geq 1}\varphi(c)\int_{-\infty}^\infty h_1\left(\frac{y}{c(x^2+y^2)}\right)dx\right)
\frac{dy}{y^2}\\
&=&\int_{0}^\infty h_1(y)\overline{h_2(y)}\frac{dy}{y^2}
+\int_{0}^\infty\overline{h_2(y)} \left(\sum_{c\geq 1}\varphi(c)\int_{-\infty}^\infty h_1\left(\frac{y}{c(x^2+y^2)}\right)dx\right)\frac{dy}{y^2}.
\ena

As a function in $y$,
\bna
y\mapsto \sum_{c\geq 1}\varphi(c)\int_{-\infty}^\infty h_1\left(\frac{y}{c(x^2+y^2)}\right)dx
\ena
is not compactly supported on $(0,\infty)$,
To tackle this problem,  we apply the Mellin transform (valid for $\re(s)=\sigma$ large), see \eqref{constant-term-P-Eisenstein-series}, one has
\bna
\langle P(,h_1),P(,h_2)\rangle&=&\int_0^\infty \overline {h_2(y)}
\left(h_1(y) +
\frac{1}{2\pi i}\int_{\sigma} H_1(s)y^{1-s}\frac{\xi(2s-1)}{\xi(2s)}ds\right)
\frac{dy}{y^2}\\
&=&\int_0^\infty h_1(y)\overline{h_2(y)}\frac{dy}{y^2}
+\int_0^\infty \overline {h_2(y)}
\frac{1}{2\pi i}\int_{\sigma} H_1(s)y^{1-s}\frac{\xi(2s-1)}{\xi(2s)}ds
\frac{dy}{y^2}\\
&=&\int_0^\infty h_1(y)\overline{h_2(y)}\frac{dy}{y^2}
+
\frac{1}{2\pi i}\int_{(\sigma)}\frac{\xi(2s-1)}{\xi(2s)}H_1(s)
\left\{\int_0^\infty \overline {h_2(y)}y^{1-s}\frac{dy}{y^2}\right\}ds\\
&=&\int_0^\infty h_1(y)\overline{h_2(y)}\frac{dy}{y^2}
+
\frac{1}{2\pi i}\int_{(\sigma)}\frac{\xi(2s-1)}{\xi(2s)}H_1(s)
\left\{\overline {\int_0^\infty h_2(y)y^{-\overline{s}}\frac{dy}{y}}\right\}ds\\
&=&\int_0^\infty h_1(y)\overline{h_2(y)}\frac{dy}{y^2}
+
\frac{1}{2\pi i}\int_{(\sigma)}\frac{\xi(2s-1)}{\xi(2s)}H_1(s)\overline{H_2(\overline s)}ds
\ena
\begin{remark}
Note that the convergence problem are now
\bna
\frac{\xi(2s-1)}{\xi(2s)}
\ena
which has meromoprhic continuation now.  So we can move the integral line from
$\re(s)=\sigma>1$ to $\re(s)=1/2$, passing simple pole at $s=1$.
\end{remark}
The first term has no convergence problem, and one has
\bna
\int_0^\infty h_1(y)\overline{h_2(y)}\frac{dy}{y^2}
&=&
\int_0^\infty
\left\{\frac{1}{2\pi i}\int_{(\sigma)}H_1(s)y^{s}ds\right\}
\overline{h_2(y)}\frac{dy}{y^2}\\
&=&\frac{1}{2\pi i}\int_{(\sigma)}H_1(s)\left\{
\overline{\int_0^\infty h_2(y) y^{-(1-\overline{s})} \frac{dy}{y}}\right\}ds\\
&=&\frac{1}{2\pi i}\int_{\sigma}H_1(s)\overline{H_2(1-\overline s)}ds\\
&\overset{\sigma=1/2}=&\frac{1}{2\pi }\int_{-\infty}^\infty H_1(\frac{1}{2}+it)\overline{H_2(\frac{1}{2}+it)}dt,
\ena
and the second term, we move the line of integration to $\re(s)=1/2$, passing a simple pole at $s=1$ coming from $\zeta(2s-1)$, one has
\bna
&&\frac{1}{2\pi i}\int_{(\sigma)}\frac{\xi(2s-1)}{\xi(2s)}H_1(s)\overline{H_2(\overline s)}ds\\
&=&\frac{1}{2\pi}\int_{-\infty}^\infty
\frac{\xi(2it)}{\xi(1+2it)}
H_1(\frac{1}{2}+it)\overline{H_2(\frac{1}{2}-it)}dt+ H_1(1)\overline{H_2(1)}
\frac{1}{2}\frac{\pi^{-1/2}\Gamma(1/2)}{\pi^{-1}\Gamma(1)\zeta(2)}\\
&=&\frac{3}{\pi} H_1(1)\overline {H_2(1)}+\frac{1}{2\pi}\int_{-\infty}^\infty
H_1(1/2+it)\overline{H_2(1/2-it)}
\frac{\xi(1-2it)}{\xi(1+2it)}
dt
\ena
and thus we have
\bna
\langle P(,h_1),P(,h_2)\rangle
=\frac{3}{\pi}H_1(1)\overline{H_2(1)}+
\frac{1}{2\pi }\int_{-\infty}^\infty H_1(\frac{1}{2}+it)\overline{
\left\{
H_2(\frac{1}{2}+it)+\frac{\xi(1+2it)}{\xi(1-2it)}H_2(\frac{1}{2}-it)\right\}
}dt.
\ena
Moreover,
note that
\bna
I&=&\frac{1}{2\pi }\int_{-\infty}^\infty H_1(\frac{1}{2}+it)\overline{
\left\{
H_2(\frac{1}{2}+it)+\frac{\xi(1+2it)}{\xi(1-2it)}H_2(\frac{1}{2}-it)\right\}
}dt\\
&\overset{-t\mapsto t}=&
\frac{1}{2\pi }\int_{-\infty}^\infty H_1(\frac{1}{2}-it)\overline{
\left\{
H_2(\frac{1}{2}-it)+\frac{\xi(1-2it)}{\xi(1+2it)}H_2(\frac{1}{2}+it)\right\}
}dt\\
&=&
\frac{1}{2\pi }\int_{-\infty}^\infty H_1(\frac{1}{2}-it)
\overline{\frac{\xi(1-2it)}{\xi(1+2it)}}\overline{
\left\{\frac{\xi(1+2it)}{\xi(1-2it)}
H_2(\frac{1}{2}-it)+H_2(\frac{1}{2}+it)\right\}
}dt\\
&=&
\frac{1}{2\pi }\int_{-\infty}^\infty H_1(\frac{1}{2}-it)
\frac{\xi(1+2it)}{\xi(1-2it)}\overline{
\left\{\frac{\xi(1+2it)}{\xi(1-2it)}
H_2(\frac{1}{2}-it)+H_2(\frac{1}{2}+it)\right\}
}dt
\ena
and thus
\bna
I=\frac{1}{4\pi}\int_0^\infty\left(H_1(\frac{1}{2}+it)+\frac{\xi(1+2it)}{\xi(1-2it)}
H_1(\frac{1}{2}-it)\right)
\overline{\left(H_2(\frac{1}{2}+it)+\frac{\xi(1+2it)}{\xi(1-2it)}
H_2(\frac{1}{2}-it)\right)}dt
\ena
Therefore, finally we have the following proposition.
\begin{prop}\label{prop-main-prop-on-innerproduct}
We have
\bna
\langle P(,h_1),P(,h_2)\rangle
&=&\frac{3}{\pi}H_1(1)\overline{H_2(1)}\\
&&+\frac{1}{4\pi}\int_0^\infty\left(H_1(\frac{1}{2}+it)+\frac{\xi(1+2it)}{\xi(1-2it)}
H_1(\frac{1}{2}-it)\right)
\overline{\left(H_2(\frac{1}{2}+it)+\frac{\xi(1+2it)}{\xi(1-2it)}
H_2(\frac{1}{2}-it)\right)}dt.
\ena
\end{prop}
Recall
\bna
P(z,h)=\frac{1}{2\pi i}\int_{\sigma}H_1(s)E(z,s)ds.
\ena
We have the following.
\bit\item Firstly,
\bna
\langle P(z,h),1\rangle&=&
\int_{\Gamma\backslash\mk h}P(z,h) \overline {1}d\mu(z)
\overset{unfold}=\int_{0}^\infty h(y)\frac{dy}{y^2}\\
&=&\int_{0}^\infty h(y)y^{-1}\frac{dy}{y}=
H(1),
\ena
\item For $s=\frac{1}{2}+it$,
\bna
\langle P(z,h),E(z,1/2+it)\rangle
&  =&\int_{\Gamma\backslash\mk h} P(z,h) E(z,\overline s)d\mu(z)\\
&\overset{\mbox{\tiny unfold $P(z,h)$}}=&
\int_{0}^\infty h(y)\left\{\int_{0}^1 E(z,\overline{s})dx\right\}\frac{dy}{y^2}\\
&=&\int_0^\infty h(y) \left(y^{\overline s}+\frac{\xi(2\overline s-1)}{\xi(2\overline s)}y^{1-\overline s}\right)\frac{dy}{y^2}\\
&=&\int_0^\infty h(y)y^{-(\frac{1}{2}+it)}\frac{dy}{y}
+\frac{\xi(-2it)}{\xi(1-2it)}\int_0^\infty h(y)y^{-(\frac{1}{2}-it)}\frac{dy}{y}\\
&=&H(\frac{1}{2}+it)+\frac{\xi(1+2it)}{\xi(1-2it)}H(\frac{1}{2}-it)
\ena
\eit
By the above argument, the main proposition in prop \ref{prop-main-prop-on-innerproduct} can be expressed as the following.

\begin{thm}\label{main-thm-inner product-of-P-series}
We have
\bna
\langle P(,h_1),P(,h_2)\rangle
&=&\frac{3}{\pi}\langle P(,h_1),1\rangle
\overline{\langle P(,h_2),1\rangle}\\
&&+\frac{1}{4\pi}
\int_{-\infty}^\infty
\langle P(,h_1),E(,1/2+it)\rangle
\overline{\langle P(,h_2),E(,1/2+it)\rangle}
dt.
\ena
It gives that
\bna
P(z,h)=\frac{3}{\pi}\langle P(,h),1\rangle+\frac{1}{4\pi}
\int_{-\infty}^\infty \langle P(z,h),E(z,1/2+it)\rangle E(z,1/2+it)dt.
\ena
\end{thm}

\subsection{Main theorem on the spectral decomposition}
\begin{thm}We have $L^2=L^2_{cusp}+L^2_{res}+L^2_{cont}$,
where $L^2_{cusp}$ is the space of cusp forms,
$L^2_{cont}$ is the space of continuous spectrum, consisting of $E(z,\frac{1}{2}+it)$, and $L^2_{res}$ is the residue spectrum coming from the residue of Eisenstein series. Moreover, $L^2_{cusp}+L^2_{res}=L^2_{disc}$ is the space
 of discrete spectrum.
 Given $f\in L^2$, we have
 \bna
 f(z)&=&\sum_{\varphi\in\mathfrak B_{cusp}}\frac{\langle f,\varphi \rangle}{\langle\varphi,\varphi\rangle}
 +\frac{3}{\pi}\int_{\Gamma\backslash\mk h}f(z)d\mu(z)+\frac{1}{4\pi}\int_{-\infty}^\infty \langle f, E(*,\frac{1}{2}+it)\rangle
 E(z,\frac{1}{2}+it)dt.
 \ena
\end{thm}
\begin{thm}\label{Parseval-identity}
We have the Parseval identity
and the Parseval identity
\bea
\langle f,g\rangle
&=&\frac{3}{\pi}\langle f,1\rangle
\overline{\langle g,1\rangle}\nonumber\\
&&+\frac{1}{4\pi}+\sum_{\varphi\in\mathcal B_{cusp}}\frac{\langle f,\varphi\rangle\overline{\langle g,\varphi\rangle}}{\langle\varphi,\varphi\rangle}
\int_{-\infty}^\infty
\langle f,E(,1/2+it)\rangle
\overline{\langle g,E(,1/2+it)\rangle}
dt.\label{Parseval-identity}
\eea

\end{thm}
\subsection{Another way}
Instead of the constant term of the Eisenstein series,
we can obtain the spectral decomposition (formally) via the relation between
$P$-series an Eisenstein series an the global propery of $E(z,s)$ as follows.

For $\re(s)>1$, by the definition of the Eisenstein series, we have
\bna
P(z,h_1)=\frac{1}{2\pi i}\int_{(\sigma)}H_1(s)E(z,s)ds.
\ena
Thus
\bna
\langle P(,h_1),P(,h_2)\rangle=\int_{\Gamma\backslash\mk h}
\left\{\frac{1}{2\pi i}\int_{\sigma}H_1(s)E(z,s)ds\right\}\overline{P(z,h_2)}dz
\ena
Moving the line of the integration to $\re(s)=1/2$, passing a simple pole at $s=1$
with  the residue
\bna
&&\frac{1}{2\pi^{-1}\Gamma(1)\zeta(2)}\int_{\Gamma\backslash\mk h}H_1(1)\overline{P(z,h_2)}d\mu(z)\\
&=&\frac{3}{\pi}H_1(1)\int_{\Gamma\backslash\mk h} \overline{P(z,h_2)}d\mu(z),
\ena
one has
\bna
\langle P(,h_1),P(,h_2)\rangle
&=&H_1(1)\int_{\Gamma\backslash\mk h}\frac{3}{\pi} \overline{P(z,h_2)}d\mu(z)
+\frac{1}{2\pi}\int_{-\infty}^\infty H_1(\frac{1}{2}+it)
\int_{\Gamma\backslash \mk h}E(z,\frac{1}{2}+it)\overline{P(z,h_2)}d\mu(z)dt.
\ena

Next, by the functional equation of the Eisenstein series,
\bna
E(z,s)=\frac{\xi(2s-1)}{\xi(2s)}E(z,1-s),
\ena
we have
\bna
\langle P(,h_1),P(,h_2)\rangle
&=&H_1(1)\int_{\Gamma\backslash\mk h}\frac{3}{\pi} \overline{P(z,h_2)}d\mu(z)\\
&&+\frac{1}{4\pi}\int_{-\infty}^\infty H_1(\frac{1}{2}+it)
\int_{\Gamma\backslash \mk h}E(z,\frac{1}{2}+it)\overline{P(z,h_2)}d\mu(z)dt.\\
&&+\frac{1}{4\pi}\int_{-\infty}^\infty H_1(\frac{1}{2}+it)
\frac{\xi(2it)}{\xi(1+2it)}
\int_{\Gamma\backslash \mk h}E(z,\frac{1}{2}-it)\overline{P(z,h_2)}d\mu(z)dt\\
&=&H_1(1)\int_{\Gamma\backslash\mk h}\frac{3}{\pi} \overline{P(z,h_2)}d\mu(z)\\
&&+\frac{1}{4\pi}\int_{-\infty}^\infty
\left(H_1(\frac{1}{2}+it)+\frac{\xi(2it)}{\xi(1+2it)}H_1(\frac{1}{2}-it)\right)
\int_{\Gamma\backslash \mk h}E(z,\frac{1}{2}+it)\overline{P(z,h_2)}d\mu(z)dt.
\ena
This gives
\bna
P(z,h)
=H_1(1)\frac{3}{\pi}+\frac{1}{4\pi}\int_{-\infty}^\infty
\left(H_1(\frac{1}{2}+it)   +
\frac{\xi(2it)}{\xi(1+2it)}
H_1(\frac{1}{2}-it)\right)E(z,\frac{1}{2}+it)dt.
\ena



\section{Poincare series and the Kuznetsov's trace formfula}
We know that
\bna
L^2(\Gamma\backslash\mk h)=L^2_{cusp}\oplus L^2_{res}\oplus L^2_{cont}
\ena
The space $L^2_{res}\oplus L^2_{cont}$ is clearly.
The problem is, how about the space $L^2_{cusp}$? How to study
the basis of $L^2_{cusp}$?

\subsection{Redefine the Poincare series}
Recall that Poincare series
\bna
P_m(z,h):=\sum_{\delta\in\Gamma_\infty\backslash\Gamma}h(\im \delta.z)e(m\delta.z)
\ena
spans $L^2(\Gamma\backslash\mk h)$. For $m>0$,
\bna
y\mapsto e^{-2\pi my}
\ena
is exponential decay as $y\rightarrow\infty$, even if we replace $h(y)$ by $y^s$ for any $s$.


\begin{definition}We redefine the Poincare series as
\bna
U_m(z,s):=\sum_{\delta\in\Gamma_\infty\backslash\Gamma}I_s(\delta.z)e(mz)
\ena
Note that for the case $m=0$, it is Eisenstein series.
\end{definition}

\begin{remark}
At least, the series is absolutely convergent for $\re(s)>1$ just like the argument as the Eisenstein series.
Moreover, by Mellin transform, $h(y)=\frac{1}{2\pi i}\int_{(\sigma)}H(s)y^s\frac{dy}{y}$ and thus
\bna
P_m(z,h)=\frac{1}{2\pi i}\int_{(\sigma)} H(s)\sum_{\delta\in\Gamma_\infty\backslash\Gamma}\im(\delta.z)^{s} e(m\delta.z)ds
=\frac{1}{2\pi i}\int_{(\sigma)}H(s) U_m(z,s)ds.
\ena
Following \cite{DeIw1982}, we study $U_m(z,s)$.
\end{remark}
Firstly, note that for $m\geq 1$, $U_m(z,s)\in L^2(\Gamma\backslash\mk h)$.
It is due to the fact that
\bna
I_s(z)e(mz)=y^{s}e^{-2\pi my} e(2\pi i m x)
\ena
which is exponential decay as $y\rightarrow\infty$.


\subsection{Fourier coefficients of Poincare series}
We consider the $n$-th Fourier coefficients of Poincare series,
\bna
a_{m,s}(n,y)=\int_0^1 U_{m}(x+iy,s) e(-nx)dx
=\int_0^ 1 \sum_{\delta\in \Gamma_\infty\backslash\Gamma}I_s(\delta.z)e(m\delta.z)e(-nx)dx
\ena
By the double coset decomposition, we have
\bna
a_{m,s}(n,y)&=&\delta_{m,n}y^{s}e^{-2\pi n y}
+\sum_{c\geq 1}\sideset{}{^*}\sum_{d\bmod c}
\int_{-\infty}^\infty
\left(\frac{y}{(cx+d)^2+c^2y^2}\right)^s e\left(m\frac{az+b}{cz+d}\right)e(-nx)dx\\
&\overset{x+\frac{d}{c}\mapsto x}=&
\delta_{m,n}y^{s}e^{-2\pi n y}
+\sum_{c\geq 1}\sideset{}{^*}\sum_{d\bmod c}
\int_{-\infty}^\infty
\left(\frac{y}{(cx)^2+c^2y^2}\right)^s e\left(m\frac{a(x-\frac{d}{c})+b+iay}{c(x-\frac{d}{c})+d+ icy}\right)e(-n(x-\frac{d}{c}))dx\\
&=&\delta_{m,n}y^{s}e^{-2\pi n y}
+\sum_{c\geq 1}\frac{y^s}{c^{2s}}\sideset{}{^*}\sum_{d\bmod c}e\left(\frac{ma+nd}{c}\right)
\int_{-\infty}^\infty
\left(\frac{1}{x^2+y^2}\right)^s e\left(m\frac{-\frac{ad-bc}{c})}
{cx+ icy}\right)e(-nx)dx\\
&=&\delta_{m,n}y^{s}e^{-2\pi ny}
+y^s\sum_{c\geq 1}\frac{1}{c^{2s}}S(m,n;c)
\int_{-\infty}^\infty
\left(\frac{1}{x^2+y^2}\right)^s e\left(\frac{-m}
{c^2(x+ iy)}-nx\right)dx.
\ena

\subsection{Poincare series spans $L^2(\Gamma\backslash\mk h)$}
Note that
\bna
\overline{U(z,s)}=\sum_{\delta\in\Gamma_\infty\backslash\mk h}\overline{I_s(\delta.z)e(mz)}
=\sum_{\delta\in\Gamma_\infty\backslash\mk h}I_{\overline{s}}(\delta.z)e(-m\overline z).
\ena

Let $\mathfrak S$ be the space spanned by all Poincare series $U_m(z,s)$.
For $f\in L^2(\Gamma\backslash\mk h)$, we have the inner product
\bea
\langle f,U_m(z,\overline s)\rangle
=\int_0^\infty y^{s-1} e^{-2m\pi y}
\left\{\color{blue}\int_0^1 f(x+iy) e(-mx)dx\right\}\frac{dy}{y}.\label{Inner-product-Poincare-autoform}
\eea
Thus if $f$ is orthogonal to the space $\mathfrak S$, then the Mellin transform
of all the Fourier coefficients of $f$ should be zero, and hence $f$ is zero.
This shows that the Poincare sereis spans $L^2(\Gamma\backslash\mk h)$.

\subsection{Inner product of Poincare series 1 - the geometric side}
Note that
\bna
\overline{U_m(z;s)}= U_m(-\overline{z};\overline s).
\ena
and
\bna
\overline{U_m(z,\overline s)}=\sum_{\delta\in\Gamma_\infty\backslash\Gamma}
I_s(\delta.z) e(-mx)e^{-2\pi m y}.
\ena
We consider
\bna
&&\langle
U_n(z,s_1),
U_m(z,\overline s_2)
\rangle \\
&=&\int_0^\infty y^{s_2-1}e^{-2\pi m y}\int_0^1 U_n(z,s_1)e(-mx)dx\frac{dy}{y}\\
&=&
\int_0^\infty y^{s_2-1}e^{-2\pi m y}
\delta_{m,n}y^{s_1}e^{-2\pi ny}\frac{dy}{y}\\
&&
+
\int_0^\infty y^{s_2-1}e^{-2\pi my}
y^{s_1}\sum_{c\geq 1}\frac{1}{c^{2s_1}}
S(n,m;c)
\int_{-\infty}^\infty\left(\frac{1}{x^2+y^2}\right)^{s_1} e\left(-\frac{n}{c^2(x+iy)}-mx\right)dx\frac{dy}{y}\\
&=&\delta_{m,n}\int_{0}^\infty e^{-2\pi(m+n)y}y^{s_1+s_2-1}\frac{dy}{y}
+\sum_{c\geq 1}\frac{1}{c^{2s_1}}S(m,n;c)
I(s_1,s_2,m,n,c)
\ena
where we have
\bna
\int_{0}^\infty e^{-2\pi(m+n)y}y^{s_1+s_2-1}\frac{dy}{y}
=\frac{1}{(2\pi(m+n))^{s_1+s_2-1}}\Gamma(s_1+s_2-1)
\ena
and (Lemma 4.1 in \cite{DeIw1982})
\bna
I(s_1,s_2,m,n,c)
&=&\int_0^\infty  y^{s_1+s_2-1}e^{-2\pi m y}
\int_{-\infty}^\infty \frac{1}{(x^2+y^2)^{s_1}}
e\left(-\frac{n}{c^2(x+iy)}-mx\right)dx\frac{dy}{y}\\
&=&-i2^{3-s_1-s_2}c^{s_1-s_2}
\left(\frac{m}{n}\right)^{\frac{s_1-s_1}{2}}
\int_{-i}^iK_{s_1-s_2}\left(\frac{4\pi\sqrt{ mn}}{c}\theta\right)
\left(\theta+\frac{1}{\theta}\right)^{s_1+s_2-2}\frac{d\theta}{\theta}.
\ena
where the integral is performed though the half circle $|z|=1$, $\re(z)>0$ starting
from the point $-i$.

The above argument gives the analytic continuation of the inner product of Poincare series for $\re(s_1)>\frac{3}{4}$ and $\re(s_2)>\frac{3}{4}$.
On taking $s_1=1+it$ and $s_2=1-it$,
we have the following proposition (Lemma 4.2 in \cite{DeIw1982}).
\begin{prop}\label{prop-tr1-geometr}
We have
\bna
\langle U_{n}(z,1+it),U_{m}(z,1+it)\rangle
=\frac{\delta_{m,n}}{4\pi\sqrt{mn}}
-2i\left(\frac{m}{n}\right)^{it}\sum_{c\geq 1}\frac{S(m,n;c)}{c^2}
\int_{-i}^iK_{2it}\left(\frac{4\pi\sqrt{mn}}{c}\theta\right)\frac{d\theta}{\theta}.
\ena
\end{prop}
\subsection{Inner products of Poincare series 1 - the spectral side}
By Parseval identity, for $f\in\mathcal B$ a basis of $L^2(\Gamma\backslash\mk h)$,
\bna
\langle U_n(z,s_1),U_m(z,\overline{s_2})\rangle
=\int_{f\in\mathcal B}\langle U_n(z,s_1),f\rangle
\overline{\langle U_m(z,\overline s_2),f\rangle} df.
\ena
So we need the inner product of Poincare series with respect to the basis.
\subsubsection{}
Since $m\geq 1$, obviously one has
\bea
\langle 1, U_m(z,\overline s)\rangle =0
\label{Poincare-inner-wt-residue}
\eea

\subsubsection{}
If  $f\in \mathfrak B_{cusp}$ with eigenvalues $\lambda=\frac{1}{4}+\nu_f^2$ and
\bna
f(z)=\sum_{n\neq 0}a_f(n)\sqrt{y} K_{i\nu_f}(2\pi |n|y) e(nx)
\ena
with  $a_f(n)=\lambda_f(|n|)a_f(\mbox{sign}(n))$.
We have
\bea
\nonumber\langle f, U_m(z,\overline s)\rangle
&=&a_f(m)\int_0^\infty y^{s-1}e^{-2\pi my}\sqrt{y}K_{i\nu_f}(2\pi |m| y)\frac{dy}{y}\\
\nonumber&=&a_f(n)\int_0^\infty y^{s-\frac{1}{2}}e^{-2\pi my}K_{i\nu_f}(2\pi |m| y)\frac{dy}{y}\\
&=&\frac{a_f(m)}{(2\pi m)^{s-\frac{1}{2}}}
\pi^{1/2}2^{\frac{1}{2}-s}\frac{\Gamma(s-\frac{1}{2}-i\nu_f)
\Gamma(s-\frac{1}{2}+i\nu_f)}{\Gamma(s)}\nonumber\\
&=&\frac{a_f(m)}{(4\pi m)^{s-\frac{1}{2}}}
\pi^{1/2}\frac{\Gamma(s-\frac{1}{2}-i\nu_f)
\Gamma(s-\frac{1}{2}+i\nu_f)}{\Gamma(s)}.
\label{Poincare-inner-wt-cusp}
\eea
where we have use the following formula  (see Lemma \ref{lemma-K-bessel-eq})
\bea
\int_0^\infty  e^{-y}y^{s-\frac{1}{2}}K_{\nu}(y)\frac{dy}{y}
=\pi^{1/2}2^{\frac{1}{2}-s}\frac{\Gamma(s-\frac{1}{2}-\nu)\Gamma(s-\frac{1}{2}+\nu)}{\Gamma(s)}
\label{K-bessel-integral-eq}.
\eea
Thus the cuspidal spectrum contributes to  $\langle U_n(z,s_1),U_m(z,\overline{s_2})
\rangle $ is
\bna
&&\sum_{f\in\mathfrak B_{cusp}}
\frac{
\overline{\langle f, U_n(z,s_1)\rangle}
\langle f,U_m(z,\overline {s_2})\rangle}
{\langle f,f\rangle}\\
&=&\sum_{f\in\mathfrak B_{cusp}}
\frac{1}{\langle f,f\rangle}
\overline{
\frac{a_f(n)}{(4\pi n)^{\overline s_1-\frac{1}{2}}}\pi^{1/2}
\frac{\Gamma(\overline s_1-1/2-i\nu_f)
\Gamma(\overline s_1-1/2+i\nu_f)
}{\Gamma(\overline s_1)}
}\\
&&\qquad
\frac{a_f(m)}{(4\pi m)^{ s_2-\frac{1}{2}}}\pi^{1/2}
\frac{\Gamma(s_2-1/2-i\nu_f)
\Gamma( s_2-1/2+i\nu_f)
}{\Gamma(s_2)}\\
&=&\frac{\pi n^{-s_1+\frac{1}{2}}m^{-s_2+\frac{1}{2}}
}{(4\pi)^{s_1+s_2-1}}
\frac{1}{\Gamma(s_1)\Gamma(s_2)}
\sum_{f\in\mathfrak B_{cusp}}\frac{a_f(m)\overline{a_f(n)}}{\langle f,f\rangle}
\prod_{i=1,2}\prod_{\pm}\Gamma(s_i-\frac{1}{2}\pm i\nu_f)
\ena

\subsubsection{}
Recall
\bna
E(z,\frac{1}{2}+i\nu)&=&y^{\frac{1}{2}+i\nu}
+\frac{\xi(2i\nu)}{\xi(1+2i\nu)}y^{\frac{1}{2}-i\nu}\\
&&
+2\frac{1}{\xi(1+2i\nu)}\sum_{n\neq 0}|n|^{i\nu}\sigma_{-2i\nu}(|n|)\sqrt{y} K_{i\nu}(2\pi |n|y) e(nx).
\ena
Thus
\bna
\langle E(z,\frac{1}{2}+i\nu),U_m(z,\overline s)\rangle
&=&\int_0^\infty y^{s-1}e^{-2\pi my}
2\frac{1}{\xi(1+2i\nu)} m^{i\nu}\sigma_{-2i\nu}(m)\sqrt{y} K_{i\nu}(2\pi my)\frac{dy}{y}\\
&=&2\frac{1}{\xi(1+2i\nu)} m^{i\nu}\sigma_{-2i\nu}(m)\int_0^\infty y^{s-\frac{1}{2}}e^{-2\pi my}K_{i\nu}(2\pi my)\frac{dy}{y}.
\ena
By \eqref{K-bessel-integral-eq} again, we have
\bea
\nonumber\langle E(z,\frac{1}{2}+i\nu),U_m(z,\overline s)\rangle
&=&2\frac{1}{\xi(1+2i\nu)} m^{i\nu}\sigma_{-2i\nu}(m)\frac{1}{(2\pi m)^{s-\frac{1}{2}}}
\pi^{1/2}2^{\frac{1}{2}-s}\frac{\Gamma(s-\frac{1}{2}-i\nu)\Gamma(s-\frac{1}{2}+i\nu)}
{\Gamma(s)}\\
&=&2\frac{m^{i\nu}\sigma_{-2i\nu}(m)}{\xi(1+2i\nu)} \frac{1}{(4\pi m)^{s-\frac{1}{2}}}
\pi^{1/2}\frac{\Gamma(s-\frac{1}{2}-i\nu)\Gamma(s-\frac{1}{2}+i\nu)}{\Gamma(s)}
\label{Poincare-inner-wt-Eisen}
\eea
and thus this part contributes
\bna
&&\frac{1}{4\pi}
\int_{-\infty}^\infty
\langle U_n(z,s_1),E(z,1/2+i\nu)\rangle
\overline{\langle U_m(z,\overline s_2),E(z,1/2+i\nu)\rangle }d\nu\\
&=&\frac{1}{4\pi}
\int_{-\infty}^\infty
\frac{\pi}{(4\pi)^{s_1+s_2-1}n^{s_1-\frac{1}{2}}m^{s_2-\frac{1}{2}}}
\frac{1}{\Gamma(s_1)\Gamma(s_2)}
\left\{\overline{\frac{2n^{i\nu}\sigma_{-2i\nu}(n)}{\xi(1+2i\nu)}}\frac{2m^{i\nu}\sigma_{-2i\nu}(m)}{\xi(1+2i\nu)}\right\}
\prod_{i=1,2}\prod_{\pm}\Gamma(s_i-\frac{1}{2}\pm i\nu)
d\nu\\
&=&
\frac{\pi n^{-s_1+\frac{1}{2}}m^{-s_2+\frac{1}{2}}}{(4\pi)^{s_1+s_2-1}}
\frac{1}{\Gamma(s_1)\Gamma(s_2)}
\frac{1}{\pi}
\int_{-\infty}^\infty
\left(\frac{m}{n}\right)^{i\nu}\frac{\sigma_{2i\nu}(n)}{\xi(1-2i\nu)}
\frac{\sigma_{-2i\nu}(m)}{\xi(1+2i\nu)}
\prod_{i=1,2}\prod_{\pm}\Gamma(s_i-\frac{1}{2}\pm i\nu)
d\nu.
\ena
Note that
\bna
\xi(1-2i\nu)
\xi(1+2i\nu)=\pi^{-1}\Gamma(\frac{1}{2}-i\nu)\Gamma(\frac{1}{2}+i\nu)
\zeta(1-2i\nu)\zeta(1+2i\nu),
\ena
and thus
\bna
&&\frac{1}{4\pi}
\int_{-\infty}^\infty
\langle U_n(z,s_1),E(z,1/2+i\nu)\rangle
\overline{\langle U_m(z,\overline s_2),E(z,1/2+i\nu)\rangle }d\nu\\
&=&\frac{\pi n^{-s_1+\frac{1}{2}}m^{-s_2+\frac{1}{2}}}{(4\pi)^{s_1+s_2-1}}
\frac{1}{\Gamma(s_1)\Gamma(s_2)}
\int_{-\infty}^\infty
\left(\frac{m}{n}\right)^{i\nu}\frac{\sigma_{2i\nu}(n)}{\zeta(1-2i\nu)}
\frac{\sigma_{-2i\nu}(m)}{\zeta(1+2i\nu)}
\frac{\prod_{i=1,2}\prod_{\pm}\Gamma(s_i-\frac{1}{2}\pm i\nu)}{\Gamma(\frac{1}{2}+i\nu)\Gamma(\frac{1}{2}-i\nu)}
d\nu.
\ena
Therefore, we have the following result.

\begin{lemma}We have
\bna
\langle U_n(z,s_1),U_m(z,\overline s_2)\rangle
&=&\frac{\pi n^{-s_1+\frac{1}{2}}m^{-s_2+\frac{1}{2}}
}{(4\pi)^{s_1+s_2-1}}
\frac{1}{\Gamma(s_1)\Gamma(s_2)}\\
&&\left\{
\sum_{f\in\mathfrak B_{cusp}}\frac{a_f(m)\overline{a_f(n)}}{\langle f,f\rangle}
\prod_{i=1,2}\prod_{\pm}\Gamma(s_i-\frac{1}{2}\pm i\nu_f)\right.\\
&&\quad + \left.
\int_{-\infty}^\infty
\left(\frac{m}{n}\right)^{i\nu}\frac{\sigma_{2i\nu}(n)}{\zeta(1-2i\nu)}
\frac{\sigma_{-2i\nu}(m)}{\zeta(1+2i\nu)}
\frac{\prod_{i=1,2}\prod_{\pm}\Gamma(s_i-\frac{1}{2}\pm i\nu)}{\Gamma(\frac{1}{2}+i\nu)\Gamma(\frac{1}{2}-i\nu)}
d\nu.
\right\}
\ena
\end{lemma}

On taking $s_1=1+it$ and  $s_2=1-it$,
by the facts
\bna
|\Gamma(\frac{1}{2}+ib)|^2=\frac{\pi}{\cosh(\pi b)},\quad
|\Gamma(ib)|^2=\frac{\pi}{b\sinh(\pi b)},
\quad \Gamma(s+1)=s\Gamma(s),
\ena
we have
\bna
\Gamma(1/2+i\nu)\Gamma(1/2-i\nu)&=&
\frac{\pi}{\sin\pi(1/2-i\nu)}=\frac{\pi}{\cos(i\nu\pi)}
=\frac{\pi}{\cosh(\pi\nu)}\\
\prod_{\pm}\prod_{\pm}
\Gamma(1/2\pm i\nu\pm it)\Gamma(1/2\pm i\nu\pm it)
&=&\frac{\pi^2}{\cosh(\pi(t+\nu))
\cosh(\pi(t-\nu))}\\
\Gamma(1+2it)\Gamma(1-2it)
&=&2it\Gamma(2it) (-2it)\Gamma(-2it)=4t^2\frac{\pi}{2t\sinh(2\pi t)}.
\ena
we have the following results on the spectral side.
\begin{prop}[Lemma 4.5 in \cite{DeIw1982}]\label{prop-tr1-spectral}
We have
\bna
\langle U_n(z,1+it),U_m(z,1+it)\rangle
&=&
\frac{1}{4\sqrt{mn}}\left(\frac{m}{n}\right)^{it}
\frac{\sinh(\pi t)}{t}
\left\{\pi\sum_{f\in\mathfrak B_{cusp}}\frac{|a_f(1)|^2}{\langle f,f\rangle}
\frac{\lambda_{f}(m)\lambda_f(n)}{\cosh(\pi(\nu_f-t))\cosh(\pi(\nu_f+t))}\right.\\
&&\qquad
\left.+\int_{-\infty}^\infty
\left(\frac{m}{n}\right)^{i\nu}\frac{\sigma_{-2i\nu}(m)\sigma_{2i\nu}(n)}{|\zeta(1+2i\nu)|^2}
\frac{\cosh\pi\nu}{\cosh\pi(\nu-t)\cosh\pi(\nu+t)}d\nu
\right\}
\ena
\end{prop}

\subsection{Inner products of Poincare series 1 - the pre trace formula}
By propositions \ref{prop-tr1-spectral} and \ref{prop-tr1-geometr},
we take
\bna
H(\nu,t):&=&\frac{\cosh (\pi\nu)}{\cosh(\pi(\nu-t))
\cosh(\pi(\nu+t))}\\
D_{2it}(x)&=&-\frac{2it}{\sinh(\pi t)}
\int_{-i}^i K_{2it}(x\theta)\frac{d\theta}{\theta}
=\frac{t}{\sinh(2\pi t)}\int_{x}^\infty (J_{2it}(\theta)+J_{-2it}(\theta))\frac{d\theta}{\theta},
\ena
we have the following result.
\begin{prop}\label{prop-tr1}
We have
\bna
&&
\frac{t}{\pi\sinh(\pi t)}
\delta_{m,n}
-\sum_{c\geq 1}\frac{4\pi\sqrt{mn}}{c^2}S(m,n;c)
D_{2it}\left(\frac{4\pi\sqrt{mn}}{c}\right)
\\
&=&\pi\sum_{f\in\mathfrak B_{cusp}}\frac{|a_f(1)|^2}{\langle f,f\rangle}
\frac{H(\nu_f,t)}{\cosh(\pi\nu_f)}\lambda_{f}(m)\lambda_f(n)
+\int_{-\infty}^\infty
\left(\frac{m}{n}\right)^{i\nu}
\frac{\sigma_{-2i\nu}(m)\sigma_{2i\nu}(n)}{|\zeta(1+2i\nu)|^2}
H(\nu,t)
d\nu
\ena
\end{prop}
\subsection{Inner products of Poincare series 2}
If one consider
\bna
\langle U_n(z,1+it), \overline{U_m(z,1-it)}\rangle
\ena
(see lemma 4.4 in \cite{DeIw1982} for the calculation of the geometric side), one has another pre-trace formula as follows.
\begin{prop}[Lemma 4.8 in \cite{DeIw1982}]We have
\bna
&&\pi\sum_{f\in\mathfrak B_{cusp}}
\frac{a_f(m)a_f(n)}{\langle f,f\rangle}\frac{H(\nu_f,t)}{\cosh(\pi \nu_f)}
+\int_{-\infty}^\infty(mn)^{i\nu}
\frac{\sigma_{-2i\nu}(m)\sigma_{-2i\nu}(n)}{\xi(1+2i\nu)^2}
H(\nu,t)d\nu\\
&=&\sum_{c\geq 1}\frac{4\pi\sqrt{mn}}{c^2}S(m,n;c)K_{2it}\left(\frac{4\pi\sqrt{mn}}{c}\right).
\ena
\end{prop}
\begin{remark}In this formula, since $m,n\geq 1$, $\int_0^1e((m+n)x)dx=0$, i.e. the diagonal term
vanishes.
\end{remark}

\subsection{Test function and integral transform}We refer to this part
in page 228 in \cite{DeIw1982}.

Let $\varphi\in C_c^3(0,\infty)$.
For real $t$, set
\bna
g(t)&=&-\int_0^\infty(J_{2it}(x)+J_{-2it}(x))\left(\frac{\varphi(x)}{x}\right)'dx\\
\hat\varphi(r)&=&\frac{\pi}{\sinh (\pi r)}\int_{0}^\infty\frac{J_{2ir}(x)-J_{-2ir}(x)}{2i}\varphi(x)\frac{dx}{x}\\
\check{\varphi}(r)&=&\frac{4}{\pi}\cosh(\pi r)\int_0^\infty K_{2ir}(x)\varphi(x)\frac{dx}{x}\\
\tilde\varphi(\ell)&=&\int_0^\infty J_\ell(y)\varphi(y)\frac{dy}{y}\\
\varphi_B(x)&=&\sum_{k>0\atop k\mbox{\,odd}}2k\tilde\varphi(k)J_k(x)\\
\varphi_H&=&\varphi-\varphi_B
\ena
Applying the following integral transform
\bna
-\int_{-\infty}^\infty t\sinh(2\pi t)H(r,t)K_{2it}(\theta)dt
&=&\theta K_{2ir}(\theta)\\
\int_{-\infty}^\infty \frac{tg(t)}{\sinh(\pi t)}dt&=&\int_0^\infty J_0(x)\varphi(x)dx\\
x\int_x^\infty \int_{-\infty}^\infty \frac{tg(t)}{\sinh(\pi t)}(J_{2it}(u)+J_{-2it}(u))dt\frac{du}{u}&=&\varphi_H(x)\\
\int_{-\infty}^\infty H(r,t)g(t)dt&=&\frac{2}{\pi}\hat\varphi(t).
\ena
One has the following
\begin{thm}[Prop 2 in \cite{DeIw1982}]
For $\varphi\in C_c^3(0,\infty)$, we have
\bna
&&\sum_{f\in\mathcal B_{cusp}}\frac{|a_f(1)|^2}{\langle f,f\rangle}
\frac{\lambda_f(m)\lambda_f(n)}{\cosh(\pi\nu_f)}\widehat\varphi(\nu_f)
+\int_{-\infty}^\infty \left(\frac{m}{n}\right)^{-i\nu}\frac{
\sigma_{-2i\nu}(n)\sigma_{2i\nu}(m)
}{|\zeta(1+2i\nu)|^2}\hat\varphi(\nu)d\nu\\
&=&
\frac{\delta_{m,n}}{2\pi}\int_0^\infty J_0(x)\varphi(x)dx
+\sum_{c\geq 1}\frac{S(m,n;c)}{c}\varphi_H\left(\frac{4\pi\sqrt{mn}}{c}\right).
\ena
\end{thm}

\subsection{Kunzetsov's trace formula - Final version}
This is a formula in Li Xiaoqing's paper.
\begin{prop}
Assume $u_j\in\mathfrak B_{cusp}$ with $\langle u_j,u_j\rangle=1$ with $\lambda_j=\frac{1}{4}+ t_j^2$ the following
\bna
u_j(z)=\sum_{n\neq 0}\rho_j(n)\sqrt{y}K_{it_j}(2\pi |n|y)e(nx).
\ena
Let $h(z)$ be a test function satisfying the following.
\bit
\item $h(z)$ is holomorphic in $|\im(z)|\leq\sigma$
\item $h(z)\ll (1+|z|)^{-\theta}$ in the strip with $\sigma>1/2$ and $\theta>2$
\eit
One has
\bna
&&\sum_{j\geq 1}h(t_j)\frac{\rho_j(n)\overline{\rho_j(m)}}{\cosh(\pi t_j)}
+\frac{1}{\pi}\int_{-\infty}^\infty\left(\frac{m}{n}\right)^{ir}
\sigma_{2ir}(n)\sigma_{-2ir}(m)\frac{h(r)}{|\zeta(1+2ir)|^2}dr\\
&=&\frac{\delta_{m,n}}{\pi^2}\int_{-\infty}^\infty
r\tanh(\pi r)h(r)dr + \sum_{c\geq 1}\frac{S(n,m;c)}{c}h^+\left(\frac{4\pi\sqrt{mn}}{c}\right),
\ena
with
\bna
\sigma_\nu(n)=\sum_{d\mid n}d^{\nu},\qquad
h^+(x)=\frac{2i}{\pi}\int_{-\infty}^\infty J_{2ir}(x)\frac{h(r)r}{\cosh \pi r}dr.
\ena
\end{prop}
\begin{remark}
Note that the L.H.S. in the cuspidal parameter should be
\bna
\sum_{j\geq 1}\frac{|\rho_{j}(1)|^2}{\langle u_j,u_j\rangle}
\lambda_j(m)\lambda_j(n)\frac{\phi(t_j)}{\cosh(\pi t_j)}.
\ena
\end{remark}

\begin{remark}
For the proof of the Kuznetsov's trace formula via the relative trace formula,
we refer to V. Blomer,
{\it The relative trace formula in analytic number theory},
arXiv:1912.08137.

\end{remark}

\section{Truncated Eisenstein series and Maass-Selberg Relation}
Now, Let $T$ be a sufficiently large parameter, we define the truncated spectral function
\bna
I_s^T(z)=\Lambda^TI_s(z)=\left\{
\begin{aligned}
&y^s,\quad&& y=\im z\leq T\\
&0,\quad&& y>T
\end{aligned}
\right.
\ena
Then we set
\bna
\Lambda^TE(z,s)=
E^T(z,s):=\sum_{\delta\in\Gamma_\infty\backslash\Gamma}I_s^T(\delta.z)
\ena
which is called Truncated Eisenstein series.





\newpage

\appendix
\section{Bessel equations}\label{appendix-Bessel-equ}
This part is a note on Chapter 7 in {\it
(Te Shu Han Shu Gai Lun, Chinese verion, Wang Zhuxi and Guo dunren)}.


We change variables in \eqref{Bessel-real} as follows.
Let $F(z)=G(iz)$.
One has
\bna
F(z)=G(i z),\quad   F'(z)=iG'(i z), \quad F''(z)=-G''(i z)
\ena and thus \eqref{Bessel-real} is
\bna
-G''(i z)+\frac{1}{z}iG'(i\zeta)-\left(1+\frac{(it)^2}{z^2}\right)G(iz)=0
\ena
on taking $iz=\zeta$, one has
\bea
G''(\zeta)+\frac{1}{\zeta}G'(\zeta)+\left(1-\frac{(it)^2}{\zeta^2}\right)G(\zeta)=0
\label{Bessel-original}
\eea
Here $\nu=it$ is called the order of the Bessel function.

\subsection{Basic solutions}
\subsubsection{}

Assume $2\nu\notin\Z$. We rewrite \eqref{Bessel-original} as
\bna
\zeta^2 G''+\zeta G'+(\zeta^2-\nu^2)G=0.
\ena
So we assume that $G=\sum_{k\geq 0} c_k z^{k+\rho}$ with $c_0\neq 0$ to obtain
the solutions
\bna
J_{\pm\nu}(\zeta)=\sum_{k\geq 0}\frac{(-1)^k}{k!}\frac{1}{\Gamma(\pm\nu+k+1)}
\left(\frac{\zeta}{2}\right)^{2k\pm\nu}
\ena
which are linear independent since $2\nu\notin\Z$.
They are  \underline{Bessel function of the  first kind}.
\subsubsection{}If $\nu=n\in \mathbb N\cup\{0\}$,
we know that $\Gamma(-n+k+1)\rightarrow\infty$ ($k<n$).
Thus
\bna
J_{-n}(\zeta)&=&\sum_{k\geq n}
\frac{(-1)^k}{k!}\frac{1}{\Gamma(-n+k+1)}
\left(\frac{\zeta}{2}\right)^{2k-n}
=\sum_{k\geq 0}
\frac{(-1)^{k+n}}{(k+n)!}\frac{1}{\Gamma(k+1)}
\left(\frac{\zeta}{2}\right)^{2k+n}\\
&=&(-1)^nJ_{n}(\zeta),
\ena
which are linear dependent. In this case, \underline{$J_n(\zeta)$ is entire function}.
So we need another linear independent solution, which is  $Y_n(\zeta)$ with
\bna
Y_\nu(\zeta):=\frac{\cos(\pi\nu) J_\nu(\zeta)-J_{-\nu}(
\zeta)}{\sin (\pi\nu)}
\ena
which are \underline{Bessel function of the second kind}.

\subsubsection{}If $2\nu\in\Z-2\Z$, i.e. $\nu=n+\frac{1}{2}$ with $n\in \Z$,
the $J_{n+\frac{1}{2}}(\zeta)$ can be expressed via elementary functions as follows.
\bna
J_{1/2}(\zeta)&=&\sum_{k\geq 0}\frac{(-1)^k}{k!}\frac{1}{\Gamma(k+\frac{3}{2})}
\left(\frac{\zeta}{2}\right)^{2k+\frac{1}{2}}
\ena
Note that
\bna
\Gamma(k+\frac{3}{2})=\Gamma(2k+2)2^{-2k-1}\sqrt{\pi}/\Gamma(k+1),
\ena
one has
\bna
J_{1/2}(\zeta)=\sqrt{\frac{2}{\pi \zeta}}\sum_{k\geq 0}\frac{(-1)^k}{(2k+1)!}\zeta^{2k+1}
=\sqrt{\frac{2}{\pi\zeta}}\sin\zeta.
\ena
and similarly,
\bna
J_{-1/2}(\zeta)
=\sqrt{\frac{2}{\pi\zeta}}\cos\zeta.
\ena
\begin{remark}
Another basis solutions are expressed as
\bna
H^1_\nu(z):=J_\nu(z)+iY_\nu(z),\qquad H^2_\nu(z):=J_{\nu}(z)-iY_{\nu}(z)
\ena
which are called Bessel function of the third kind.
\end{remark}
\subsection{Solutions of \eqref{Bessel-real}}
Consider \eqref{Bessel-real}.
We know $\nu=it\notin\Z$ and  $F(z)=G(iz)=G(\zeta)$.
Solutions for \eqref{Bessel-real} are
\bna
J_{it}(iz),\quad J_{-it}(iz)
\ena
which are linear independent.

We introduce new functions defined by
\bna
I_\nu(z)&=&
\left\{
\begin{aligned}
e^{-\nu\frac{\pi i}{2}}J_\nu(ze^{\frac{\pi i}{2}}),\quad (-\pi<\arg z<\frac{\pi}{2})\\
e^{\nu\frac{3\pi i}{2}}J_\nu(z^{-\frac{3\pi i}{2}}),\quad (\frac{\pi}{2}<\arg z<\pi)
\end{aligned}
\right.\\
&=&\left(\frac{z}{2}\right)^{\nu}
\sum_{k=0}^\infty\frac{1}{k!}\frac{1}{\Gamma(\nu+k+1)}\left(\frac{z}{2}\right)^{2k}
\ena
Then we know $I_{it}(z)$ are one of the solution of \eqref{Bessel-real}.
\bit
\item
If $\nu\notin\Z$, $I_{\pm\nu}(z)$ are two linear independent solutions.
One define
\bna
K_{\nu}(z)=\frac{\pi}{2\sin(\pi\nu)}(I_{-\nu}(z)-I_{\nu}(z))
\ena
and use $K_\nu(z)$ and $I_\nu(z)$ as linear independent solutions of \eqref{Bessel-real}.
\item
If $\nu=n\in\Z$, $I_{-n}(z)=I_n(z)$.
In this case, $K_\nu(z)$ and $I_\nu(z)$ are still linear independent.

\eit
Therefore, we have the following result.
\begin{prop}Two linear independent solutions of \eqref{Bessel-real}
are
\bna
K_{it}(z)\quad\mbox{and}\quad I_{it}(z).
\ena
\end{prop}

\subsection{Important property of $J$-Bessel function}
We have
\bna
J_{\pm \nu}(z):=\sum_{k\geq 0}\frac{(-1)^k}{k!}\frac{1}{\Gamma(\pm\nu+k+1)}\left(\frac{z}{2}\right)^{2k\pm \nu},
\qquad \re(\nu)\geq 0
\ena
Special values are given as
\bna
J_{-n}(z)=(-1)^nJ_n(z),\quad
J_{1/2}(z)=\sqrt{\frac{2}{\pi z}}\sin z,
\quad J_{-1/2}(z)=\sqrt{\frac{2}{\pi z}}\cos z
\ena
\underline{and the  fact that $J_n(z)$ is entire function in $z$}.

The recurrent differential formula are
\bna
&\frac{d}{dz}(z^\nu J_\nu)=z^{\nu} J_{\nu-1},\quad
\frac{d}{dz}(z^{-\nu} J_\nu)=-z^{-\nu} J_{\nu+1} &\\
&\left(\frac{d}{zdz}\right)^m(z^\nu J_\nu)=z^{\nu-m} J_{\nu-m},\quad
\left(\frac{d}{zdz}\right)^m(z^{-\nu} J_\nu)=(-1)^mz^{-\nu-m} J_{\nu+m} &
\ena
and some recurrent formulas  are
\bna
&J_{\nu-1}+J_{\nu+1}=\frac{2\nu}{z}J_\nu,
\quad J_{\nu-1}-J_{\nu+1}=2J_v'&\\
&J_{\pm\nu}(ze^{\pi i})=e^{\pm \nu\pi i}J_{\pm \nu}(z)&
\ena

Integral representations of $J$-Bessel function are
\bna
J_\nu(z)&=&\frac{(z/2)^{\nu}}{\sqrt{\pi}\Gamma(\nu+\frac{1}{2})}
\int_{-1}^1 e^{izt}(1-t^2)^{\nu-\frac{1}{2}}dt,\quad \re(\nu)>-1/2,\arg(1-t^2)=0\\
&=&\frac{(\frac{z}{2})^\nu}{\sqrt{\pi}\Gamma(\nu+\frac{1}{2})}
\int_0^\pi e^{iz\cos\theta}\sin^{2\nu}\theta d\theta,\quad\mbox{by taking $t=\cos\theta$ as above}\\
&=&
\frac{(\frac{z}{2})^\nu}{\sqrt{\pi}\Gamma(\nu+\frac{1}{2})}
\int_0^\pi\cos(z\cos\theta)\sin^{2\nu}\theta d\theta\\
&=&\frac{(z/2)^\nu}{2\pi i}\int_{-\infty}^{0+}e^{t-\frac{z^2}{4t}}t^{-\nu-1}dt,\quad |\arg t|<\pi\\
&=&\frac{1}{2\pi i}\int_{-\infty}^{0+}
e^{\frac{z}{2}(t-\frac{1}{t})}t^{-\nu-1}dt,\quad |\arg z|<\frac{\pi}{2},|\arg t|<\pi
\ena
We also has the following important integral representation for
the K-Bessel function, namely
\bea
\int_0^\infty e^{-\frac{y}{2}(t+\frac{1}{t})}t^s\frac{dt}{t}=2 K_s(y).
\label{integral-transform-K-Bessel-in-appendix}
\eea
which implies
\bna
K_{s}(y)=K_{-s}(y)
\ena by $t\mapsto t^{-1}$ and $s\mapsto -s$.


\subsection{Asymptotic formulas as $|z|\rightarrow\infty$}
Let
\bna
(\nu,0)&=&1\\
(\nu,p)&=&\frac{\Gamma(\frac{1}{2}+\nu+p)}{p!\Gamma(\frac{1}{2}+\nu-p)}=(-\nu,p).
\ena
Some asymptotic formulas as $|z|\rightarrow\infty$ are
\bna
J_\nu(z)&\sim& \sqrt{\frac{2}{\pi z}}
\left[\cos(z-\frac{\pi}{2}\nu-\frac{\pi}{4})\sum_{m\geq 0}(-1)^m\frac{(\nu,2m)}{(2z)^{2m}}
\right.\\
&&\qquad\left.
-\sin(z-\frac{\pi}{2}\nu-\frac{\pi}{4})
\sum_{m\geq 0}\frac{(-1)^m(\nu,2m+1)}{(2z)^{2m+1}}
\right],\quad -\pi<\arg z<\pi
\ena
Other range of $\arg z$, for example, $0<\arg z<2\pi$ can be obtained by the  recurrent relation
\bna
J_\nu(z)=e^{\nu\pi i}J_{\nu}(ze^{-\pi i})
\sim e^{(\nu+\frac{1}{2})\pi i}
&\sim& \sqrt{\frac{2}{\pi z}}
\left[\cos(z+\frac{\pi}{2}\nu+\frac{\pi}{4})\sum_{m\geq 0}(-1)^m\frac{(\nu,2m)}{(2z)^{2m}}
\right.\\
&&\qquad\left.
-\sin(z+\frac{\pi}{2}\nu+\frac{\pi}{4})
\sum_{m\geq 0}\frac{(-1)^m(\nu,2m+1)}{(2z)^{2m+1}}
\right]
\ena

One has also
\bna
K_\nu(z)&\sim& \sqrt{\frac{\pi}{2z}} e^{-z}\left[1+\sum_{n\geq 1}\frac{(\nu,n)}{(2z)^n}\right],
\quad |\arg z|<\frac{3}{2}\pi\\
I_\nu(z)&\sim&\frac{e^z}{\sqrt{2\pi z}}
\sum_{n\geq 0}\frac{(-1)^n (\nu,n)}{ (2z)^n}
+\frac{e^{-z+(\nu+\frac{1}{2})\pi i}}{\sqrt{2\pi z}}\sum_{n\geq 0}\frac{(\nu,n)}{(2z)^n},
\quad -\frac{\pi}{2}<\arg z<\frac{3}{2}\pi.
\ena
\subsection{Asymptotic formulas as $|\nu|\rightarrow\infty$ }
For fixed $z$, as $|\nu|$ large,
\bna
J_\nu(z)\sim \exp\left(\nu+\nu\log\frac{z}{2}-(\nu-\frac{1}{2})\log \nu\right)
\left[c_0+\frac{c_1}{\nu}+\frac{c_2}{\nu^2}+\cdots\right]
\ena
with $c_0=\frac{1}{\sqrt{2\pi}}$.

\subsection{Asymptotic formulas as both $|\nu|$ and $|z|$ large}
We refer to sections 7.11-7.12 in the Book on special functions by Wang Zhuxi-Guo Dunren.
%\begin{CJK}{UTF8}{gkai}
%ÍõÖñϪ-¹ù¶ØÈÊ, {\it ÌØÊ⺯Êý¸ÅÂÛ}, ±±¾©´óѧ³ö°æÉç, 2012
%\end{CJK}



\section{Spherical Whittaker function in Whittaker model}\label{appendix-spherical-whittaker-model}
Some notations are as follows.
\bit
\item For compact subgroup,
\bna
O(2)=SO(2)\bigcup\bma -1\\&1\ema SO(2),\quad
PSO(2)= SO(2)/\{\pm I\}.
\ena
Note that $GL_2(\R)=GL_2^+(\R)\bigcup GL_2^{-}(\R)$. We have
\bna
\mk h=SL_2(\R)/SO(2)=GL_2(\R)/Z(\R)O(2)
\ena
\item  Next, $PGL_2(\R)=GL_2(\R)/Z(\R)$ and
\bna
 PGL_2(\R)/PSO(2)=GL_2(\R)/Z(\R)SO(2).
\ena
\eit
We are dealing with spherical representations for $PGL_2=GL_2/Z$ over $\R$.
Note that for $g\in GL_2(\R)$,
\bna
g=z\bma 1&x\\&1\ema
\bma\pm 1\\&1\ema
\bma y^{1/2}\\&y^{-1/2}\ema \kappa_\theta,
\quad \kappa_\theta\in SO(2), z\in Z(\R).
\ena
Let
$\pi_\infty$ be an irreducible unramified unitary
infinite dimensional representation of $G_\infty$
with trivial central character,
which can be realized as the normalized induced representation
\bna
\pi(\epsilon_\pi,it_\pi)=\mathrm{Ind}_{B(\R)}^{G(\R)}\chi_{\epsilon_\pi,it_\pi},
\ena
where $B$ is the standard parabolic subgroup of $G$ and
$\chi_{\epsilon_\pi,it_\pi}$ is a character of $B(\R)$ given by
\bna
\chi_{\epsilon_\pi,it_\pi} \bma a&b\\0&d\ema=\mathrm{sgn}(a)^{\epsilon_{\pi}}\mathrm{sgn}(d)^{\epsilon_{\pi}}
\left|\frac{a}{d}\right|^{it_\pi},\quad \bma a&b\\0&d\ema\in B(\R).
\ena
Here $\epsilon_\pi\in\{0,1\}$ and $\{\pm t_\pi\}$ is the  set of spectral parameters of $\pi$ such that
\bit
\item either $t_\pi\in \R$, in which case $\pi_\infty=\pi(\epsilon_{\pi},it_{\pi})$ is a principal series,
\item or $t_\pi\in i\R$ with $0<|t_\pi|<\frac{1}{2}$,
 in which case $\pi_\infty=\pi(\epsilon_{\pi},it_{\pi})$
is a complementary series.
\eit
\subsection{spherical Whittaker function}
The spherical vector in $\pi_\infty=\pi(\epsilon_\pi,it_\pi)$ is the function
\bna
f_0:GL_2(\R)\rightarrow\C
\ena
such that
\bna
f_0\left(z\bma 1&x\\&1\ema
\bma\pm 1\\&1\ema\bma
y\\&1\ema\kappa_\theta\right)
=(\pm 1)^{\epsilon_\pi}y^{\frac{1}{2}+it_\pi}.
\ena
Via the theory of intertwining operator and Whittaker function,
 for given a non-degenerate character on $N(\R)$ (i.e. $a\neq 0$),
\bna
\psi_a:N(\R)\rightarrow\C,\quad \psi_a(n(x))=e^{2\pi iax}=e(ax),
\ena
the associated spherical Whittaker function (in the Whittaker model) is defined by
\bna
W_0(g,\psi_a):&=&\int_{N_P(\R)\cap w_sN_P(\R)w_s^{-1}\backslash N_P(\R)}
f_0(w_s^{-1}n g)\overline{\psi_a(n)}dn\\
&=&\int_{-\infty}^{\infty}f_0\left(\bma&-1\\1\ema \bma1&x\\&1\ema  g
\right)e(-ax)dx
\ena
Obviously one has
\bna
&&W_0\left(z\bma 1&x_0\\&1\ema\bma \pm y_0\\&1\ema
g\kappa_\theta,\psi_{a}\right)\\
&=&e(ax_0)
\int_{-\infty}^\infty f_0
\left(\bma&-1\\1\ema \bma1&x\\&1\ema \bma \pm y_0\\&1\ema
g\right)e(-ax)dx\\
&=&e(ax_0)
\int_{-\infty}^\infty f_0
\left(\bma \pm y_0\\&\pm y_0\ema
\bma \pm y_0^{-1}\\&1\ema
\bma&-1\\1\ema \bma1&\pm y_0^{-1}x\\&1\ema
g\right)e(-ax)dx\\
&=&(\pm 1)^{\epsilon_\pi}y_0^{\frac{1}{2}-it}
e(ax_0)
\int_{-\infty}^\infty f_0
\left(
\bma&-1\\1\ema \bma1&x\\&1\ema
g\right)e(-(\pm y_0a)x)dx\\
&=&(\pm 1)^{\epsilon_\pi}y_0^{\frac{1}{2}-it}
e(ax_0)W_0(g,\psi_{\pm y_0 a}).
\ena
Based on the argument above, by the Iwasawa decomposition,
we need only to determine the values
\bna
W_0\left(\bma y&\\&1\ema,\psi_1\right)&=&\int_{-\infty}^\infty
f_0\left(\bma &-1\\ 1\ema\bma 1&x\\&1\ema\bma y\\&1\ema\right)e(-x)dx\\
&=&
\int_{-\infty}^\infty
f_0\bma &-1\\ y&x\ema e(-x)dx
=\int_{-\infty}^\infty\left(\frac{y}{x^2+y^2}\right)^{\frac{1}{2}+it}e(-x)dx
\ena
where we have used the following explicit Iwasawa decomposition.
The expression above coincides with \eqref{tilde-whittaker}
 and the calculation are in Lemma \ref{lemma-integral-Whitaker-Bessel}.

\begin{lemma}[Explicit Iwasawa decomposition]
For $v$ real, then $\forall g=\bma a&b\\c&d\ema\in GL_2(\R)^+$, it has unique decomposition
\bea
g=\bma a&b\\c&d\ema=\bma z&\\&z\ema\bma 1&x\\&1\ema\bma y^{1/2}&\\&y^{-1/2}\ema\kappa_\theta
\label{Iwassawa-local-Arch}
\eea
with
\bna
z=z(g)=\sqrt{ad-bc},\quad x=x(g)=\frac{ac+bd}{c^2+d^2},\quad y=y(g)=\frac{ad-bc}{c^2+d^2}
\ena
and
\bna
\theta=\theta(g)=\arctan(-c/d),\quad\mbox{of period $\pi$}.
\ena
\end{lemma}

\section{Mellin transform - Harmonic analysis on $L^2(\R^+,\frac{dy}{y})$}
\subsection{Mellin transform}
\begin{prop}
For $f\in C_c^\infty(\R^+,\frac{dy}{y})$, the Mellin transform is defined by
\bna
(\mathcal M(f)(s))=\varphi(s)=\int_0^\infty f(t) t^{s}\frac{dt}{t}
\ena
and the Mellin inversion formula is
\bna
f(y)=(\mathcal M^{-1}\varphi)(y)=\frac{1}{2\pi i}\int_{c-i\infty}^{c+i\infty}
\varphi(s)y^{-s}ds.
\ena
\end{prop}
\begin{remark}
For $f\in L^2(\R^+,\frac{dy}{y})$, at the discontinuous point,
\bna
\frac{1}{2}\left(f(y+)+f(y-)\right)=\frac{1}{2\pi i}\lim_{r\rightarrow\infty}
\int_{c-ir}^{c+ir}(\mathcal Mf)(s)y^{-s}ds.
\ena
\end{remark}

\subsection{Harmonic analysis}
Now we build the harmonic analysis as follows.
For $L^2(\R^+,\frac{dy}{y})$,
the Laplacian operator on $\R^+$ is $\Delta=-y^2\frac{d^2}{dy^2}$.
Since $\widehat \R^+$ is commutative multiplicative group with invariant measure $\frac{dy}{y}$,
The spectrum of $L^2(\R^+)$ (eigenfunctions of $\Delta$)
can also obtained via duality theory.

Let
\bna
\widehat{\R^+}=\{\chi:\R^+\rightarrow S^1,\quad\mbox{continuous}, \chi(ab)=\chi(a)\chi(b)\}.
\ena
be the group of the continuous characters of $\R^+$.
Via the differentiable map
$\log:\R^+\rightarrow\R$, we know that
the multiplicative character $\chi$ is of the form
\bna
\chi:\R^+\xrightarrow{\log}\R\xrightarrow{e()}S^1,
\quad
\chi(y)=e^{i\theta\log y}
\ena
for some $s=i\theta\in\R$. One has
\bna
\widehat \R^+\simeq i\R,\qquad \chi_{i\theta}\mapsto  i\theta.
\ena
\begin{remark}
Obviously $\chi_{s}(y)=y^{s}$ are eigenfunctions of $\Delta=-y^2\frac{d^2}{y^2}$ with eigenvalue
\bna
s(1-s).
\ena
\end{remark}
Note that $\chi_{s}(y)$
 are not in $L^2(\R^+)$.
For $f\in L^2(\R^+)$ and $\chi_\theta\in \widehat {R^+}$,
we have  the inner product
\bna
\varphi(i\theta)&=&
\langle f,\chi_{i\theta}\rangle =\int_{\R^+}f(y)e^{-i\theta\log y}\frac{dy}{y}
=\int_0^\infty f(y)y^{-i\theta}\frac{dy}{y}=\mathcal M(f)(-i\theta)
\ena
which gives a function in $s=i\theta\in i\R$.

If the spectrum are all discrete, we expect to establish
\bna
f(y)=\sum_{\theta\atop{\mbox{\tiny spectral}}}\frac{\langle f,\chi_{i\theta}\rangle }{
\langle\chi_{i\theta},\chi_{i\theta}\rangle} \chi_{i\theta};
\ena
If the spectrum are all continuous,  one expect that
\bna
f(y)=\int_{s=i\theta\atop{\mbox{\tiny spectrum}}}
\langle f,\chi_{i\theta}\rangle \chi_{i\theta}(y)
\frac{d\theta}{\langle\chi_{i\theta},\chi_{i\theta}\rangle}
\ena
But we know $\chi_{i\theta}$ is not in $L^2(\R^+)$ and
$\frac{1}{\langle\chi_{i\theta},\chi_{i\theta}\rangle}$ has no means. However, one can replace
it by a function $\mu(\theta)$, which is known as `spectral measure' (Plancheral measure), and establish
\bna
f(y)&=&\int_{-\infty}^{\infty}
\mathcal M(f)(-i\theta)\chi_{i\theta}(y) {\color{blue}\mu(\theta)}d\theta\\
&=&\frac{1}{2\pi i}\int_{-i\infty}^{i\infty}
\mathcal M(f)(i\theta)y^{-i\theta}\left(2\pi\mu(-\theta)\right)d(i\theta)\\
&=&\frac{1}{2\pi i}\int_{-i\infty}^{i\infty}M(f)(s)y^{-s} \left(2\pi\mu(is)\right)ds.
\ena
So if we can prove that the spectral measure $\mu(i\theta)=\frac{1}{2\pi}$,
then the theory of the spectral decomposition on $L^2(\R^+)$
is just the theory of Mellin and Mellin inverse transform,
by some tricks on analytic continuation for the convergence.

\begin{lemma}The spectral measure is $\mu(i\theta)=\frac{1}{2\pi}$.
\end{lemma}
\begin{proof}
choose suitable test function $f$ and consider the value of $f$ at a special point, for example, at $y=1$.

\end{proof}

\subsection{Proof via Fourier transform}
Recall the argument above,
\bna
\varphi(i\theta)=\int_{\R^+}f(y)e^{-i\theta\log y}\frac{dy}{y}.
\ena
On taking $y=e^{2\pi x}$,
\bna
\varphi(i\theta)=(\mathcal Mf)(i\theta)=
2\pi\int_{-\infty}^\infty f(e^{2\pi x})e^{-2\pi i x\theta}dx=\int_{-\infty}^\infty
F(x)e^{-2\pi i x\theta}dx,\qquad F(x)=2\pi f(e^{2\pi x}).
\ena
and thus
\bna
F(x)&=&2\pi f(e^{2\pi x})=2\pi f(y)\\
&=&\int_{-\infty}^\infty \varphi(i\theta) e^{2\pi i\theta x}d\theta
\overset{x=\frac{1}{2\pi}\log y}{=}
\int_{-\infty}^\infty \varphi(i\theta) y^{i\theta }d\theta
\ena
which is
\bna
f(y)&=&\frac{1}{2\pi}\int_{-\infty}^{\infty}(\mathcal Mf)(-i\theta)y^{i\theta}d\theta\\
&=&\frac{1}{2\pi i}\int_{\re(s)=0}(\mathcal Mf)(s)y^{-s}ds.
\ena
We finish the proof of the Mellin transform.
\subsection{Useful table for Mellin and Mellin inversion formula}\label{subsec-useful-Mellin-transform}
We list the following useful table for Mellin transforms.
\begin{align*}
f(y),y>0&\quad &&\mathcal M(f)(s)\\
\exp(-ay),y>0&\quad &&a^{-s}\Gamma(s)\\
\exp\left[-\frac{a}{2}(y+\frac{1}{y})\right],a>0&\quad
&&2K_s(a)
\end{align*}

An application of the Mellin transform can be found in Lemma \ref{lemma-integral-Whitaker-Bessel}.

\section{Gauss Hypergeometric function}
The Gauss hypergeometric function is defined by
\bea
F(\alpha,\beta,\gamma,z)&=&\sum_{ n\geq 0}\frac{(\alpha)_n(\beta)_n}{n!(\gamma_n)} z^n,\quad |z|<1
\nonumber\\
&=&\frac{\Gamma(\gamma)}{\Gamma(\alpha)\Gamma(\beta)}\sum_{n\geq 0}\frac{1}{n!}
\frac{\Gamma(\alpha+n)\Gamma(\beta+n)}{\Gamma(\gamma+n)} z^n,\quad |z|<1.
\label{hypergeometric-func}
\eea
where
\bna
(\alpha)_0=1,\quad (\alpha)_n=\frac{\Gamma(\alpha+n)}{\Gamma(\alpha)}.
\ena

Here \eqref{hypergeometric-func} is defined for $|z|<1$. $F(\alpha,\beta,\gamma,z)$
is analytic continuated to $z\in\C$ except for $z=1$ and $z=\infty$,
which may be branch points of $F(\alpha,\beta,\gamma,z)$.

\subsection{Integral representations}
We have
\bna
F(\alpha,\beta,\gamma,z)=\frac{\Gamma(\gamma)}{\Gamma(\beta)\Gamma(\gamma-\beta)}
\int_0^1 t^{\beta-1}(1-t)^{\gamma-\beta-1}(1-zt)^{-\alpha}dt,
\quad \re(\gamma)>\re(\beta)>0,\quad |\arg(1-z)|>\pi.
\ena
Here $\frac{1}{z}$ is the singular pint of $\frac{1}{(1-zt)^{-\alpha}}$(except that $\alpha$ is negative integer),
and thus $z=1$ and $z=\infty$ are branches points.

\subsection{Barnes integral representations}
We have
\bna
F(\alpha,\beta,\gamma,z)=\frac{\Gamma(\gamma)}{\Gamma(\alpha)\Gamma(\beta)}
\frac{1}{2\pi i}\int_{-i\infty}^{i\infty}\frac{\Gamma(\alpha+s)\Gamma(\beta+s)}{\Gamma(\gamma+s)}
\Gamma(-s) (-z)^sds,
\ena
where one need that $\arg(-z)<\pi$ and
$\re(s)$ is in the left of poles of $\Gamma(-s)$,
and right of poles of $\Gamma(\alpha+s)\Gamma(\beta+s)$. This implies that
\bna
\mbox{$\alpha$ and $\beta$  are not zero or negative integers.}
\ena
\begin{lemma}[Barnes lemma]
For $\gamma-\alpha-\beta\in \Z$,
\bna
\frac{1}{2\pi i}\int_{-i\infty}^{i\infty}\Gamma(\alpha+s)\Gamma(\beta+s)
\Gamma(\gamma-s)\Gamma(\delta-s)ds
=\frac{\Gamma(\alpha+\gamma)\Gamma(\alpha+\delta)
\Gamma(\beta+\gamma)\Gamma(\beta+\delta)}
{\Gamma(\alpha+\beta+\gamma+\delta)}
\ena
\end{lemma}
\subsection{Special values}
One has
\bna
F(\alpha,\beta,\gamma,1)=\frac{\Gamma(\gamma)\Gamma(\gamma-\alpha-\beta)}
{\Gamma(\gamma-\alpha)\Gamma(\gamma-\beta)},
\quad \gamma\notin\{0\}\cup\Z^-,\quad \gamma-\alpha-\beta\notin\Z
\ena
Other cases are the following.
\bit
\item [1.]
If $\alpha-\beta=m$ with $m\in \mathbb N$,  see formula (2) in page 120
\item [2.] If $\gamma-\alpha-\beta\in \Z$, see (8) in page 123
\item [3.] If $\gamma-\alpha-\beta\in \mathbb N$, see (9) in page 123
\item [4.]
Moreover, if $\alpha$ or $\beta$ is $-n$, in this case, $F(\alpha,\beta,\gamma,z)$
is a polynomial, called Jacobi polynomial (see (1) page). In this case,
\bna
F(-n,\beta,\gamma,1)=\frac{\Gamma(\gamma)\Gamma(\gamma-\beta+n)}{\Gamma(\gamma+n)\Gamma(\gamma-\beta)}.
\ena
\eit
\subsection{Relation with Chebechy polynomial}
\subsection{Kummer's formula}
\bna
F(\alpha,\beta,1+\alpha-\beta,-1)=\frac{\Gamma(1+\alpha-\beta)\Gamma(1+\frac{\alpha}{2})}
{\Gamma(1+\alpha)\Gamma(1+\frac{\alpha}{2}-\beta)}
\ena
\subsection{asymptotic formulas}As $z\rightarrow\infty$ and the parameter $\gamma\rightarrow\infty$,
see section 4.14.

\section{Legendre functions}
Legendre functions  are solutions of the Legendre equation
\bea
(1-x^2)\frac{d^2y}{dx^2}-2x\frac{dy}{dx}+\nu(\nu+1)y=0,\label{Legendre-equation}
\eea
which is obtained via the separable method from the Laplacian differential equation
under the polar coordinate.


\subsection{Legendre polynomial}
Case $\nu=n\in\mathbb N$, \eqref{Legendre-equation} has solutions, which are called
Legendre polynomials,
\bna
P_n(x)&=&\frac{1}{2^n}\sum_{r=0}^{[\frac{n}{2}]}
(-1)^r\frac{(2n-2r)!}{r!(n-r)!(n-2r)!} x^{n-2r}\\
&=&\frac{(2n)!}{2^n (n!)^2}x^nF\left(-\frac{n}{2},\frac{1-n}{2},\frac{1}{2}-n,x^{-2}\right)\\
&=&F(n+1,-n,1,\frac{1-x}{2})
\ena
and $P_n(1)=1$.

One has integral representation of
\bna
P_n(x)=\frac{1}{\pi}\int_0^\pi (x+\sqrt{x^2-1}\cos\theta)d\theta
\ena
and the recurrent formulas
\bna
&P_1(x)-xP_0(x)=0&\\
&(n+1)P_{n+1}(x)-(2n+1)xP_n(x)+nP_{n-1}(x)=0,\quad n\geq 1.&
\ena
\subsubsection{Orthogonal property of Legendre polynomials}
\begin{prop}Let $f(x)$ be a polynomial with $\deg f=k$. If $k<n$, one has
\bna
\int_{-1}^1 f_k(x)P_n(x)dx=0
\ena
Moreover,
\bna
\int_{-1}^1P_m(x)P_n(x)=\frac{2}{2n+1}\delta_{m,n}.
\ena
If $(1-x^2)^{-1/4}f(x)$ is integrable over $[-1,1]$, then
\bna
\frac{1}{2}(f(x+0)+f(x-0))=\sum_{k\geq 0}\frac{2k+1}{2}P_k(x)
\int_{-1}^1 f(t)P_k(t)dt
\ena
\end{prop}
\subsubsection{Zeroes of $P_n(x)$}
$P_n(x)$ has $n$ number of simple pole appears in $[-1,1]$.

\subsection{Lengedre function}
The solutions of
\bna
(1-z^2)\frac{d^2u}{dz^2}-2z\frac{du}{dz}+[\nu(\nu+1)-\frac{\mu^2}{1-z^2}]u=0
\ena
are the associated Legendre function defined by
\bna
P_\nu^\mu(z):=\frac{1}{\Gamma(1-\mu)}\left(\frac{z+1}{z-1}\right)^{\mu/2}
F(-\nu,\nu+1,1-\mu,\frac{1-z}{2}),
\quad |\arg(z\pm 1)|<\pi.
\ena
One has integral representation
\bna
P_s^a(z)=
\frac{\Gamma(s+a+1)}{2\pi\Gamma(s+1)}\int_0^{2\pi} (z+\sqrt{z^2-1}\cos\alpha)^s
e^{ia\alpha}d\alpha,\quad \re(z)>0
\ena

For the order $a=0$,
\bna
P_s(z)&\sim&\frac{1}{\sqrt{\pi}}\frac{\Gamma(s+\frac{1}{2})}{\Gamma(s+1)} (2z)^s,\quad z\rightarrow\infty\\
P_{-s}(z)&=&P_{s-1}(z),\quad \re(z)>0\\
P_s(z)&=&F(-s,s+1,1,\frac{1-z}{2}),\quad |z-1|<2.
\ena


\section{Philosophy of Eisenstein series}\label{Sec-Eisenstein}
This part is based on Casselman's paper at Casselman's homepage.

\subsection{The lift of classical modular and maass forms to $GL_2$}\label{section-lift}
Let $\Gamma$ be a principal congruence subgroup of level $N$ in $SL(2,\mathbb Z)$, $\mathfrak h^2$ the Poincare
upper half plane. By strong approximate theorem, we know
\bea
\Gamma\backslash \mathfrak h^2\longrightarrow GL_2(\mathbb \Q)\backslash GL_2(\mathbb A)/K_\mathbb R K_f
\label{strong-approximate-formula}
\eea
is bijective,
where $K_\mathbb R=SO(2)$, and $K_f$ is compact open subgroup of $GL_2(\mathbb A_f)$ defined in the following.
\begin{itemize}
\item For $p\nmid N$, define $K_p=GL_(\Z_p)$.
\item For $p\mid N$, consider the diagonal embedding
$$
SL(2,\Z)\hookrightarrow \prod_{p\mid N}GL_2(\Z_p).
$$
Let $K_N$ be open subgroup of $\prod_{p\mid N}GL_2(\Z_p)$ such that
\bit
\item the preimage of $K_N$ in $SL(2,\Z)$ is $\Gamma$
\item $\det\left(K_N\right)=\prod_{p\mid N}\mathbb Z_p^\times$.
\eit
Eg, $$K_N=\left\{k, k\equiv \bma 1&\\&*\ema\bmod N\right\}$$
\item Let $K_f=K_N\prod_{p\nmid N}K_p$.
\end{itemize}

\remark By (\ref{strong-approximate-formula}), we can lift the maass cusp form on $\Gamma\backslash \mathfrak h^2$
to be a function on $GL_2(\Q)\backslash G(\A)$ which is fixed by right action under $K_\R K_f$,
and lift holomorphic modular form to be a function on $GL_2(\Q)\backslash G(\A)$ fixed by $K_f$
and transforming  in a certain way under $K_\R$.

\subsection{Hecke operators}
For any $g\in G(\A_f)$, we can define Hecke operator $T_g$ in the following.
\bit
\item Consider the double cosets $K_fg K_f$, we have right coset decomposition
$$
K_fg K_f=\bigcup_{i}g_iK_f.
$$
Thus for $f\in GL(2,\Q)\backslash GL(2,\A)/K_\R K_f$, define the action of $T_g$  on $f$ via
$$
T_gf(x)=\sum_i f(xg_i)
$$
\item Adele scheme allows us to separable global problem into local ones. For $p\nmid N$,
the Hecke operator $T_p$ and $T_{p,p}$ corresponds to
$$
K_p\bma 1&\\&p\ema K_p,\qquad K_p\bma p&\\&p\ema K_p,
$$
and the action of $T_p$ and $T_{p,p}$ can be expressed as convolution with characteristic function of the above double cosets.
\eit


\subsection{Automorphic forms}
Let $G$ be a reductive group defined over a number field $F$.
Automorphic forms on $G$ is just functions $f:G(F)\backslash G(\A)\rightarrow \C$ satisfying
\bit
\item a condition of moderate growth on adele version of Siegel set,
\item smooth at real primes, and is $(Z(\mk g^\C_\infty),K_\infty)$-finite,
\item fixed with respect to the right translation of open subgroup $K_f$ of $G(\A_f)$.
\eit

Here $G$ is reductive over $F$. We know that $G$ is unramified over $F_v$ for almost all $v$.
That is to say, $G$ is unramified outside a finite set of primes $D_G$, or equivalently
\bit
\item $G$ arises by base extension from smooth reductive group over $O_F[1/N]$ for some  integer $N$,
\item For $p\mid N$, $G/F_p$ arise by base extension from smooth reductive group scheme over $O_p$, i.e.
$$
G(F_p)=G(O_p)\otimes_{O_p}F_p.
$$
\eit

\remark In general, we dealing with automorphic forms on connected reductive groups $G$ over $F$ with center character
defined by a Hecke character $\omega$.

Hecke operators are determined through convolution by functions on $K_f\backslash G(\A_f)/K_f$ with $K_f$ as above.
One can express $K_f=K_S\prod_{p\notin S}K_p$ where $S\supset D_G$ such that $K_p=G(O_F)$ for $p\notin S$.
So by local global principle, the Hecke operators involving convolution of functions
(generated by characteristic functions of double cosets) on $K_p\backslash G(F_p)/K_p$,
i.e. functions in $\mathcal H(G(F_p),K_p)$ for $p\notin S$.
We will show that $\mathcal H(G(F_p),K_p)$ is commutative algebra, and the action on spherical vectors
gives a scalar, by prove the Satake isomorphism.


\subsection{Eisenstein series on $\mathfrak h$}
Let $\Gamma=SL(2,\Z)$ and $z=x+iy\in\mathfrak h$. For $s\in\C$ with $\re(s)>1/2$,
\bna
E_s(z):=\sum_{c\geq 0\atop{(c,d)=1}} \frac{y^{s+\frac{1}{2}}}{|cz+d|^{2s+1}}=\sum_{\Gamma_P\backslash \Gamma} Im(\gamma.z)^{s+\frac{1}{2}},
\ena
where
\bna
\Gamma_P=\{\gamma.(i\infty)=(i\infty),\gamma\in\Gamma\}=P\cap \Gamma=\left\{\pm I,\bma 1&n\\&1\ema,n\in \Z\right\}
\ena
It is well defined since $Im(\gamma.z)$ is $\Gamma_P$ invariant
and is absolutely convergent for $\re(s)>1/2$. Moreover,
\bit
\item It is eigenfunction of Laplacian operator with eigenvalue
$$
\Delta E_s=(s^2-\frac{1}{4})E_s.
$$
\item
$$
E_s(z)\sim y^{1/2+s}+c(s)y^{\frac{1}{2}-s},\quad y\rightarrow \infty.
$$
i.e., the constant term along parabolic subgroup $P$ is
$$
\int_0^1 E_s(x+iy)dx=y^{1/2+s}+c(s)y^{\frac{1}{2}-s}
$$
\item It satisfies the functional equation
$$
E_s(z)=c(s)E_{1-s}(z)
$$
which implies that
$$
c(s)c(1-s)=1.
$$
\eit
We can calculate the constant term directly by
\bna
\int_0^1 E_s(x+iy)dx
&=& y^{s+\frac{1}{2}}+y^{s+\frac{1}{2}}\left(\int_{-\infty}^\infty\frac{dw}{|w^2+y^2|^{s+\frac{1}{2}}}\right)\sum_{c>0}\frac{\varphi(c)}{c^{2s+1}}\\
&=& y^{s+\frac{1}{2}}+y^{s+\frac{1}{2}}\frac{\zeta_\R(2s)}{\zeta_\R(2s+1)}\sum_{c>0}\frac{\varphi(c)}{c^{2s+1}}
\ena
where $\zeta_\R(s)=\pi^{-\frac{s}{2}}\Gamma(s/2)$, and $\varphi(c)=\sum_{n\leq c\atop{(n,c)=1}}1$. Moreover,
\bna
\sum_{c>0}\frac{\varphi(c)}{c^{2s+1}}=\prod_p \sum_{n\geq 0}\frac{\varphi(p^n)}{p^{n(2s+1)}}
=\frac{\zeta(2s)}{\zeta(2s+1)}
\ena
It implies that
$$
c(s)=\frac{\xi(2s)}{\xi(2s+1)}.
$$

\subsection{The lift of classical Eisenstein series}
We can lift the Eisenstein series to $SL(2,\R)$.
Via the decomposition $SL(2,\R)=BK$, or
\bna
SL(2,\R)\rightarrow\mathfrak h^2,\quad \bma a&b\\c&d\ema\mapsto \frac{ai+b}{ci+d}.
\ena
we can define
$$
\tilde E_s(g_\R)=E_s(g.i).
$$
Moreover, since $$SL(2,\A)=SL(2,\Q)(SL(2,\R)K_f)$$ with $K_f=\prod_p K_p$,
we can lift the Eisenstein series to be functions on $SL(2,\A)$ via
$$
\mathcal E_s(\gamma g_\R k_f)= \tilde E_s(g_\R)=E_s(g_\R.i),\quad g=\gamma g_\R k_f.
$$

\subsection{Philosophy of cusp form}
Recall Parabolic induction in representation of Lie groups.
Consider $G=SL(2,\R)$ and $B(\R)=M(\R)\ltimes N(\R)$.
Let $\sigma$  be a representation of the levi-component $M(\R)$. We inflate it to be a representation of $P(\R)$ by letting $N(\R)$
acting trivially,  and normalized parabolic induced to be a representation of $G(\R)$ via
$$
i_{M,P}^G \sigma=ind_{P}^G(\sigma\otimes\delta^{1/2})
$$

Let $\Gamma$ be a principal  congruence  subgroup of $SL(2,\Z)$.
The question is,
how to obtain an automorphic forms on $G(\R)$ w.r.t. $\Gamma$ from automomrphic forms on $M(\R)$ via the above philosophy?
\bit
\item
Assume we have an automorphic form on $M(\R)$ with respect to $M_\Gamma=M(\R)\cap \Gamma$, i.e. a function
$$
I_s:M_\Gamma\backslash M(\R)\rightarrow \C,\quad I_s\left(\bma \pm y^{\frac{1}{2}}\\&\pm y^{-1/2}\ema \right)= |y|^s
$$
\item Inflate it to be a function on $P(\R)$ via letting $N(\R)$ acts trivially,
we have an automorphic forms on $P$, namely
$$
I_s:P_\Gamma N(\R)\backslash P(\R)\rightarrow \C,\quad I_s\left(\bma \pm y^{\frac{1}{2}}&x\\&\pm y^{-1/2}\ema\right)= |y|^s
$$
\item Finally, we use the normalized induction to obtain a function on $G(\R)$, i.e. to obtain a function
$$
\varphi_s: P_\Gamma N(\R)\backslash G(\R)\rightarrow\C,\quad \varphi_s\left(\bma \pm y^{1/2}&x\\&\pm y^{-1/2}\ema g\right)=|y|^{s+\frac{1}{2}}\varphi_s(g).
$$
Note that we also need $\varphi_s$ is right $K_\R$-finite.
\item The above function $\varphi_s$ is defined on $G(\R)$, but only with automorphism with respect to $P_\Gamma$.
To obtain a real automorphic form on $G(\Z)\backslash G(\R)$, we finally define
$$
\mathcal E(g,\varphi_s):=\sum_{\gamma\in P_\Gamma\backslash \Gamma}\varphi_s(\gamma g)
$$
It is an automorphic forms on $\Gamma\backslash G(\R)$.
\eit

\subsection{Constant term of Eisenstein series}
Now we consider the constant term of Eisenstein series.
Recall that if $\Phi$ is an automorphic form on $\Gamma\backslash G(\R)$, the constant term is defined by
$$
\hat \Phi(g)=\int_{N(\Z) \backslash N(\R)}\Phi(n g)dn
$$
Since for $\gamma\in P_\Gamma$, $ n\gamma g= \gamma n'g$.
It is easy to see that $\hat \Phi(g)$ is a function on $P_\Gamma N(\R)\backslash G(\R)$.


\remark The constant term along parabolic subgroup $P=M^P\ltimes N^P$ is defined by
$$
\hat\Phi_P(g):=\int_{ N^P(\Z)\backslash N^P(\R)}\Phi(ng)dn
$$
maps automorphic form to be function on $P_\Gamma N(\R)\backslash G(\R).$

\subsection{Adele version}
Recall that we have
\bna
\A^\times=\Q^\times(\R^\times_+ \prod_p \Z_p^\times).
\ena
For $G=SL_2$, it gives
\bna
G(\A)= G(\Q)\left(G(\R)(\prod_p G(Z_p))\right),\quad M(\A)= M(\Q)\left(M(\R)(\prod_p M(Z_p))\right).
\ena
Let $\Gamma$ be a principal congruent subgroup of level $N$ in $SL(2,\Z)$. Recall $K_f=K_N\prod_p K_p$ is defined in
section \ref{section-lift}.

Question: How to obtain an Adele analogue of function $\varphi_s$ on $P_\Gamma N(\R)\backslash G(\R)$?
\bit
\item
By Langlands decomposition, we have
$$
G(\A)=P(\A)K_\R K_f
$$
Moreover, the decomposition
\bna
P(\A)=M(\A)N(\A),\quad M(\A)=M(\Q)(M(\R)(\prod_p M(Z_p))
\ena
implies that
\bea
P(\Q)N(\A)\backslash G(\A)/K_f\simeq  P_\Gamma N(\R)\backslash G(\R)
\eea
\item Let $\varphi_s$ be the unique function on $P(\Q)N(\A)\backslash G(\A)/K_\R K_f$
such that
\bna
\varphi_s(pg)=\delta_P(p)^{1/2+s}\varphi_s(g),\qquad \varphi_s(1)=1.
\ena
Here $\delta_P$ is the modulus function on $P(\A)$ defined by the product of local ones.
\item The Eisenstein series $\mathcal E(*,\varphi_s)$ is defined by
$$
\mathcal E(g,\varphi_s):=\sum_{\gamma\in P(\Q)\backslash G(\Q)}\varphi_s(\gamma g).
$$
\eit

Question: How about the constant term of the Eisenstein series?
\bna
\int_{N(\Q)\backslash N(\A)}\mathcal E_s(ng)dn=\int_{N(\Q)\backslash N(\A)}\sum_{\gamma\in P(\Q)\backslash G(\Q)}
\varphi_s(\gamma n g)dn.
\ena
\bit
\item
The idea is to use the Bruhat decomposition, i.e.
$$
G(\Q)=P(\Q)\cup \bigcup_{w\neq 1} P(\Q) w N_w'(\Q),\quad
P(\Q)\backslash G(\Q)=\{1\}\cup\bigcup_{w\neq 1} w N_w'(\Q),
$$
where $N=N_w\cdot N_w'=N_w'\cdot N_w$ with
$$
N_w= wNw^{-1}\cap N,\quad N_w'=w\overline {N} w^{-1}\cap N.
$$
\item Thus we can express the constant term as
$$
E_P(g,\varphi_s)=\varphi_s(g)+\sum_{w\neq 1}\int_{N_w'(\A)}\varphi_s(w ng)dn.
$$
In the case $G=SL_2$, we have
$$
E_P(g,\varphi_s)=\varphi_s(g)+\int_{N(\A)}\varphi_s(w ng)dn.
$$
\item
So we need to calculate the local integral
$$
\int_{N(\Q_v)}\varphi_{s,v}(w n)dn,
$$
where
$\int_{N(\Q_v)}\varphi_{s,v}(wn)dn$ is a constant given by intertwining operator acts
on spherical vector of spherical representations, for almost all $v$.
\eit


\section{Some integral transform}
We review the proof of Lemma \ref{lemma-integral-in-bump} (Lemma \ref{lemma-integral-Whitaker-Bessel}).
\begin{lemma}
We have
\bna
\pi^{-s}\Gamma(s)y^s\int_{-\infty}^\infty\frac{1}{(x^2+y^2)^s}e(-nx)dx
=\left\{
\begin{aligned}
&\pi^{-s+\frac{1}{2}}\Gamma(s-\frac{1}{2})y^{1-s},\quad && n=0\\
&2|n|^{s-\frac{1}{2}}\sqrt{y}K_{s-\frac{1}{2}}(2\pi|n|y),\quad &&n\neq 0.
\end{aligned}
\right.
\ena
\end{lemma}
\begin{proof}
Note that
\bna
\Gamma(s)=\int_0^\infty e^{-t}t^{s}\frac{dt}{t}.
\ena
Thus
\bna
&&\pi^{-s}y^s\int_{0}^\infty e^{-t}t^s\frac{dt}{t}
\int_{-\infty}^\infty\frac{1}{(x^2+y^2)^s}e(-nx)dx\\
&=&\int_{-\infty}^\infty\left(\int_0^\infty
\left(\frac{y}{\pi(x^2+y^2)}t\right)^{s} e^{-t}\frac{dt}{t}
\right)dx
=\int_{-\infty}^\infty\left(\int_0^\infty
t^{s} e^{-t\frac{\pi(x^2+y^2)}{y}}\frac{dt}{t}
\right)dx\\
&=&\int_0^\infty
t^{s} e^{-t\pi y}
\left(\int_{-\infty}^\infty
 e^{-\frac{\pi t}{y}x^2} e^{-2\pi i n x}dx\right)
 \frac{dt}{t}
\ena
Since
\bna
\int_{-\infty}^\infty
 e^{-\frac{\pi t}{y}x^2} e^{-2\pi i n x}dx=\left\{
 \begin{aligned}
 &\sqrt{\frac{y}{t}},\quad &n=0\\
 &\sqrt{\frac{y}{t}}e^{-\frac{\pi y n^2}{t}},\quad &n\neq 0
 \end{aligned}
 \right.
\ena
The result follows immediately by the expression of $\Gamma$-function and $K$-Bessel
function.
\end{proof}

Next, we consider \eqref{K-bessel-integral-eq}.
\begin{lemma}\label{lemma-K-bessel-eq}
One has
\bna
\int_0^\infty  e^{-y}y^{s-\frac{1}{2}}K_{\nu}(y)\frac{dy}{y}
=\pi^{1/2}2^{\frac{1}{2}-s}\frac{\Gamma(s-\frac{1}{2}-\nu)\Gamma(s-\frac{1}{2}+\nu)}{\Gamma(s)}
\ena

\end{lemma}
\begin{proof}
Note that $K$-Bessel function is Mellin transform of $\frac{1}{2}e^{a(y+\frac{1}{y})}$, (see \eqref{integral-transform-K-Bessel-in-appendix}),
\bna
\int_0^\infty e^{-\frac{y}{2}(t+\frac{1}{t})}t^s\frac{dt}{t}=2 K_s(y).
\ena
Applying this one has
\bna
L.H.S.&=&\int_0^\infty e^{-y}y^{s-\frac{1}{2}}
\frac{1}{2}\int_0^\infty t^\nu e^{-\frac{y}{2}(t+\frac{1}{t})}\frac{dt}{t}
\frac{dy}{y}\\
&=&\frac{1}{2}\int_0^\infty \int_0^\infty  t^{\nu}y^{s-\frac{1}{2}}
 t^\nu e^{-\frac{y}{2}(t+\frac{1}{t}+2)}\frac{dt}{t}
\frac{dy}{y}\\
\ena
Let $yt=u_1$, $\frac{y}{t}=u_2$ and thus
\bna
y=\sqrt{u_1u_2},\quad t=\sqrt{\frac{u_1}{u_2}}
\ena
and thus
\bna
dy=\frac{1}{2}\sqrt{\frac{u_2}{u_1}}du_1+\frac{1}{2}\sqrt{\frac{u_1}{u_2}}du_2\\
dt=\frac{1}{2}\sqrt{\frac{1}{u_1u_2}}du_1-\frac{1}{2}\sqrt{\frac{u_1}{u_2^3}}du_2\\
\ena
and thus
\bna
ty=u_1,\quad dydy=-\frac{1}{2}\frac{1}{u_2}du_1du_2.
\ena
It gives
\bna
L.H.S.&=&\frac{1}{2}\int_0^\infty\int_0^\infty e^{-\sqrt{u_1u_2}} (u_1u_2)^{\frac{s}{2}-\frac{1}{4}}\left(\frac{u_1}{u_2}\right)^{\frac{\nu}{2}}\frac{1}{2}\frac{du_1du_2}{u_1u_2}\\
&=&\frac{1}{4}\int_0^\infty\int_0^\infty e^{-\sqrt{u_1u_2}} u_1^{\frac{s}{2}+\frac{\nu}{2}-\frac{1}{4}}u_2^{\frac{s}{2}-\frac{\nu}{2}-\frac{1}{4}}
e^{-\frac{(\sqrt{u_1}+\sqrt{u_2})^2}{2}}\frac{du_1du_2}{u_1u_2}\\
\ena

\end{proof}

\begin{thebibliography}{100}

\bibitem[IwKo2004]{IwKo2004}
H. Iwaniec and E. Kowalski, {\it Analytic number theory},
American Mathematical Society Colloquium Publications, 53.
American Mathematical Society, Providence, RI, 2004.

\bibitem[Ap2013]{Ap2013}
T.M. Apostol, {\it Introduction to analytic number theory},
Undergraduate Texts in Mathematics. Springer-Verlag, New York-Heidelberg, 1976.

\bibitem[BaFr2016]{BF1} O. Balkanova and D. Frolenkov, {\it A uniform asymptotic formula for the second moment of primitive $L$-functions on the critical line},
    Proceedings of the Steklov Institute of Mathematics 294.1 (2016): 13-46.

\bibitem[BaFr2018]{BF} O. Balkanova and D. Frolenkov, {\it Non-vanishing of automorphic L-functions of prime power level},
    Monatshefte f\"ur Mathematik 185.1 (2018): 17-41.


\bibitem[Bu1998]{B}
D. Bump, {\it Automorphic forms and representations}, Cambridge Studies in Advanced Mathematics, 55. Cambridge University Press, Cambridge, 1997.


\bibitem[Bl2019]{Bl2019}
V. Blomer,
{\it The relative trace formula in analytic number theory},
arXiv:1912.08137.
\bibitem[BuHe2006]{BuHe2006}
C.J. Bushnell and G. Henniart,
{\it The local Langlands conjecture for $GL(2)$},
Grundlehren der Mathematischen Wissenschaften, 335. Springer-Verlag, Berlin, 2006.

\bibitem[By1996]{By} V. Bykovskii, {\it A trace formula for the scalar product of Hecke series and its applications}, Journal of Mathematical Sciences 89.1 (1998): 915-932.

\bibitem[ByFr2017]{BF2}V. Bykovskii and D. Frolenkov, {\it Asymptotic formulas for the second moments of
$L$-series associated to holomorphic cusp forms on the critical line}, Izvestiya: Mathematics 81.2 (2017): 239-268.



\bibitem[Co2004]{Co2004}
J.W. Cogdell, {\it Analytic Theory of $L$-Functions for $GL_n$},
An Introduction to the Langlands Program. Birkh\"auser, Boston, MA, 2003. 197-228.


\bibitem[DeIw1982]{DeIw1982}
J.M. Deshouillers and H. Iwaniec, {\it Kloosterman sums and Fourier coefficients of cusp forms}, Inventiones mathematicae 70.2 (1982): 219-288.


\bibitem[Du1995]{Du1995}
W. Duke, {\it The critical order of vanishing of automorphic L-functions with large level}, Inventiones Mathematicae 119.1 (1995): 165-174.


\bibitem[DuGu1975]{DuGu1975}
J.J. Duistermaat and V.M. Guillemin,
{\it The Spectrum of Positive Elliptic Operators and Periodic Bicharacteristics}. Inventiones mathematicae 29.1 (1975): 39-79.

\bibitem[FeWh2009]{FW}
B. Feigon and D. Whitehouse,
{\it Averages of central $L$-values of Hilbert modular forms with an application to subconvexity},
Duke Mathematical Journal 149.2 (2009): 347-410.


\bibitem[GaHoSe2009]{GaHoSe2009}
S. Ganguly, J. Hoffstein and J. Sengupta, {\it Determining modular forms on $SL_2(\mathbb Z)$ by central values of convolution L-functions},
Mathematische Annalen 345.4 (2009): 843-857.


\bibitem[Ge1996]{Ge}
S. Gelbart, {\it Lectures on the Arthur-Selberg trace formula},
University Lecture Series, 9. American Mathematical Society, Providence, RI, 1996.

\bibitem[Go2006]{Go}
D. Goldfeld,
{\it Automorphic Forms and L-functions for the Group
$GL(n,\R)$}, Cambridge Studies in Advanced Mathematics, 99. Cambridge University Press, Cambridge, 2006.

\bibitem[HoLo1994]{H-L}J. Hoffstein and P. Lockhart, {\it Coefficients of Maass forms and the Siegel zero}, Ann. of Math. 140.2 (1994): 161-181.

\bibitem[Iw2002]{Iw2002-spectral-method}
H. Iwaniec,
{\it Spectral methods of automorphic forms},
Graduate Studies in Mathematics, 53. American Mathematical Society, Providence, RI; Revista Matem$\acute{a}$tica Iberoamericana, Madrid, 2002.

\bibitem[IwKo2004]{IwKo2004}
H. Iwaniec and E. Kowalski, {\it Analytic number theory},
American Mathematical Society Colloquium Publications, 53.
American Mathematical Society, Providence, RI, 2004.

\bibitem[IwLuSa2000]{IwLuoSarnak-lowlying-zeroes}
H. Iwaniec, W. Luo and P. Sarnak,
{\it Low lying zeros of families of $L$-functions},
Publications Math$\acute{e}$matiques De Linstitut Des Hautes $\acute{e}$tudes Scientifiques 91.1 (2000): 55-131.


\bibitem[IwSa2000]{IwSa2000}
H. Iwaniec and P. Sarnak, {\it The non-vanishing of central values of automorphicL-functions and Landau-Siegel zeros}, Israel Journal of Mathematics 120.1 (2000): 155-177.


\bibitem[Ja1972]{Ja}
H. Jacquet, {\it Automorphic Forms on GL(2) Part II}, Lecture Notes in Mathematics, Vol. 278. Springer-Verlag, Berlin-New York, 1972.

\bibitem[JaYe1996]{JaYe}
H. Jacquet and Y. Ye, {\it Distinguished representations and quadratic base change for $GL(3)$},
Transactions of the American Mathematical Society 348.3 (1996): 913-939.

\bibitem[JaKn2015]{JaKn}
J. Jackson, and A. Knightly,
{\it Averages of twisted $L$-functions},
Journal of the Australian Mathematical Society 99.2 (2015): 207-236.

\bibitem[Kh2010]{Khan2010}
R.R. Khan, {\it Non-vanishing of the symmetric square $L$-function},
Proceedings of the London Mathematical Society 100.3 (2010): 736-762.

\bibitem[KiSa2003]{KS2003}H. Kim and P. Sarnak,
{\it Appendix: Refined estimates towards the Ramanujan and Selberg conjectures},
J. Amer. Math. Soc 16.1 (2003): 175-181.


\bibitem[KnLi2006a]{KnLi2010a}
A. Knightly and C. Li,
{\it A relative trace formula proof of the Petersson
trace formula}, Acta Arithmetica 122.3 (2006): 297-313.

\bibitem[KnLi2006b]{KnLi2006}
A. Knightly and C. Li. {\it Traces of Hecke Operators},
 Mathematical Surveys and Monographs, 133. American Mathematical Society, Providence, RI, 2006.


\bibitem[KnLi2010]{KnLi2010b}
A. Knightly and C. Li,
{\it Weighted averages of modular $L$-values},
Transactions of the American Mathematical Society 362.3 (2010): 1423-1443.


\bibitem[KnLi2012]{KnLi2012}
A. Knightly and C. Li, {\it Modular L-values of cubic level},
Pacific Journal of Mathematics 260.2 (2012): 527-563.

\bibitem[KnLi2013]{KnLi2013}
A. Knightly and C. Li,
{\it Kuznetsov's Trace Formula and the Hecke Eigenvalues of Maass Forms},
Mem. Amer. Math. Soc. 224 (2013), no. 1055.



\bibitem[KnLi2015]{KnLi2015}
A. Knightly and C. Li, {\it
Simple supercuspidal representations of ${\rm GL}(n)$},
Taiwanese Journal of Mathematics 19.4 (2015): 995-1029.

\bibitem[Ko1998]{Ko1998}
E. Kowalski, {\it The rank of the jacobian of modular curves: analytic methods},
Ph.D. thesis, Rutgers University, May 1998, http://www.math.ethz.ch

\bibitem[KoMi1999]{KoMi1999}
E. Kowalski and P. Michel, {\it The analytic rank of $J_0(q)$ and zeros of automorphic L-functions}, Duke Mathematical Journal 100.3 (1999): 503-542.

\bibitem[LaM\"{u}2009]{LaMu2009}
E. Lapid and W. M\"{u}ller, {\it Spectral asymptotics for arithmetic quotients of $SL(n,\mathbb R)/SO(n)$}, Duke Mathematical Journal 149.1 (2009): 40-44.

\bibitem[Ku1983]{Ku} N.V. Kuznetsov, {\it Convolution of the Fourier coefficients of Eisentein-Maass series}, Auomorphic functions and number theory. Part I, Zap. Nauchn. Sem. LOMI 129 (1983): 43-84.

\bibitem[LaWa2011]{Lau-Wang-Quantitive-joint-distribution}
Y.K. Lau and Y. Wang, {\it Quantitative version of the joint distribution of eigenvalues of the Hecke operators}, Journal of Number Theory 131.12 (2011): 2262-2281.

\bibitem[Li2011]{Li2011}
X. Li, {\it A weighted Weyl law for the modular surface},
International Journal of Number Theory 7.1 (2011): 241-248.




\bibitem[Luo2015]{Luo2015}
W. Luo, {\it Nonvanishing of the central $L$-values with large weight},
Adv. Math. 285 (2015): 220-234.

\bibitem[Nel2017]{Nel2017} P. Nelson,
{\it Analytic isolation of newforms of given level}, Arch. Math. 108 (2017): 555-568.

\bibitem[Popa2008]{Popa0}
A.A. Popa, {\it
Whittaker newforms for archimedean representations of $GL(2)$},
Journal of Number Theory 128.6 (2008): 1637-1645.

\bibitem[Popa]{Popa}
A.A. Popa, {\it
 Whittaker newforms for local representations of $GL(2)$}, Preprint.

\bibitem[RaRo2005]{RR}
D. Ramakrishnan and J.D. Rogawski, {\it Average values of
modular $L$-series via the relative trace formula},
Pure Appl. Math. Q. 1.4 (2005): 701-735.


\bibitem[Ro1983]{Ro1}
J.D. Rogawski, {\it Representations of $GL(n)$ and division algebras
over a $p$-adic field}, Duke Math. J 50.1 (1983): 161-196.



\bibitem[Rou2011]{Rouymi2011}
D. Rouymi, {\it Formules de trace et non-annulation de fonctions $L$ automorphes au niveau $p^\nu$}, Acta Arithmetica 147.1 (2011): 1-32.

\bibitem[Rou2012]{Rou} D. Rouymi, {\it Mollification et non annulation de fonctions $L$ automorphes en niveau primaire},
    J. Number Theory 132.1 (2012): 79-93.


\bibitem[Roy2001]{Roy} E. Royer, {\it Sur les fonctions L de formes modulaires}, Ph.D. thesis, Universit\'{e} de Paris-Sud (2001).

\bibitem[Su2015]{Su}
S. Sugiyama,
{\it Asymptotic behaviors of means of central values of automorphic $L$-functions for $GL(2)$},
Journal of Number Theory, 156 (2015): 195-246.

\bibitem[SuTs2016]{ST}
S. Sugiyama and M. Tsuzuki
{\it Relative trace formulas and subconvexity estimates of $L$-functions for Hilbert modular forms},
Acta Arithmetica, 176.1 (2016): 1-63.

\bibitem[Ye1989]{Ye}
Y. Ye, {\it Kloosterman integrals and base change for $GL(2)$},
J. reine angew. Math 400.57 (1989): 57-121.

\bibitem[Zh2004]{Zhang}
S.W. Zhang, {\it Gross-Zagier Formula for $GL(2)$, II},
 Heegner points and Rankin L-series, 191-214,
Math. Sci. Res. Inst. Publ., 49, Cambridge Univ. Press, Cambridge, 2004.



\end{thebibliography}
\end{document}
